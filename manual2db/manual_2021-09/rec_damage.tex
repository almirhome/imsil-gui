This record is used to describe damage related parameters. For a description of
many of the parameters see Section~\ref{s:dam}. The \texttt{\&DAMAGE} record has
the index variables \texttt{REGION}, \texttt{ATOM1}, and \texttt{ATOM2}.
\texttt{ATOM1} is the atom index of a projectile, \texttt{ATOM2} the index of the target atom
(recoil). Regarding the definition of atom indices, see Section~\ref{s:atom}.

\ifprivate
\begin{keydescription}{\texttt{AFTERMD} --- Read the displacement of atoms
after MD simulations}
%
  This parameter specifies that the deterministic defects file will be read 
  in format which contain displacement of atoms extracted after MD simulations. 
  Cell size  is limited on 10*10*2000 unit cells. Relevant only if  (\texttt{ldet=T}).
 
 \begin{keytab} 
 Type: \> logical \\ 
 Default: \> \texttt{F} \\ 
 Range: \>  \texttt{T}, \texttt{F}
  \end{keytab}
\end{keydescription}
\fi

\begin{keydescription}{\texttt{DAMAMO} --- Amorphization damage 
    concentration}
%
  This parameter specifies the threshold damage concentration for
  amorphization of the crystalline layer. If the damage concentration
  exceeds \texttt{DAMAMO} locally, every collision is treated as in an
  amorphous target. The units are cm$^{-3}$.
  \begin{keytab}
    Type:    \> real \\
    Default: \> $2.7 \times 10^{22}$ in Si unless target is read from a \texttt{CELL} file, 
                \texttt{DENSITY} otherwise \\
    Range:   \> $\texttt{DAMAMO} \ge 0$
  \end{keytab}
\end{keydescription}

\begin{keydescription}{\texttt{DAMSAT} --- Damage saturation concentration}
%
  This parameter specifies the damage saturation concentration. The damage
  concentrations may not exceed \texttt{DAMSAT} unless 
  $\texttt{DAMAMO} < \texttt{DAMSAT}$. The units are cm$^{-3}$.
  \begin{keytab}
    Type:    \> real \\
    Default: \> $\infty$ \\
    Range:   \> $> 0$ 
  \end{keytab}
\end{keydescription}

\begin{keydescription}{\texttt{DENSITY} --- Reference atom density}
%
  This parameter specifies the number of target atoms (irrespective of
  their species) per cm$^3$ to be used as a reference density for damage
  saturation and amorphization.  This means that in a monatomic target the
  interstitial and vacancy concentrations will asymptotically approach
  \texttt{DAMSAT} if the density equals \texttt{DENSITY}.  Likewise, if the
  interstitial or vacancy density exceeds \texttt{DAMAMO}, it will be set to
  \texttt{DENSITY}.
  \begin{keytab}
    Type:    \> real \\
    Default: \> density of the crystalline region if defined \\ 
             \> density of the only region if only one region is defined \\
             \> density of silicon, otherwise \\
    Range:   \> $> 0$
  \end{keytab}
\end{keydescription}

\begin{keydescription}{\texttt{DIBIIA} --- interval of energy
    deposition for IBIIA effect}
%
  This parameter specifies the size of the interval around an
  amorphous/crystalline interface which the energy deposition is
  averaged over. The units are \AA.
  \begin{keytab}
    Type:    \> real \\
    Default: \> 10 \\
    Range:   \> $> 0$ 
  \end{keytab}
\end{keydescription}

\ifprivate
\begin{keydescription}{\texttt{DIFFUSIVITY} --- radiation induced diffusivity}
%
  This parameter specifies the radiation induced diffusivity in units of 
  [cm$^2$/dpa]. The number of displacements per atom (dpa) is calculated from 
  the damage energy using the modified Kinchin-Pease model. Diffusivity is 
  only considered for the implanted species and only in dynamic simulations. 
  The radiation induced diffusivity equals $\frac{1}{6}\lambda^2 k$ 
  where $\lambda$ is the jump distance and $k$ the number of jumps per dpa.
  \begin{keytab}
    Type:    \> real \\
    Default: \> 0. \\
    Range:   \> $\ge 0$ 
  \end{keytab}
\end{keydescription}
\fi

\begin{keydescription}{\texttt{DNEAR} --- maximum near distance from the
    surface}
%
  This parameter specifies the maximum distance from the surface where recoils
  are followed down to energies lower than \texttt{EF} when \texttt{LSPUT=T}. 
  For distances from the surface less than \texttt{DNEAR}/2, they are followed 
  down to \texttt{ESURF}. For distances between \texttt{DNEAR}/2 and
  \texttt{DNEAR} the cutoff energy is interpolated linearly. Also, within 
  \texttt{DNEAR} from the surface virtual recoils are generated
  (see Section~\ref{s:sput}). Smaller values of \texttt{DNEAR} speed up
  the simulation, but too small values artificially reduce the sputtering 
  yield if $\texttt{ESURF} < \texttt{EF}$. The units are \AA.
  \begin{keytab}
    Type:    \> real \\
    Default: \> 30 \\
    Range: \> $\ge 0$
  \end{keytab}
\end{keydescription}

\ifprivate
\begin{keydescription}{\texttt{DPARELAX} --- Critical number of displacements 
per target atom for cell relaxation}
%
  This parameter specifies the minimum number of displacements per target atom  
  between cell relaxations which is required so that any of the cell
  corners may be moved during relaxation. In converting damage energy into 
  displacements per atom the modified Kinchin-Pease model is used. The units of 
  \texttt{DPARELAX} are 1. Note that the energy density accumulated depends on 
  the implanted dose between relaxations. This dose equals 
  \texttt{DOSE*NIONUPD/NION} with \texttt{DOSE} and \texttt{NION} from the 
  \texttt{IONS} record and \texttt{NIONUPD} from the \texttt{GEOM} record. 
  \texttt{DPARELAX} is only effective for \texttt{RELAXMODEL=1}.
  \begin{keytab}
    Type:    \> real \\
    Default: \> $10^{-3}$ \\
    Range:   \> $> 0$
  \end{keytab}
\end{keydescription}

\begin{keydescription}{\texttt{DRDEFLOW} --- Flow coefficient}
%
  This parameter specifies the flow coefficient. An additional displacement
  of size $\texttt{DRDEFLOW}(E_\texttt{EMIN})$ along the final direction of motion 
  is added to the trajectory end point. Here $E$ denotes the final projectile 
  energy. Only effective if \texttt{LFLOW=T}. The units of \texttt{DRDEFLOW} are
  \AA/eV.
  \begin{keytab}
    Type:    \> real \\
    Default: \> 0.2 \\
    Range:   \> $>0$
  \end{keytab}
\end{keydescription}
\fi

\begin{keydescription}{\texttt{ED} --- Displacement energy}
%
  This parameter specifies the displacement energy of atom
  \texttt{ATOM1} in pure material \texttt{ATOM2}.  The displacement energy
  of \texttt{ATOM1} in each region will be calculated by linear superposition
  with weights according to the abundance of each atom in the region. The
  units are eV. 
  \begin{keytab}
    Type:    \> index variable array (\texttt{ATOM2},\texttt{ATOM1}) 
                of real \\
    Default: \> 15 \\
    Range:   \> $> 0$
  \end{keytab}
\end{keydescription}

\begin{keydescription}{\texttt{ELIMKP} --- Energy limit for Kinchin
    Pease model} 
%
  This parameter specifies the energy limit in eV below which the
  recoil trajectories are terminated and damage is calculated according to the 
  modified Kinchin Pease model. As a cutoff energy, \texttt{ELIMKP} operates 
  in addition to \texttt{EF},
  i.e., trajectories are terminated if their energy is either less than 
  \texttt{ELIMKP} or \texttt{EF}. The default value of 0 thus means that 
  recoils are followed down to \texttt{EF}. \texttt{ELIMKP} is not applied to 
  externally started ions. It is set to zero if \texttt{LCAP=T} or 
  \texttt{LPOCK=T}. 
  \begin{keytab}
    Type:    \> real \\
    Default: \> 0 \\
    Range:   \> $\ge 0$
  \end{keytab}
\end{keydescription}

\begin{keydescription}{\texttt{EMELT} --- Heat of melting}
%
  This parameter specifies the energy per target atom in eV needed to
  melt the target. \texttt{EMELT} is only effective if
  \texttt{LPOCK=T}. The actual value to be used will usually be larger
  than the theoretical one since some energy will dissipate without
  generating a molten zone.
  \begin{keytab}
    Type:    \> real \\
    Default: \> 0.52 \\
    Range:   \> $> 0$
  \end{keytab}
\end{keydescription}

\ifprivate
\begin{keydescription}{\texttt{EMINFLOW} --- Minimum energy for flow}
%
  This parameter specifies the energy in eV below which no motion of the 
  projectiles as a result of ``flow'' is possible. It is only eefective if
  \texttt{LFLOW=T}. The units of \texttt{EMINFLOW} are eV. 
  \begin{keytab}
    Type:    \> real \\
    Default: \> 5 \\
    Range:   \> $0 < \texttt{EMINFLOW} \le \texttt{EF}$
  \end{keytab}
\end{keydescription}
\fi

\begin{keydescription}{\texttt{ESURF} --- Surface binding energy}
%
  This parameter specifies the surface binding energy of atom
  \texttt{ATOM1} in pure material \texttt{ATOM2}.  The surface binding energy
  of \texttt{ATOM1} in each region will be calculated by linear superposition
  with weights according to the abundance of each atom in the region. The
  units are eV.
  \begin{keytab}
    Type:    \> index variable array (\texttt{ATOM2},\texttt{ATOM1}) of real \\
    Default: \> \texttt{ESURF(ATOM1,ATOM1)}: heat of sublimation \\
             \> \texttt{ESURF(ATOM1,ATOM2)} = mean value of
                \texttt{ESURF(ATOM1,ATOM1)} and \\
             \> \texttt{ESURF(ATOM2,ATOM2)} \\
    Range:   \> $\ge 0$
  \end{keytab}
\end{keydescription}

\ifprivate
\begin{keydescription}{\texttt{FACFLOW} --- Flow factor for deposited energy}
%
  This parameter specifies a factor that determines an additional displacement
  along the final projectile direction. It is applied to the total energy
  deposited within a sphere of radius \texttt{RCAP}, to obtain the additional
  displacement. Only effective if \texttt{LFLOW=T} on the \texttt{\&SNPAR}
  record. The units of \texttt{FACFLOW} are \AA/eV.
  \begin{keytab}
    Type:    \> real \\
    Default: \> 0. \\
    Range:   \> $\ge 0$
  \end{keytab}
\end{keydescription}
\fi 

\begin{keydescription}{\texttt{FPOCK} --- Factor for amorphous pocket 
    recrystallization}
%
  This parameter specifies a factor used to calculate the reduction in
  amorphous pocket size due to athermal recrystallization during quenching of 
  the cascade.  It is related to a shrinking radius $\Delta r$ according to 
  $\texttt{FPOCK} = (4 \pi \texttt{DENS(IR)} / 3) ^ {1/3} \Delta r$, if it is 
  assumed that the amorphous pocket is spherical and shrinks by $\Delta r$
  independent of the pocket size.  \texttt{FPOCK} is only effective if
  \texttt{LPOCK=T}.

  \begin{keytab} 
     Type:    \> real \\ 
     Default: \> 1 \\ 
     Range:   \> $\ge 0$
  \end{keytab}
\end{keydescription}

\iffalse           % abandoned
\begin{keydescription}{\texttt{FRAC110} --- Split $\langle 110 \rangle$
    interstitial fraction}
%
  This parameter specifies the fraction of split $\langle 110 \rangle$
  interstitials among the interstitials. The fraction of split
  $\langle 110 \rangle$ interstitials among all displaced atoms is
  \texttt{FRAC110*FRACI}. The ramaining fraction \texttt{1-FRAC110} of
  interstitials is assumed at tetrahedral interstitial lattice sites
  during damage consideration in the crystalline layer.
  \begin{keytab}
    Type:    \> real \\
    Default: \> 0 \\
    Range:   \> $0 \le \texttt{FRAC110} \le 1$
  \end{keytab}
\end{keydescription}
\fi

\iffalse %abandoned
\begin{keydescription}{\texttt{FRACI} --- Interstitial fraction}
%
  This parameter specifies the fraction of interstitials among the
  displaced atoms. The interstitials are assumed at tetrahedral
  interstitial lattice sites during damage consideration in the
  crystalline layer, or they are assumed to be split $\langle 110
  \rangle$ interstitials (see \texttt{FRAC110}). All other displaced
  atoms are assumed at random locations (``amorphous'' atoms, see also
  \texttt{LAMO2}).
  \begin{keytab}
    Type:    \> real \\
    Default: \> 0 \\
    Range:   \> $0 \le \texttt{FRACI} \le 1$
  \end{keytab}
\end{keydescription}
\fi

\ifprivate
\begin{keydescription}{\texttt{FRACDI} --- Di-interstitial fraction}
%
  This parameter specifies the fraction of di-interstitials among the
  displaced atoms. It will be used for the population of the cell during the
  simulation of deterministic defects. Relevant only if
  \texttt{LDET=T}.
  \begin{keytab}
    Type:    \> real \\
    Default: \> 0 \\
    Range:   \> $0 \le \texttt{FRACDI} \le 1$
  \end{keytab}
\end{keydescription}

\begin{keydescription}{\texttt{FRACFOUR} --- Tetra-interstitial fraction}
%
  This parameter specifies the fraction of tetra-interstitials among
  the displaced atoms. It will be used for the population of the cell during
  the simulation of deterministic defects. Relevant only if
  \texttt{LDET=T}. 
  \begin{keytab}
    Type:    \> real \\
    Default: \> 0 \\
    Range:   \> $0 \le \texttt{FRACFOUR} \le 1$
  \end{keytab}
\end{keydescription}

\begin{keydescription}{\texttt{FRACHEX} --- Hexagonal interstitial fraction}
%
  This parameter specifies the fraction of hexagonal interstitials
  among the displaced atoms. It will be used for the population of the cell
  during the simulation of deterministic defects. Relevant only if
  \texttt{LDET=T}. 
  \begin{keytab}
    Type:    \> real \\
    Default: \> 0 \\
    Range:   \> $0 \le \texttt{FRACHEX} \le 1$
  \end{keytab}
\end{keydescription}
\fi

\begin{keydescription}{\texttt{FRACRG} --- Fraction of \texttt{FREC-1}
    generated during recoil generation}
%
  This parameter specifies the fraction of the additional Frenkel pairs
  generated due to \texttt{FREC} at the position of recoil generation
  with respect to the additional Frenkel pairs generated at both the
  position of recoil generation and at the position where the recoil
  comes to rest.
  \begin{keytab}
    Type:    \> real \\
    Default: \> 1 \\
    Range:   \> $0 \le \texttt{FRACRG} \le 1$
  \end{keytab}
\end{keydescription}

\ifprivate
\begin{keydescription}{\texttt{FRACSPL} --- Split-$\langle 110 \rangle$ interstitial fraction}
%
  This parameter specifies the fraction of split-$\langle 110 \rangle$
  interstitials among the displaced atoms. It will be used for
  the population of the cell during the simulation of deterministic defects.
  Relevant only if \texttt{LDET=T}. 
  \begin{keytab}
    Type:    \> real \\
    Default: \> 1 \\
    Range:   \> $0 \le \texttt{FRACSPL} \le 1$
  \end{keytab}
\end{keydescription}


\begin{keydescription}{\texttt{FRACTET} --- Tetragonal interstitial fraction}
%
  This parameter specifies the fraction of tetragonal interstitials among
  the displaced atoms. It will be used for the population of the cell during
  the simulation of deterministic defects. Relevant only if
  \texttt{LDET=T}. 
  \begin{keytab}
    Type:    \> real \\
    Default: \> 0 \\
    Range:   \> $0 \le \texttt{FRACTET} \le 1$
  \end{keytab}
\end{keydescription}

\begin{keydescription}{\texttt{FRACTRI} --- Tri-interstitial fraction}
%
  This parameter specifies the fraction of tri-interstitials among the
  displaced atoms. It will be used for the population of the cell during the
  simulation of deterministic defects. Relevant only if
  \texttt{LDET=T}. 
  \begin{keytab}
    Type:    \> real \\
    Default: \> 0 \\
    Range:   \> $0 \le \texttt{FRACTRI} \le 1$
  \end{keytab}
\end{keydescription}
\fi

\begin{keydescription}{\texttt{FREC} --- Recombination factor within
    recoil cascade}
%
  This parameter specifies a factor to be applied to the generated damage 
  concentrations.  It is the ratio of the number of stable defects to
  the number of defects calculated with the displacement energy before
  any recombination with defects from previous ions is taken into
  account.  Is set to unity if \texttt{LCAP=T} or \texttt{LPOCK=T}.
  \begin{keytab}
    Type:    \> real \\
    Default: \> \texttt{SPEC='B'}: 0.3 \\
             \> \texttt{SPEC='BF2'}: 1.2 \\
             \> \texttt{SPEC='SI','P'}: 0.8 \\
             \> \texttt{SPEC='AS'}: 2. \\
             \> otherwise: $0.0264 \times M_\mathrm{Ion}$ \\
    Range:   \> $> 0$
  \end{keytab}
\end{keydescription}

\begin{keydescription}{\texttt{KSURF} --- Nonplanarity parameter of surface
potential}
%
  This parameter specifies the nonplanarity parameter of the surface
  potential. \texttt{KSURF=0} corresponds to a planar surface potential,
  \texttt{KSURF=1} to an isotropic potential. \texttt{KSURF$<$1} means
  refraction away from the surface normal, \texttt{KSURF$>$1} towards the
  surface normal. \texttt{KSURF$>$0} corresponds to refraction at an ellipsoidal
  surface, while \texttt{KSURF$<$0} corresponds to refraction at a hyperboloidal
  surface. For projectiles entering the target, currently any value other than
  \texttt{KSURF=0} is treated like \texttt{KSURF=1}.
  \begin{keytab}
    Type:    \> real \\
    Default: \> 0. \\
    Range:   \> any real number
  \end{keytab}
\end{keydescription}

\begin{keydescription}{\texttt{LAMO2} --- 2-D amorphization flag}
%
  This parameter specifies how the positions of ``amorphous'' displaced atoms
  are calculated. If \texttt{LAMO2=T} an amorphous displaced atom is generated
  from a regular lattice atom by displacing it perpendicularly to the ion
  trajectory with a uniform probability distribution within the radius
  \texttt{PMAX} of the \texttt{\&CRYSTAL} record from the undeflected ion 
  trajectory. If \texttt{LAMO2=F} amorphous displaced atoms are generated at 
  random positions uniformly distributed in the lattice.

  \begin{keytab}
    Type:    \> logical \\
    Default: \> \texttt{T} \\
    Range:   \> \texttt{T}, \texttt{F}
  \end{keytab}
\end{keydescription}

\begin{keydescription}{\texttt{LAMZON} --- amorphous zone model flag}
%
  This parameter specifies whether the displaced atoms are combined into
  amorphous zones of the size of a histogram box (\texttt{LAMZON=T}), or
  considered individually (\texttt{LAMZON=F}). If the amorphous zone
  model is used, the probability of the occurence of an amorphous zone
  is chosen such that displaced atoms are encountered with the same
  overall probability as in the \texttt{LAMZON=F} case.
  \begin{keytab}
    Type:    \> logical \\
    Default: \> \texttt{F} \\
    Range:   \> \texttt{T}, \texttt{F}
  \end{keytab}
\end{keydescription}

\begin{keydescription}{\texttt{LCAP} --- Capture radius flag}
%
  This parameter specifies whether damage recombination within a recoil
  cascade is performed by recording the positions of all vacancies and
  interstitials and recombining those which are closer to each other
  than the capture radius \texttt{RCAP}.
  \begin{keytab}
    Type:    \> logical \\
    Default: \> \texttt{F} \\
    Range:   \> \texttt{T}, \texttt{F}
  \end{keytab}
\end{keydescription}

\begin{keydescription}{\texttt{LDAM} --- Damage flag}
%
  This parameter specifies whether a refined damage model is used, and whether
  damage is considered in the crystalline layer (if any) for the trajectories 
  of subsequent ions and recoils.  For \texttt{LDAM=F} and \texttt{LRCOIL=F}
  no damage is recorded. For \texttt{LDAM=F} and \texttt{LRCOIL=T} interstitial 
  and vacancy distributions are recorded, where one interstitial is generated 
  when and where a projectile stops and one vacancy is generated when and where
  a target atom is recoiled. For \texttt{LDAM=T} and \texttt{LRCOIL=F} the 
  amount of damage is calculated according to the modified Kinchin Pease model, 
  it can be considered as the density of dispaced atoms or Frenkel pairs. For 
  \texttt{LDAM=T} and \texttt{LRCOIL=T} one of the refined damage models is
  used: Default is the model described in Section~\ref{s:dam}, while when
  \texttt{LCAP=T}\ifprivate or \texttt{LDET=T} \fi, the capture radius model
  \ifprivate or the deterministic defect model (see Section~\ref{s:det}) \fi
  is invoked\ifprivate, respectively\fi. For \texttt{LDAM=T}, damage is
  considered in the selection of collision partners in a crystalline layer.  Note that it is not possible to 
  use one of the refined damage models without considering damage for
  subsequent trajectories in a crystalline layer. If this is desired, use a
  very small implantation dose (\texttt{DOSE} of the \texttt{\&ION} record).
  Note that moments and histograms of the damage distributions are only output
  if the appropriate flags are set on the \texttt{\&OUTPUT} record.  Nuclear
  energy deposition is calculated independent of \texttt{LDAM}.
  \begin{keytab}
    Type:    \> logical \\
    Default: \> \texttt{.NOT.LDYN} \\
    Range:   \> \texttt{T}, \texttt{F}
  \end{keytab}
\end{keydescription}

\ifprivate
\begin{keydescription}{\texttt{LDET} --- Deterministic defects flag}
%
  This parameter specifies whether damage is considered deterministically as
  given by atom coordinates specified in a column defined by \texttt{WCOL},
  using periodic boundary conditions outside the column.  If \texttt{LDET=F},
  defects are generated statistically using the damage histograms.  The
  deterministic defects may either be read from the file \texttt{DEFFILE} or
  generated from damage profiles read from the \texttt{HISFILE} and
  \texttt{BKFILE} files. See also \texttt{USEHIS}, \texttt{USEBK} and
  \texttt{USEDEF} in the \texttt{\&SETUP} record.
  \begin{keytab}
    Type:    \> logical \\
    Default: \> \texttt{F} \\
    Range:   \> \texttt{T}, \texttt{F}
  \end{keytab}
\end{keydescription}

\begin{keydescription}{\texttt{LFLOW} --- Recoil flow flag}
%
  This parameter specifies whether additional recoil flow is considered for
  projectile energies between \texttt{EF} and \texttt{EMIN}. See also 
  \texttt{DRDE}.
  \begin{keytab}
    Type:    \> logical \\
    Default: \> \texttt{F} \\
    Range:   \> \texttt{T}, \texttt{F}
  \end{keytab}
\end{keydescription}
\fi

\begin{keydescription}{\texttt{LIBIIA} --- IBIIA flag}
%
  This parameter specifies whether amorphous zones may grow according
  ion beam induced interfacial amorphization (IBIIA).
  \begin{keytab}
    Type:    \> logical \\
    Default: \> \texttt{F} \\
    Range:   \> \texttt{T}, \texttt{F}
  \end{keytab}
\end{keydescription}

\begin{keydescription}{\texttt{LPOCK} --- Amorphous pocket flag}
%
  This parameter specifies whether amorphous pockets are formed based on
  the parameters \texttt{EMELT} and \texttt{EPOCK}. Currently amorphous
  pockets are not used in subsequent trajectory simulations, but have
  only the effect of increased damage. Currently \texttt{LPOCK=T} must not
  be specified at the same time as \texttt{LVAC=T}.
  \begin{keytab}
    Type:    \> logical \\
    Default: \> \texttt{F} \\
    Range:   \> \texttt{T}, \texttt{F}
  \end{keytab}
\end{keydescription}

\begin{keydescription}{\texttt{LRCOIL} --- Recoil cascade flag}
%
  This parameter specifies whether the recoil trajectories are to be
  simulated. Is set true if \texttt{LCAP=T} or \texttt{LPOCK=T}. \texttt{LRCOIL}
  also determines how damage is calculated (see \texttt{LDAM}). Note that
  \texttt{LRCOIL} overwrites both \texttt{LRCOIL1} and \texttt{LRCOIL2} with its
  value if specified in the input file.
  \begin{keytab}
    Type:    \> logical \\
    Default: \> \texttt{F} \\
    Range:   \> \texttt{T}, \texttt{F}
  \end{keytab}
\end{keydescription}

\begin{keydescription}{\texttt{LRCOIL1} --- Recoil cascade before reentering
flag}
%
  This parameter specifies whether projectiles with the reentered flag unset can
  generate recoils whose trajectories are followed. The reentered flag is
  set if a projectile or one of its parents has been emitted from a surface
  and has reentered the target at a different point in space. The reentered flag
  is always unset in a cartesian 1D simulation. Note that for recoils to be
  followed at all, also \texttt{LRCOIL=T} is required. The purpose of
  \texttt{LRCOIL1} is to provide the possibility of switching off recoil
  generation before reentering while keeping recoil generation after by setting
  \texttt{LRCOIL=T}, \texttt{LRCOIL1=F}, \texttt{LRCOIL2=T}.
  \begin{keytab}
    Type:    \> logical \\
    Default: \> \texttt{T} \\
    Range:   \> \texttt{T}, \texttt{F}
  \end{keytab}
\end{keydescription}

\begin{keydescription}{\texttt{LRCOIL2} --- Recoil cascade after reentering
flag}
%
  This parameter specifies whether projectiles with the reentered flag set can
  generate recoils whose trajectories are followed. The reentered flag is
  set if a projectile or one of its parents has been emitted from a surface
  and has reentered the target at a different point in space. The reentered flag
  is always unset in a cartesian 1D simulation. Note that for recoils to be
  followed at all, also \texttt{LRCOIL=T} is required. The purpose of
  \texttt{LRCOIL2} is to provide the possibility of switching off recoil
  generation after reentering while keeping recoil generation before by
  setting \texttt{LRCOIL=T}, \texttt{LRCOIL1=T}, \texttt{LRCOIL2=F}.
  \begin{keytab}
    Type:    \> logical \\
    Default: \> \texttt{T} \\
    Range:   \> \texttt{T}, \texttt{F}
  \end{keytab}
\end{keydescription}

\begin{keydescription}{\texttt{LREPL} --- Replacement collision flag}
%
  This parameter specifies whether replacement collisions are allowed.
  Replacement collision are collisions where the projectile stops and the 
  recoil moves on, provided projectile and recoil are the same species. In
  a replacement collision no defects are generated, while if the same collision
  is treated without replacement, a Frenkel pair is created.
  \begin{keytab}
    Type:    \> logical \\
    Default: \> \texttt{T} \\
    Range:   \> \texttt{T}, \texttt{F}
  \end{keytab}
\end{keydescription}

\ifprivate
\begin{keydescription}{\texttt{LSTRMD} --- Molecular dynamics strain}
%
  This parameter specifies whether the coordinates of defects and strained
  regions extracted from molecular dynamics simulations to be used.  It will
  be used for the population of the cell during the simulation of deterministic
  defects. Relevant only if \texttt{LDET=T}.

  \begin{keytab} 
    Type: \> logical \\ 
    Default: \> \texttt{F} \\ 
    Range: \>  \texttt{T}, \texttt{F}
  \end{keytab}
\end{keydescription}
\fi

\iffalse                       % eliminated
\begin{keydescription}{\texttt{LVAC} --- Vacancy flag}
%
  This parameter specifies whether vacancy distributions have to be
  recorded.  If \texttt{LVAC=F} no vacancy distributions are provided,
  and the vacancy distribution is assumed to be identical to the
  interstitial distribution for internal damage consideration.
  \begin{keytab}
    Type:    \> logical \\
    Default: \> \texttt{F} \\
    Range:   \> \texttt{T}, \texttt{F}
  \end{keytab}
\end{keydescription}
\fi

\ifprivate
\begin{keydescription}{\texttt{MODBULK} --- Bulk modulus}
%
  This parameter specifies the bulk modulus used in the 
  viscoelastic relaxation model. The units of \texttt{MODBULK} are Pa.
  Ignored unless \texttt{RELAXMODEL='viscoelastic'}.
  \begin{keytab}
    Type:    \> real \\
    Default: \> $4.8 \times 10^{10}$ \\
    Range:   \> $>0$
  \end{keytab}
\end{keydescription}

\begin{keydescription}{\texttt{MODSHEAR} --- Shear modulus}
%
  This parameter specifies the shear modulus used in the 
  viscoelastic relaxation model. The units of \texttt{MODSHEAR} are Pa.
  Ignored unless \texttt{RELAXMODEL='viscoelastic'}.
  \begin{keytab}
    Type:    \> real \\
    Default: \> $3.3 \times 10^{10}$ \\
    Range:   \> $>0$
  \end{keytab}
\end{keydescription}
\fi

\iffalse                       % this was never completed and thus abandoned
\begin{keydescription}{\texttt{NCLUS} --- Cluster size}
%
  This parameter specifies the number of atoms per cluster.
  \texttt{NCLUS} random interstitials are assumed to be clustered to
  one cluster each. The cluster is assumed to be a cylinder around the
  ion trajectory with radius \texttt{PMAX}. \texttt{NCLUS} may only be
  specified with \texttt{LAMO2=T}.
  \begin{keytab}
    Type:    \> integer \\
    Default: \> 1 \\
    Range:   \> $\ge 1$ \\
    Remark:  \> not yet implemented
  \end{keytab}
\end{keydescription}
\fi

\begin{keydescription}{\texttt{NDIM} --- Dimensionality of damage histogram}
%
  This parameter specifies the dimensionality of the histogram from which
  the damage concentration is read for consideration in the crystalline 
  layer (if defined) for the trajectories of subsequent ions and recoils.
  Note that not all dimensionalites are possible depending on the choices 
  of \texttt{DOSEUNITS} and \texttt{XINIT} and \texttt{YINIT} of the
  \texttt{\&IONS} record (see Section XXX). The default values represent the
  largest values possible. It is necessary to demand output of the respective
  damage histogram by setting \texttt{LHIS=T}, \texttt{LHIS2=T}, or
  \texttt{LHIS3=T} on the \texttt{\&OUTPUT} record, depending on \texttt{NDIM}. 
  \begin{keytab}
    Type:    \> integer \\
    Default: \> 1 for \texttt{DOSEUNITS='cm-2'} and 
                      \texttt{XINIT(1)=XINIT(2)}, \\
             \> 2 for \texttt{DOSEUNITS='cm-2'} and \\
             \> \ \ \ \ \ \ \ \ \ \texttt{XINIT(1)>XINIT(2)} and 
                        \texttt{YINIT(1)=YINIT(2)}, \\
            \> \ \ \ \ or \texttt{DOSEUNITS='cm-1'} and
                      \texttt{YINIT(1)=YINIT(2)}, \\
             \> 3 otherwise \\
    Range:   \> $1 \le \texttt{NDIM} \le 3$ and not larger than the default
                values
  \end{keytab}
\end{keydescription}

\begin{keydescription}{\texttt{NPOCKMIN} --- Minimum number of atoms
    in pocket}
%
  This parameter specifies the minimum number of atoms in an amorphous
  pocket.  If the energy of a tentative amorphous pocket is such that
  less than \texttt{NPOCKMIN} atoms are contained after quenching,
  separated point defects are assumed to be generated instead.
  \texttt{NPOCKMIN} is only effective if \texttt{LPOCK=T}.
  \begin{keytab}
     Type:    \> integer \\
     Default: \> 10 \\
     Range:   \> $\ge 0$
  \end{keytab}
\end{keydescription}

\ifprivate
\begin{keydescription}{\texttt{PFACFLOW} --- Flow factor for deposited momentum}
%
  This parameter specifies a factor that determines an additional displacement
  proportional to the total momentum deposited within a sphere of radius 
  \texttt{RCAP}. Only effective if \texttt{LFLOW=T} on the \texttt{\&SNPAR}
  record. The units of \texttt{FACFLOW} are \AA/$\sqrt{\mathrm{eV}}$.
  \begin{keytab}
    Type:    \> real \\
    Default: \> 0. \\
    Range:   \> $\ge 0$
  \end{keytab}
\end{keydescription}
\fi

\begin{keydescription}{\texttt{PSI110} --- Critical angle around
    $\langle 110 \rangle$}
%
  This parameter specifies a critical angle with respect to 
  $\langle 110 \rangle$ directions for application of the modified Kinchin-Pease
  model.  The modified Kinchin-Pease model is not applied if the angle between 
  the direction of motion and a $\langle 110 \rangle$ direction is less than 
  \texttt{PSI110}.  The units of \texttt{PSI110} are
  degrees.
  \begin{keytab}
    Type:    \> real \\
    Default: \> 0 \\
    Range:   \> $0 \le \texttt{PSI110} \le 180$
  \end{keytab}
\end{keydescription}

\begin{keydescription}{\texttt{RCAP} --- Capture radius}
%
  This parameter specifies the capture radius to be used if \texttt{LCAP=T}
  or \texttt{LFLOW=T}.
  The units of \texttt{RCAP} are \AA.
  \begin{keytab}
    Type:    \> real \\
    Default: \> $(3/8 \pi \texttt{DAMSAT})^{1/3}$ \\
    Range:   \> $>0$
  \end{keytab}
\end{keydescription}

\begin{keydescription}{\texttt{REENTER} --- Reenter flag for recoils}
%
  This parameter specifies whether recoils are allowed to traverse 
  vacuum regions and to reenter the target somewhere else. If
  \texttt{REENTER=F}, the recoil trajectories are terminated as soon
  as they leave the target, and the recoils are counted as backscattered or
  transmitted depending on their direction of motion. \texttt{REENTER=T} 
  must not be specified in a 1-D layer simulation (\texttt{NDIM=1},
  \texttt{LCYLX=LCYLY=LCYLZ=LSPHER=F}).
  \begin{keytab}
    Type:    \> logical \\
    Default: \> \texttt{NDIM>1.OR.LCYLX.OR.LCYLY.OR.LCYLZ.OR.LSPHER} \\
    Range:   \> \texttt{T}, \texttt{F}
  \end{keytab}
\end{keydescription}

\ifprivate
\begin{keydescription}{\texttt{RELAXMODEL} --- Relaxation model}
%
  This parameter specifies the model used for target relaxation in a
  dynamic simulation. \texttt{RELAXMODEL='full'} specifies full relaxation
  of the target densities along a steepest descent path of the square of the 
  deviation of cell volumes from the ideal values. 
  \texttt{RELAXMODEL='viscoelastic'} specifies a viscoelastic model with
  radiation induced viscosity.
  \begin{keytab}
    Type:    \> character(80) \\
    Default: \> \texttt{'none'} \\
    Range:   \> \texttt{'none'}, \texttt{'full'}, \texttt{'viscoelastic'}
  \end{keytab}
\end{keydescription}
\fi

\begin{keydescription}{\texttt{RIBIIA} --- IBIIA growth rate constant}
%
  This parameter specifies the displacement of the
  amorphous/crystalline interface per deposited nuclear energy
  density for the IBIIA effect. Its units are \AA /(eV/\AA $^3$).
  \begin{keytab}
    Type:    \> real \\
    Default: \> 100 \\
    Range:   \> $> 0$ 
  \end{keytab}
\end{keydescription}

\ifprivate
\begin{keydescription}{\texttt{VISCOSITY} --- Radiation induced viscosity}
%
  This parameter specifies the radiation induced viscosity used in the 
  viscoelastic relaxation model. The units of \texttt{VISCOSITY} are Pa dpa.
  The number of displacements per atom (dpa) is calculated from the damage 
  energy using the modified Kinchin-Pease model. Ignored unless 
  \texttt{RELAXMODEL='viscoelastic'}.
  \begin{keytab}
    Type:    \> real \\
    Default: \> $0.966 \times 10^9$ \\
    Range:   \> $>0$
  \end{keytab}
\end{keydescription}

\begin{keydescription}{\texttt{WCOL} --- Column width}
%
  This parameter specifies the column width to be used with
  determinstic defects \texttt{LDET=T}. The defects are restricted to a
  columnar domain with dimensions \texttt{WCOL(1)*WCOL(2)}. The starting
  points of the ion trajectories are randomly distributed in the intersection
  of the surface with the column, but the collision cascade is allowed to
  develop in the target. Defects around the ions and recoil trajectories are
  generated by assuming periodicity in lateral dimensions.  If
  \texttt{USEHIS=T} or \texttt{USEBK=T} in the \texttt{\&SETUP} record the
  column will be populated with defects placed at the random positions
  according to histograms defined in the \texttt{HISFILE} or \texttt{BKFILE} file.
  Otherwise the column width, together with the defect positions is to be defined
  in \texttt{DEFFILE} of the \texttt{\&SETUP} record. The units are a quarter
  lattice constant.
  \begin{keytab} 
    Type: \> simple array (3) of integer \\ 
    Default: \> 40,40,8000 \\
    Range: \> $(4i,4j,4k)>0$, where $i$, $j$, $k$ are integers \\ 
  \end{keytab}
\end{keydescription}
\fi

\begin{keydescription}{\texttt{ZMAXKP} --- Maximum depth for Kinchin
    Pease model}
%
  This parameter specifies a maximum depth at which the modified
  Kinchin Pease model is invoked. For depth larger than
  \texttt{ZMAXKP} the Kinchin Pease model is not applied.  The units
  of \texttt{PSI110} are degrees.
  \begin{keytab}
    Type:    \> real \\
    Default: \> $\infty$ \\
    Range:   \> $\ge 0$
  \end{keytab}
\end{keydescription}

