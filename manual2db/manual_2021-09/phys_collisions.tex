\ifprivate
Aside from being a binary collision simulator, IMSIL also has a
simplified molecular dynamics mode which treats the ion target
interaction by integration of the equations of motion of the moving
particles. It is simplified inasmuch as only interations between the
ion or recoil and the target atoms are taken into acoount and
interactions among the target atoms are disregarded. Also, it is
restricted to simulations in a single crystalline layer, and no
lattice damage may be taken into account. Nevertheless, the user is
given access to this mode by the \texttt{LMD} flag. There are only a
few parameters specific to the MD mode, such as the accuracy of time
integration, which are hard-coded. For more information and
applications see \cite{I9701,I0101}. 
\fi

In the binary collision approximation, the collisions are treated as if the ions
interacted with only one target atom at a time.  In \textbf{amorphous} targets,
the algorithm used is similar to that implemented in TRIM \cite{I8001}, however,
with a different choice of the maximum impact parameter. $p_\mathrm{max}$ is
calculated depending on energy and ion and target atom species, based on two
criteria \cite{hobler_acceleration_1995}. The energy transfer criterion
guarantees that scattering events with energy transfers larger than a
user-specified value $T_\mathrm{min}$ are taken into account. The scattering
angle criterion guarantees that all scattering events with scattering angles
larger than the user-specified value $\psi_\mathrm{min}$ are taken into account.
The resulting  $p_\mathrm{max}$ values are limited by $p_\mathrm{max,max}$ and
$p_\mathrm{max,\psi}$, respectively. The larger of the maximum impact parameters
determined from the two criteria is taken. In addition, the impact parameter is
forced to be at least $p_\mathrm{max,min}$. It is possible to use a fixed
maximum impact parameter by setting all of $p_\mathrm{max,max}$,
$p_\mathrm{max,\psi}$, and $p_\mathrm{max,min}$ to the desired value. 

The user can choose between a deterministic and a random free flight path. The
latter assumes an exponential probability density function with a mean value
equal to the deterministic free flight path. The (mean) free flight path
$\overline{L}$ is calculated from the relation 
$\overline{L} \cdot \pi p_{max}^2 = N^{-1}$, which considers the correct atomic 
density $N$ of the target. As usual \cite{I8001}, the square of the impact
parameter $p$ is selected randomly with a uniform distribution between 0 and 
$p_{max}^2$, and the azimuthal position of the target atom with respect to the 
ion path is chosen randomly between 0 and $2\pi$. 

In a \textbf{crystalline} target region, after each collision, the next target
atom to be hit has to be selected from the lattice positions surrounding the 
ion. This is done among all lattice sites that lead to impact parameters
$p < p_\mathrm{max,c}$ and free flight paths $0 < L < L_\mathrm{max,c}$. In
addition, to limit the size of the search cell, only atoms with a distance from
the projectile less than the search distance $r_\mathrm{min,c}$ are guaranteed
to be checked. For consistency, the following condition must hold:
%
\begin{equation}
   L_\mathrm{max,c}^2 + p_\mathrm{max,c}^2 \le r_\mathrm{min,c}^2   
   \label{eq:search}
\end{equation}
%
For the diamond crystal structure, it is usually not necessary to change the
default values of these parameters. If it is done, it is important to obey
Eq.~(\ref{eq:search}). For other crystal structures, it might be essential to
set $p_\mathrm{max,c}$ in the input file. With larger $p_\mathrm{max,c}$, more
distant collision partners are taken into account, which might increase accuracy
at low energies on the expense of increasing computation times. Too small
$p_\mathrm{max,c}$ might introduce inaccuracies into the simulation. For crystal
structures other than the default diamond structure, $r_\mathrm{min,c}$ is
calculated by the program. The choice of $L_\mathrm{max,c}$ is uncritical, it
only affects the computational efficiency.

The default values of $p_\mathrm{max,\psi}$ and $p_\mathrm{max,c}$ are implant
energy dependent for energies less than $10$~keV in order to allow more accurate
simulation of low-energy implantations.

Small displacements from the lattice positions due to thermal vibrations are
taken into account. These are assumed to be uncorrelated and distributed
according to a Gaussian function. The RMS displacement in one direction,
$x_\mathrm{rms}$, may be calculated using the Debye theory \cite{I7403}. A value
of 490~K is recommended for the Debye temperature of Si
\cite{hobler_role_1993,hobler_verification_1996}. The values of the Debye
temperatures for all elements are tabulated within IMSIL and appropriate
vibrational amplitudes are calculated using the Debye function \cite{I7403}.

It has been shown that ``simultaneous collisions'' are not beneficial
for dopant profile calculation \cite{hobler_study_1997,
hobler_modeling_1997,hobler_useful_2001}. ``Simultaneous collisions'' refer to
an algorithm which treats collisions with several target atoms simultaneously,
when the collisions are separated by a free flight path smaller than a (small)
value $L_\mathrm{sim}$ \cite{I7401}. This treatment can only be approximate,
since the three-body problem cannot be solved exactly, as is known from
classical mechanics. To our experience, no improvement can be obtained by using
``simultaneous collsions''. This feature has therefore been eliminated from
IMSIL.  The quality of the treatment of collisions in crystalline targets is
also supported by the good results obtained in round robin simulations
\cite{gartner_round_1995} and other comparisons with MD simulations
\cite{schlueter_absence_2020,hobler_channeling_2019}.

The motion of the ions near \textbf{interfaces} between regions and between
a region and vacuum is carefully treated. For this purpose, when the ion moves
near to an interface, potential collision partners from both adjoining regions
are considered.  This is particularly important for dynamic simulations
(\texttt{LDYN=T}) where the composition and density of the target may change
from cell to cell with the cells typically being quite small.  In a dynamic
simulation, the implanted ions are recorded and taken into account as potential
collision partners for subsequent ions and recoils.
%%% we need a more extended description of the dynamic mode, otherwise
%%% it might be difficult to set up an input file

\begin{center}
\begin{tabular}{lll}
parameter \quad    & IMSIL name     & to be specified in record \\
\hline
\ifprivate
MD mode flag            & \texttt{LMD}  & \texttt{\&SETUP} \\
\fi
dynamic simulation flag & \texttt{LDYN}     & \texttt{\&SETUP} \\
$\psi_\mathrm{min}$     & \texttt{PSIFFP}   & \texttt{\&SNPAR} \\
$T_\mathrm{min}$        & \texttt{DENFFP}   & \texttt{\&SNPAR} \\
$p_\mathrm{max,min}$    & \texttt{PMAXMIN}  & \texttt{\&SNPAR} \\  
$p_\mathrm{max,max}$    & \texttt{PMAXMAX}  & \texttt{\&SNPAR} \\  
$p_\mathrm{max,\psi}$   & \texttt{PMAXPSI}  & \texttt{\&SNPAR} \\  
$N$                     & \texttt{DENSITY}  & \texttt{\&MATERIAL} \\
$p_\mathrm{max,c}$      & \texttt{PMAX}     & \texttt{\&CRYSTAL} \\
$L_\mathrm{max,c}$      & \texttt{FFPMAX}   & \texttt{\&CRYSTAL} \\
$r_\mathrm{min,c}^2-1$  & \texttt{NRAD2}    & \texttt{\&CRYSTAL} \\
$x_\mathrm{rms}$        & \texttt{VIB}      & \texttt{\&MATERIAL} \\

\end{tabular}
\end{center}
