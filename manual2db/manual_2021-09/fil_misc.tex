\bigskip
%
\begin{center}
\begin{tabular}{|l|p{0.5\textwidth}|l|l|}
\hline
Name      & File content                             & flag          & record \\
\hline
\tt BCK   & Backup = data necessary to continue the simulation 
                                                     & \tt NIONBK    & \tt \&OUTPUT \\
\ifprivate
\tt DEF   & Deterministic defects output             & \tt LDEF      & \tt \&OUTPUT \\
\fi
\tt SE    & table of stopping powers                 & \tt LTAB      & \tt \&SEPAR \\
\tt PMAX  & maximum impact parameters                & \tt LPMAXTAB  & \tt \&SNPAR \\
\tt TRA   & Information about the trajectories       & \tt LTRA,ITRA & \tt \&OUTPUT \\
\hline
\end{tabular}
\end{center}

\bigskip

The backup file \texttt{BCK} contains all information about statistic quantities
at the end of the simulation. It may be loaded by another simulation specifying
\texttt{USEBK=T} on the \texttt{\&SETUP} record. In new simulations, the backup
file should only be used for debugging purposes, see the remark towards the end
of Section~\ref{s:input}.

\ifprivate
The deterministic defect file \texttt{DEF} is written by the following FORTRAN
statements:
%
\begin{verbatim}
      WRITE (43,*) ICOLMIN(1:3),ICOLMAX(1:3)
      WRITE (43,*) NLSDEF
      DO, ILS=1,NLSDEF
         WRITE (43,*) IPOS(1:3,ILS)
         WRITE (43,*) TYPE(ILS)
         IF (TYPE(ILS)==-1) THEN
            WRITE (43,*) NATOM(ILS)
            DO, IA=1,NATOM(ILS)
               WRITE (43,*) Z(IA,ILS),DPOS(1:3,IA,ILS)
            END DO
         ELSE
            WRITE (43,*) IOR(ILS),Z(1:NATOM(ILS),ILS)
         END IF
      END DO
\end{verbatim}
%
It is only intended to be written if the default silicon lattice is used.
\texttt{ICOLMIN}, \texttt{ICOLMAX} denote the lower/upper $x$, $y$, and $z$
coordinates of the simulation box containing the defects. Coordinates
are given in quarter lattice constants. \texttt{NLSDEF} denotes the number of
lattice site defects. \texttt{IPOS(:,ILS)} are the lattice site coordinates of
the lattice site defect \texttt{ILS} given in the crystal coordinate system,
i.e., the coordinate system connected with the [100], [010], and [001] axis of
the crystal. \texttt{TYPE(ILS)} denotes the defect type (see table below),
\texttt{NATOM(ILS)} the number of atoms at the lattice site, \texttt{Z(IA,ILS)}
is the atomic number and \texttt{DPOS(:,IA,ILS)} the displacement vector from
the lattice site of the \texttt{IA}-th atom at the \texttt{ILS}-th lattice site.
\texttt{IOR(ILS)} is an integer number between 1 and 24 indexing the symmetry
equivalent orientations of the lattice.

Possible defect types are:
%
\bigskip
%
\begin{center}
\begin{tabular}{|c|p{0.5\textwidth}|}
\hline
Defect type & Defect name                               \\
\hline
\tt -1      & Arbitrary configuration                   \\
\tt 1       & Vacancy                                   \\
\tt 2       & Split-$\langle 110 \rangle$ interstitial  \\
\tt 3       & Di-interstitial                           \\
\tt 4       & Tri-interstitial                          \\
\tt 5       & Hexagonal interstitial                    \\
\tt 6       & Tetrahedral interstitial                  \\
\tt 7       & Four interstitial 1. lattice site         \\
\tt 8       & Four interstitial 2. lattice site         \\
\tt 9       & Four interstitial 3. lattice site         \\
\tt 10      & Four interstitial 4. lattice site         \\
\hline
\end{tabular}
\end{center}

\bigskip
\fi

The electronic stopping power file \texttt{SE} lists for each region, ion atom
species, and target atom species the electronic energy loss as a function of
energy. Each block is written by the following FORTRAN statements:
%
\begin{verbatim}
      WRITE (122,*) 3,101
      WRITE (122,*) 0,0,0
      DO, IE=0,100
         WRITE (122,*) E(IE),DEDXNL(IE),DEELOC(IE)
      ENDDO
\end{verbatim}
%
\texttt{E} denotes the energy [eV], which is logarithmically distributed between
the implant energy (\texttt{ENERGY} parameter of the \texttt{\&IONS} record) and
the cut-off energy for ion trajectories (\texttt{EF} parameter of the
\texttt{SNPAR} record). \texttt{DEDXNL} is the nonlocal energy loss per path
length [eV/\AA], and \texttt{DEELOC} the local energy loss [eV] at the impact
parameter specified by \texttt{PTAB} on the \texttt{\&SEPAR} record.

The impact parameter file \texttt{PMAX} is written by the following FORTRAN statements:
%
\begin{verbatim}
      WRITE (122,*) 4,1000
      WRITE (122,*) 0,0,0,0
      DO, IE=1,1000
         WRITE (122,*) E(IE),POUT(IE),PNEAR(IE),PIN(IE)
      ENDDO
\end{verbatim}
%
\texttt{E} denotes the energy [eV], which is logarithmically distributed between
the implant energy \texttt{ENERGY} and the cut-off energy for ion trajectories
\texttt{EF}. \texttt{POUT} is the maximum impact parameter used outside the
target. \texttt{PNEAR} is the maximum impact parameter used inside the target at
distances from the surface smaller than \texttt{DNEAR} (see
Section~\ref{s:sput}). \texttt{PIN} is the maximum impact parameter used inside
the target at distances from the surface larger than \texttt{DNEAR}. The column
for \texttt{POUT} is not written unless \texttt{PSIFFP2} is set (see Section
\ref{s:snpar}). The column for \texttt{PNEAR} is not written if
\texttt{LSPUT=F}. If not all columns are written, the first two lines of the
file are modified to reflect the reduced number of columns. Impact pararameter
tables are written one after the other to the \texttt{PMAX} file for all regions
(in case of a non-dynamic simulation) and all atom species.

The trajectory file \texttt{TRA} contains information about the trajectories
depending on the variables \texttt{LTRA} and \texttt{ITRA} of the
\texttt{\&OUTPUT} record. Each line is written by the following FORTRAN
statement:
%
\begin{verbatim}
   WRITE (23,*) X,Y,Z,DIRX,DIRY,DIRZ,E,I1,IG,IFLAG
\end{verbatim}
%
\texttt{X}, \texttt{Y}, and \texttt{Z} denote the coordinates in units of [\AA],
\texttt{DIRX}, \texttt{DIRY}, \texttt{DIRZ} the direction vector
(\texttt{DIRX**2+DIRY**2+DIRZ**2=1}), and \texttt{E} the energy of the
projectile in units of [eV]. \texttt{I1} is the atom index of the projectile,
and \texttt{IG} the index of the recoil generation (\texttt{IG=1} for the ion,
\texttt{IG=2} for the primary recoils, \texttt{IG=3} for the secondary recoils
...).  In case of a virtual recoil (see Section \ref{s:sput}), \texttt{I1} is
replaced by \texttt{-I1}. \texttt{IFLAG} indicates the position of the point in
the trajectory:
%
\bigskip
%
\begin{center}
\begin{tabular}{|c|p{0.65\textwidth}|}
\hline
\texttt{IFLAG} & Position in the trajectory                               \\
\hline
0              & no trajectory (recoil with subthreshold energy or when \texttt{LRCOIL=F}) \\
-1, 1          & at the start of a trajectory (where a vacancy is generated) \\
2              & along a trajectory \\
3              & at the end of a trajectory inside the target (where an interstitial is 
                 generated) \\
4              & at the end of the trajectory of a backscattered ion or recoil \\
5              & at the end of the trajectory of a transmitted ion or recoil \\
\hline
\end{tabular}
\end{center}

\texttt{IFLAG=-1} writes the position before lattice vibrations and shifts due
to the time integral. \texttt{IFLAG=1} writes the position after consideration
of these two shifts. Note that there is a function in the
\texttt{read\_output.py} script for reading trajectory files (see
Section~\ref{s:scripts_read}).
