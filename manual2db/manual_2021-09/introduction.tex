Ion implantation has traditionally been the most important doping technique in
semiconductor technology. One-dimensional doping profiles of the most common ion species 
implanted into (100) Si at 7$^\circ$ tilt angle have accurately and extensively 
been measured and parameterized.  On the other hand, physics-based simulations 
may be invaluable if less common implant parameters are used (such as large tilt
angles or different materials), if the dopant profiles are difficult to
measure (such as at ultra-low energies or in two dimensions), or if
information on implant damage is desired. 

Another related phenomenon is sputtering. Sputtering can occur unintentionally
during high-dose ion implantation or intentionally to remove material, possibly
in a local manner using a focused ion beam. Other related phenomena occur in
nuclear technology such as defect creation in fission and plasma--wall 
interaction in fusion technology. In all these situations the binary
collision (Monte Carlo) IMplant and Sputter sImuLator IMSIL\footnote{earlier,
``IMSIL'' was considered an abbreviation of IMplant simulator for SILicon
technology. The new interpretation reflects the evolution of IMSIL, covering
sputtering as well as implantation.} may be the tool of choice.


This manual describes IMSIL as of \today. 
\ifprivate
This manual describes IMSIL as of December 2003.  IMSIL was converted from
FORTRAN77 to FORTRAN90 over the years, and this is the first full FORTRAN90
version.  In this transition, almost all of the code was changed.  Most
features have thoroughly been tested, but some still await testing, in
particular the molecular dynamics mode is temporarily disabled.  If this
feature is needed, use of IMSIL 2002 is recommended.
\fi 

IMSIL allows targets composed of a crystalline region with a lattice structure
that can be described by a cubic unit cell, and an arbitrary number regions of
arbitrary amorphous material \cite{hobler_monte_1995}. The target may either be
a 1-D layered structure defined by the interface positions of the materials
\cite{hobler_effect_1992,hobler_channeling_1999}, or a 2-D structure composed of
regions defined by arbitrary polygons \cite{hobler_monte_1989,
hobler_simulation_2003}. For 1-D targets, geometry and material changes may be
simulated in a dynamic manner. For all target geometries, damage produced by
the ions in the crystalline layer may dynamically be taken into account for the
simulation of subsequent ion trajectories
\cite{simionescu_two-dimensional_1993}, which can have a profound effect on the
distribution of the implanted ions due to the suppression of the channeling
effect. The motion of recoils may be followed in order to obtain a more accurate
distribution of the damage \cite{hobler_two-dimensional_1988}, to obtain the
distribution of atoms recoiled from one target region into another
\cite{Moe85}, and must be followed to simulate sputtering. The ions may either
be atoms (B, P, As, ...) or molecules composed of an arbitrary number of atoms
(BF$_2$, ...).

The models implemented in IMSIL have been carefully calibrated against
experimental data for conditions commonly found in semiconductor industry.  In
particular, electronic stopping is well described in silicon both under random
and channeling conditions for all implant species.\footnote{Nuclear stopping is
generally considered accurate under most conditions anyway. This should be taken
for granted for any binary collision simulator.}  Damage generation in silicon
has been calibrated for B, BF$_2$, Si, P, and As, and can be interpolated for
any ion species.  In order to deal with special conditions such as different
materials, target temperatures different from room temperature, or high dose
rates, the user is given access to all model parameters via the input file.

IMSIL was first released in 1993. Until 2003, it was mainly developed as an
implant simulator. Apart from the carefully calibrated models for silicon
technology \cite{hobler_monte_1995, simionescu_model_1995, hobler_towards_2000,
hobler_status_2003}, another unique feature of IMSIL is its physics-based model
of extended free flight paths in amorphous targets
\cite{hobler_acceleration_1995}, which promote computational efficiency at high
ion energies.

Two major extensions of IMSIL 2005 were the possibility to calculate RBS/C
energy and depth spectra \cite{kovac_investigation_2006}, and to use a
deterministic model of damage as an alternative to the standard statistic model
\cite{hobler_multiscale_2005,otto_coupled_2005}. The deterministic defect model
was used for coupling IMSIL with kinetic Monte Carlo simulations. It has
been disabled, because it has not been used since then. 

The major extension of IMSIL 2007 was the capability to use crystal structures 
different from diamond, namely wurtzite, zincblende, and the 4H and 6H
polytypes of SiC \cite{g._hobler_simulation_2018}. More recently the
possibility to read in arbitrary orthogonal unit cells and the fcc, bcc, and sc lattice
were implemented \cite{schlueter_absence_2020}. Also, a polycrystalline mode has 
been implemented \cite{schlueter_absence_2020} where the crystal lattice is
randomly rotated before each ion impact.

Starting 2008, various extensions for more accurate sputtering simulations have 
been implemented. These include the implementation of a surface binding energy
(planar, isotropic and intermediate models) \cite{ebm_assessment_2009,
hobler_assessment_2013} and of a surface roughness model that provides more
realistic sputter yields for glancing incidence of the ions
\cite{lindsey_simple_2017}. The introduction of ``virtual recoils''
\cite{hobler_dynamic_2011} promotes computational efficiency by following
recoils down to the surface binding energy only near the surface, so that both
sputtering and displacement statistics are correctly obtained, while unnecessary
trajectory simulations down to the surface binding energy in the bulk are
avoided. Targets with rotational and spherical symmetry have been implemented in
2016 \cite{urbassek_sputtering_2018,bradley_second_2021}.
Arbitrary 3-D geometries still wait for implementation, although the 
geometry-related code has been written with the extension to 3-D in mind.
Some activities on 2-D dynamic target relaxation have been started
\cite{hobler_dynamic_2011,hobler_combined_2015}, and some traces can be found in
the code. It is planned to separate this feature from IMSIL into a separate code
with some coupling to IMSIL. 

As a beam model, traditionally a punctiform beam and a homogeneous windowed beam
had been available with an optional simple beam divergence model (isotropic up
to a maximum angle). The beam divergence model has recently (2021) been
extended to include superpositions of Gaussian distributions with different
standard deviation in $x$ and $y$ direction for both the position and the
direction of the incident ions. The former allows to describe quite general
focused ion beams, while the latter is relevant to channeling if the beam is
close to a low-index crystallographic direction. 

IMSIL provides a multitude of options for writing spatial, angular, and energy 
distributions to output files (``histogram files''). These include the 1-D, 2-D,
and 3-D spatial distributions of stopped ions, vacancies, and recoils as well as
nuclear and electronic energy deposition. In addition, 1-D and 2-D angular and
1-D energy distributions of backscattered or transmitted ions and recoils can be
written to files. Also, ion and recoil trajectories or just their initial and/or
final states can be saved to a file.

The remainder of this manual is organized as follows: After an introduction to
the options for the specification of the ions and the target in Chapter~2, the
physical models implemented in IMSIL are described in Chapter~3. The formats of
the input and output files are given in Chapter~4. Chapter~5 describes all the
parameters that may be specified in the input file.  After a description of
Python scripts that facilitate the running of IMSIL and the plotting of results
in Chapter~6, typical examples of input files are presented in Chapter~7. 

