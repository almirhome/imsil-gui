IMSIL currently supports 1-D, 2-D, and rotational symmetric 3-D geometries for
static simulations, and 1-D geometries for dynamic simulations.

\textbf{1-D geometries} describe layered targets with surfaces and interfaces
parallel to the $x$-$y$ plane. They are therfore defined by an array of $z$
coordinates which correspond to the position of the top surface, any
interfaces between layers (regions), and the bottom surface. Semi-infinite and
infinite targets may be approximated by large enough values of the surface
coordinates. 

1-D geometries combined with \textbf{cylindrical or spherical symmetry} describe
targets composed of cylindrical or spherical shells. The surface and interface
positions then refer to the distance from the symmetry line or center. The
``top'' surface is the inner surface, and the ``bottom'' surface the outer
surface. If a hollow body is not desired, the position of the inner surface has
to be assigned a negative value.

During a \textbf{dynamic simulation}, the cell densities are allowed to change
until some threshold is reached for the deviation from equilibrium. Then the
cell geometries are adapted so that the densities relax towards equilibrium.

\textbf{2-D geometries} are defined by polygons in the $x$-$z$ plane. The
polygons are defined by vertices and their connectivity (for details see
Section~\ref{s:geom}). Since these specifications can be lengthy, they can be
put into a separate geometry file. 2-D geometries also may be specified to be
periodic in $x$ direction. Internally, the geometry information is converted
into a distance array for each region. The distance values are defined on a mesh
covering the respective region.

\textbf{3-D rotational symmetric geometries} may be realized by combining 2-D
geometries with cylindrical symmetry. For a symmetry axis parallel to the $x$
axis, the 3-D geometry is defined by rotating the polygons about this axis.
Other orientations of the symmetry axis are realized by permutations of $x$,
$y$, and $z$ \cite{bradley_second_2021}.

Finally, a simple \textbf{surface roughness} model is provided for both 1-D and
2-D simulations. This so-called density gradient model
\cite{lindsey_simple_2017} reduces the target density linearly towards the
surface in a sub-surface layer. The model has been introduced to provide more
realistic sputter yields at glancing incidence. 


\begin{center}
\begin{tabular}{lll}
parameter \quad                   & IMSIL name    & to be specified in record \\
\hline
number of regions                 & \texttt{NR}      & \texttt{\&SETUP} \\
dimensionality                    & \texttt{NDIM}    & \texttt{\&SETUP} \\
1-D surface/interface coordinates & \texttt{POSIF}   & \texttt{\&GEOM} \\
cylindrical symmetry flag         & \texttt{LCYLX}   & \texttt{\&GEOM} \\
                                  & \texttt{LCYLY}   & \texttt{\&GEOM} \\
                                  & \texttt{LCYLZ}   & \texttt{\&GEOM} \\
spherical symmetry flag           & \texttt{LSPHER}  & \texttt{\&GEOM} \\
position of symmetry line/center  & \texttt{CENTER}  & \texttt{\&GEOM} \\
number of ions for cell update    & \texttt{NIONUPD} & \texttt{\&GEOM} \\
threshold change for cell update  & \texttt{FRACUPD} & \texttt{\&GEOM} \\
fixed coordinate in cell update   & \texttt{POSFIX}  & \texttt{\&GEOM} \\
2-D geometry file                 & \texttt{GEOMFILE} & \texttt{\&GEOM} \\
polygon vertices                  & \texttt{POS}     & \texttt{\&GEOM} \\
polygon connectivity              & \texttt{POINTS}  & \texttt{\&GEOM} \\
periodicity flag                  & \texttt{LPERX}   & \texttt{\&GEOM} \\
periodic boundary positions       & \texttt{XPER}    & \texttt{\&GEOM} \\
box width of distance grid        & \texttt{WBOX2}   & \texttt{\&GEOM} \\
maximum size of distance grid     & \texttt{NBOX2}   & \texttt{\&GEOM} \\
thickness of roughness layer      & \texttt{WROUGH}  & \texttt{\&GEOM} \\

\end{tabular}
\end{center}

