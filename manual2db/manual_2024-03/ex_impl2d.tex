This example shows the template and parameters file for the simulation of 
boron implantations into trenches with uneven surfaces. It shows two new
features: The definition of a 2-D geometry, and the use of a template and
parameters file to automatically generate two input files.

The template file reads:

\begin{verbatim}
Boron implantation into a bumpy 2D silicon trench
B in #text#-Si, 25 keV, 7 tilt
 &setup    ndim=2 natom=2 nr=1 nion=100000 /
 &ions     name='B' energy=25 dose=1e15 tilt=7 diverg=0
           xinit=-10000,10000 /
 &material region=1 name='Si' xtal=#xtal# /
 &geometry point=1 pos=-15000,0 /
 &geometry point=2 pos=-15000,30000 /
 &geometry point=3 pos=15000,30000 /
 &geometry point=4 pos=15000,670 /
 &geometry point=5 pos=5830,670 /
 &geometry point=6 pos=5000,3500 /
 &geometry point=7 pos=4830,5000 /
 &geometry point=8 pos=5000,8330 /
 &geometry point=9 pos=4830,10580 /
 &geometry point=10 pos=4420,14000 /
 &geometry point=11 pos=4250,17170 /
 &geometry point=12 pos=4170,21000 /
 &geometry point=13 pos=4000,22830 /
 &geometry point=14 pos=2670,23500 /
 &geometry point=15 pos= 830,23580 /
 &geometry point=16 pos=-1830,23500 /
 &geometry point=17 pos=-3500,23170 /
 &geometry point=18 pos=-4330,22750 /
 &geometry point=19 pos=-4500,20830 /
 &geometry point=20 pos=-4580,17580 /
 &geometry point=21 pos=-4500,13830 /
 &geometry point=22 pos=-4580,9170 /
 &geometry point=23 pos=-4170,3750 /
 &geometry point=24 pos=-4170,2420 /
 &geometry point=25 pos=-4580,0 /
 &geometry region=1 points=1,2,3,4,5,6,7,8,9,10,11,12,13,14,15,16,17,18,19,20 /
 &geometry region=1 points=21,22,23,24,25 /
 &output   lmom=t lhis=t nbox2=0,0 wbox2=500,500
           lhis2=t nionbk=10000 / nbox2=100,100
\end{verbatim}

The parameter \texttt{ndim=2} on the \texttt{\&SETUP} record specifies that a
2-D geometry will be defined. On the \texttt{\&IONS} record,
\texttt{xinit=-10000,10000} specifies that the reference points for the initial
ion position are uniformly distributed between $x=-10000$~\AA\ and
$x=10000$~\AA. Since \texttt{yinit} and \texttt{zinit} are not specified, their
default values \texttt{0} are assumed.

The \texttt{\&GEOMETRY} records with \texttt{point} index variables define the
vertices of the polygons describing the region boundaries. 
E.g., \texttt{pos=-15000,0} specifies the vertex $(x=-15000$\AA$, z=0)$. 
The \texttt{\&GEOMETRY} records with \texttt{region} index variables define the
region boundaries of the target by connecting the points with the given indices
in the given order.

Note that we have two variables in this example: \texttt{text} and
\texttt{xtal} indicated by the \texttt{\#} signs. Their values are specified in
the parameters file:

\begin{verbatim}
FILE: btrench1
text=a
xtal='no'

FILE: btrench2
text=<100>
xtal='yes'
\end{verbatim}

The input files that will be generated by \texttt{mimsil.py}, will be
\texttt{btrench1.inp} and \texttt{btrench2.inp}.

As in the previous example (Section~\ref{s:ex_impl1d}), the damage
histograms are updated every \texttt{nthread} ions if the program is run with
$\texttt{nthread} > 1$ threads. Since now the computational cost for merging the
2-D histograms (\texttt{ndamdim} of the \texttt{\&DAMAGE} record is set to
\texttt{ndim} by default) is significant, the efficiency of the simulation can
be improved by specifying, e.g., ~\texttt{nionhis=80} on the \texttt{setup}
record. Note that increasing \texttt{nionhis} too much affects the accuracy of
the simulation.
