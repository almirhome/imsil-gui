This record is used to specify some basic parameters of the
simulation.  It has to appear in each input file.  It has no index
variables.

\begin{keydescription}{\texttt{ATOM1} --- Atom index of histogram file}
%
  This variable specifies the index of the atom in the histogram file 
  \texttt{HISFILE} whose concentration will be used to initialize the
  concentration of atoms with index \texttt{ATOM2}. Relevant only if 
  \texttt{USEHIS=T}.
%
  \begin{keytab}
    Type:    \> integer \\
    Default: \> ~ -- ~~~ (obligatory if \texttt{USEHIS=T} ) \\
    Range:   \>  $\texttt{ATOM1} > 0$
  \end{keytab}
\end{keydescription}

\begin{keydescription}{\texttt{ATOM2} --- Atom index of atom species}
%
  This variable specifies atom index of atom species which concentration will
  initilized with \texttt{ATOM1} concentration from histogram file
  \texttt{HISFILE}. 
  \ifprivate
  If \texttt{USEBK=T} and \texttt{LDET=T} distribution of
  \texttt{ATOM2} from from backup histogram file file \texttt{BKFILE} will be
  used to create simulation cell. 
  \fi
  Relevant only if \texttt{USEHIS=T} 
  \ifprivate
  or \texttt{USEBK=T}
  \fi
  .
%
  \begin{keytab}
    Type:    \> integer \\
    Default: \> ~ -- ~~~ (obligatory if \texttt{USEHIS=T} ) \\
    Range:   \>  $1 <= \texttt{ATOM2} < \texttt{NATOM}$
  \end{keytab}
\end{keydescription}

\begin{keydescription}{\texttt{BKFILE} --- Backup file name}
%
  This variable specifies the name of the backup file to be used to continue 
  a previous simulation. Relevant only if \texttt{USEBK=T}.
%
  \begin{keytab}
    Type:    \> character string \\
    Default: \> \texttt{BCK} \\
    Range:   \> Any name ($\le 80$ characters) of an IMSIL backup file 
  \end{keytab}
\end{keydescription}

\begin{keydescription}{\texttt{CELLFILE} --- Cell file name}
%
  This parameter specifies the name of the file containing the atom
  densities in the 1-D target cells for dynamic simulation
  (\texttt{LDYN=T}). The format of the file is described in
  Chapter~\ref{k:files}). Relevant only if \texttt{USECELL=T}.
  
  \begin{keytab}
    Type:    \> character string \\
    Default: \> ~ -- ~~~ (obligatory if \texttt{USECELL=T}) \\
    Range:   \> any name ($\le 80$ characters) of an IMSIL cell file
  \end{keytab}
\end{keydescription}

\begin{keydescription}{\texttt{CELLREC} --- Record to be read from
cell file}
%
  This variable specifies the index of the record to be read from the
  \texttt{CELL} or \texttt{CELL2} file, where record means the
  data describing one snapshot of the cells. \texttt{CELLREC=0} means the last
  record will be read.

  \begin{keytab}
    Type:    \> integer \\
    Default: \> 0 \\
    Range:   \> $\texttt{CELLREC} \ge 0$
  \end{keytab}
\end{keydescription}

\ifprivate
\begin{keydescription}{\texttt{DEFFILE} --- Defects file name}
%
  This parameter specifies the name of the file containing the defects to be
  used. The format of the file is described in Chapter~\ref{k:files}). Relevant
  only if \texttt{USEDEF=T}.
  
  \begin{keytab}
    Type:    \> character string \\
    Default: \> ~ -- ~~~ (obligatory if \texttt{USEDEF=T}) \\
    Range:   \> any name ($\le 80$ characters) of an IMSIL defects file
  \end{keytab}
\end{keydescription}
\fi

\begin{keydescription}{\texttt{EQUICELL} --- Equidistant cell grid flag}
%
  This parameter specifies whether in a 1-D dynamic simulation (\texttt{LDYN=T}) 
  the cell contents are interpolated back to an equidistant grid.  Note that
  this will introduce artificial diffusion.  \texttt{EQUICELL} is ignored
  for a 2-D simulation, since 2-D simulations are always done on equidistant
  grid. \texttt{EQUICELL=T} can be set to make a 1-D simulation
  consistent with a 2-D simulation. Otherwise, \texttt{EQUICELL=F} is 
  recommended.
  \begin{keytab}
    Type:    \> logical \\
    Default: \> \texttt{F} \\
    Range:   \> \texttt{T}, \texttt{F} 
  \end{keytab}
\end{keydescription}

\begin{keydescription}{\texttt{HISFILE} --- Histogram file name}
%
  This variable specifies the name of the histogram file to be used to create
  a damage profile for subsequent simulation. The format of the histogram file
  (\texttt{HIS}) is described in file in Chapter~\ref{k:files}). Relevant only
  if \texttt{USEHIS=T}.
%
  \begin{keytab}
    Type:    \> character string \\
    Default: \> ~ -- ~~~ (obligatory if \texttt{USEHIS=T}) \\
    Range:   \> Any name ($\le 80$ characters) of an IMSIL histogram file 
  \end{keytab}
\end{keydescription}

\begin{keydescription}{\texttt{IARAND} --- Seed of random number generator}
%
  This variable specifies one of the integer numbers defining the seed of
  the random number generator.
%
  \begin{keytab}
    Type:    \> integer \\
    Default: \> 15678 \\
    Range:   \> $1 \le \texttt{IARAND} \le 70000$ for \texttt{RNG='Haas'} \\
             \> $1 \le \texttt{IARAND} \le 131071$ otherwise
  \end{keytab}
\end{keydescription}

\begin{keydescription}{\texttt{IBRAND} --- Seed of random number generator}
%
  This variable specifies one of the integer numbers defining the seed of
  the random number generator.
%
  \begin{keytab}
    Type:    \> integer \\
    Default: \> 67638 \\
    Range:   \> $1 \le \texttt{IBRAND} \le 90000$ for \texttt{RNG='Haas'} \\
             \> $1 \le \texttt{IBRAND} \le 262143$ otherwise
  \end{keytab}
\end{keydescription}

\begin{keydescription}{\texttt{IRAND} --- Seed of random number generator}
%
  This variable specifies one of the integer numbers defining the seed of
  the random number generator. Note that changing only \texttt{IRAND}
  may result in an identical series of random numbers after a
  relatively small number of numbers. 
%
  \begin{keytab}
    Type:    \> integer \\
    Default: \> 5421 \\
    Range:   \> $1 \le \texttt{IRAND} \le 10000$ for \texttt{RNG='Haas'} \\
             \> $1 \le \texttt{IRAND} \le 16383$ otherwise
  \end{keytab}
\end{keydescription}

\begin{keydescription}{\texttt{LDAMDYN} --- Dynamic damage consideration flag}
%
  This variable specifies whether damage accumulated during the simulation is
  considered for recombination with new damage and/or for selecting the
  collision partners in a crystalline region.
%
  \begin{keytab}
    Type:    \> logical \\
    Default: \> \texttt{LDAM} of the \texttt{\&DAMAGE} record \\
    Range:   \> \texttt{T}, \texttt{F}
  \end{keytab}
\end{keydescription}

\begin{keydescription}{\texttt{LDYN} --- Dynamic simulation flag}
%
  This variable specifies whether the target densities shall be
  updated in a spatially resolved, cell-based target description.  In
  the dynamic mode the cell densities are updated during the simulation
  and subsequently used in the selection of collision partners.  The cell 
  densities can be output to the \texttt{CELL*} files (see 
  Section~\ref{k:files}) using \texttt{LCELL=T} on the \texttt{\&OUTPUT}
  record.
%
  \begin{keytab}
    Type:    \> logical \\
    Default: \> \texttt{F} \\
    Range:   \> \texttt{T}, \texttt{F}
  \end{keytab}
\end{keydescription}

\begin{keydescription}{\texttt{LSPUT} --- Sputtering simulation flag}
%
  This variable specifies whether accurate sputtering simulation is to be
  performed. If \texttt{LSPUT=T} the surface binding energy is
  considered. In addition, ``virtual'' recoils are generated (see 
  Section~\ref{s:sput}). Also the free flight paths in amorphous target are 
  adjusted such that no collision with energy transfer larger than the surface 
  binding energy is missed.
%
  \begin{keytab}
    Type:    \> logical \\
    Default: \> \texttt{F} \\
    Range:   \> \texttt{T}, \texttt{F}
  \end{keytab}
\end{keydescription}

\begin{keydescription}{\texttt{MRAND} --- Seed of random number generator}
%
  This variable specifies one of the integer numbers defining the seed of
  the random number generator.
%
  \begin{keytab}
    Type:    \> integer \\
    Default: \> 6103 \\
    Range:   \> $1 \le \texttt{MRAND} \le 8000$ for \texttt{RNG='Haas'} \\
             \> $1 \le \texttt{MRAND} \le 16383$ otherwise
  \end{keytab}
\end{keydescription}

\begin{keydescription}{\texttt{NATOM} --- Number of atom species}
%
  This variable specifies the number of atom species contained in the
  ion and the target regions. E.g. for BF$_2$ implantation in a
  two-layer structure of SiO$_2$ and Si, \texttt{NATOM=4} corresponding
  to the atom species B, F, Si, and O.  A target atom is always
  considered different from an ion atom, even if the same chemical
  element is specified.  E.g., for Si implantation into bare Si
  \texttt{NATOM=2}.  This is because by default the mass of an ion
  atom is set to the mass of the most abundant isotope while the
  default mass of the target atom is set to the atom mass averaged
  over all isotopes weighted with their natural abundance.  Also, it
  allows to ``label'' the ion atoms so they can be distinguished from
  the target atoms.
  
  If the target information is read from a \texttt{CELL} or \texttt{CELL2}
  file (\texttt{USECELL=T}), \texttt{NATOM} depends on whether a dynamic
  simulation is performed: If \texttt{LDYN=T}, \texttt{NATOM} must equal
  the number of atom species stored in the \texttt{CELL*} file.  If 
  \texttt{LDYN=F}, \texttt{NATOM} must equal the number of ion atoms (e.g., 2
  for BF$_2$) plus the number of atom species  stored in the \texttt{CELL*} 
  file.
%
  \begin{keytab}
    Type:    \> integer \\
    Default: \> ~ -- ~~~ (obligatory) \\
    Range:   \> must equal the number of atoms in ion and target
  \end{keytab}
\end{keydescription}

\begin{keydescription}{\texttt{NDIM} --- Number of dimensions}
%
  This variable specifies the number of dimension of the geometry.
  Depending on \texttt{NDIM}, the geometry is defined by the position of
  the interface (for \texttt{NDIM=1}), or by polygons (for
  \texttt{NDIM=2}) defined by corner points (interface points). 
  Alternatively, for 2-D and 3-D targets (\texttt{NDIM=2} or 
  \texttt{NDIM=3}) the geometry may be specified by signed distance
  functions. See the \texttt{\&GEOMETRY} record
%
  \begin{keytab}
    Type:    \> integer \\
    Default: \> 1 \\
    Range:   \> 1, 2, 3
  \end{keytab}
\end{keydescription}

\begin{keydescription}{\texttt{NION} --- Number of simulated ions}
%
  This parameter specifies the number of ions to be simulated. The
  simulation time will be roughly proportional to \texttt{NION} while
  statistical errors decrease with the square root of \texttt{NION}.
  In case of the continuation of a previous run (\texttt{USEBK=T}),
  \texttt{NION} specifies the total number of ions including those
  read from the backup file. If \texttt{NION} is not a multiple of
  \texttt{NTHREAD}, it is adjusted to the next higher multiple 
  of \texttt{NTHREAD}.
%
  \begin{keytab}
    Type:    \> integer \\
    Default: \> ~ -- ~~~(obligatory) \\
    Range:   \> $> 0$
  \end{keytab}
\end{keydescription}

\begin{keydescription}{\texttt{NIONBK} --- Number of ions for backup}
	%
	This parameter specifies that a backup file shall be written/updated 
	after every	\texttt{NIONBK} ions. \texttt{NIONBK=0} means that no backup 
	file shall be written. In case \texttt{NIONBK>NION}, a backup file is 
	written	after the last ion. \texttt{NIONBK} cannot be specified in case 
	of a dynamic simulation (\texttt{LDYN=T}). If \texttt{NIONBK} is not a
	multiple of \texttt{NIONCHUNK}, it is adjusted to the next higher multiple of
	\texttt{NIONCHUNK}. Note that only updated histograms are stored in the
	simulation. Therefore, temporary ion and damage distributions are lost if
	\texttt{NIONHIS>0} and \texttt{NIONBK} is not a multiple of \texttt{NIONHIS}.
	\begin{keytab}
		Type:    \> integer \\
		Default: \> 0  \\
		Range:   \> $\ge 0$
	\end{keytab}
\end{keydescription}

\begin{keydescription}{\texttt{NIONCHUNK} --- Number of simulated ions per
chunk}
%
  This parameter specifies the number of ions to be simulated in one ``chunk''.
  Larger chunks slightly reduce the simulation time, while smaller chunks
  allow more refined damage and/or geometry updates and/or output on the backup
  file (see \texttt{NIONHIS}, \texttt{NIONGEOM}, and \texttt{NIONBK}), since
  these actions can only be done after integer multiples of \texttt{NIONCHUNK}
  simulated ions, and at the end of the simulation.
  If \texttt{NIONCHUNK} is not a multiple of \texttt{NTHREAD}, it is 
  adjusted to the next higher multiple of \texttt{NTHREAD}.
  Usually, the default value of \texttt{NIONCHUNK} can be used, therefore it 
  often is not necessary to specify this parameter in the input file.
%
  \begin{keytab}
    Type:    \> integer \\
    Default: \> \texttt{NIONHIS}, if \texttt{LDAMDYN=T} or \texttt{LDYN=T},
                and \texttt{NIONHIS>0} \\ 
    		 \> \texttt{NIONGEOM} otherwise, if \texttt{LDYN=T} \\
    		 \> \texttt{NION} otherwise \\
    		 \> greatest common factor of value above and \texttt{NIONTRA}, 
    		    if \texttt{ITRA} or \\
    		 \>  \texttt{LTRA} specified on \texttt{\&OUTPUT} record \\
    Range:   \> $> 0$
  \end{keytab}
\end{keydescription}

\begin{keydescription}{\texttt{NIONGEOM} --- Number of ions for geometry updates}
%
  This parameter specifies the number of implanted ions after which the
  cells are checked for necessary geometry updates in a dynamic simulation
  (\texttt{LDYN=T}). See also \texttt{FRACUPD} of the \texttt{\&GEOMETRY}
  record. Note that updating the geometry requires some computational effort in
  1D, and considerable computational effort in 2D. This effort decreases with 
  larger \texttt{NIONGEOM} at the expense of accuracy. On the other hand, too
  infrequent geometry updates affect the simulation results if \texttt{LDYN=T}.
  If \texttt{NIONGEOM} is not a multiple of \texttt{NIONHIS}, it is adjusted 
  to the next higher multiple of \texttt{NIONHIS}. \texttt{NIONGEOM} is ignored 
  if \texttt{LDYN=F}.
  \begin{keytab}
    Type:    \> integer \\
    Default: \> \texttt{NIONHIS}, if \texttt{NIONHIS} is defined and $> 0$ \\
    		 \> \texttt{NIONCHUNK} otherwise, if \texttt{NIONCHUNK} is defined \\
    		 \> \texttt{NTHREAD} otherwise \\
    Range:   \> $> 0$
  \end{keytab}
\end{keydescription}

\begin{keydescription}{\texttt{NIONHIS} --- Number of ions for histogram updates}
%
  If greater than zero, this parameter specifies the number of implanted ions
  after which the ion and damage histograms (\texttt{LDAMDYN=T}) or the cell
  densities (\texttt{LDYN=F}) are updated. All other histograms (e.g., the 
  histograms of backscattered and transmitted atoms) are not affected
  by \texttt{NIONHIS}; they are only updated at the end of the simulation. 
  \texttt{NIONHIS=0} means that the histograms or cell densities are 
  updated whenever an event leads to a contribution to a histogram or cell
  density during the development of a collision cascade. This is only
  allowed when \texttt{NTHREAD=1}. Note that updating histograms requires some
  additional computational effort, in particular in parallel processing. This 
  effort decreases with larger \texttt{NIONHIS} at the expense of accuracy. On
  the other hand, too infrequent damage updates affect the simulation results
  if \texttt{LDAMDYN=T}. If \texttt{NIONHIS} is not a multiple of
  \texttt{NIONCHUNK}, it is adjusted to the next higher multiple of
  \texttt{NIONCHUNK}. \texttt{NIONHIS} is ignored if \texttt{LDAMDYN=F} and
  \texttt{LDYN=F}.
  \begin{keytab}
    Type:    \> integer \\
    Default: \> 0, if $\texttt{NTHREAD} = 1$ \\
             \> \texttt{NIONCHUNK}, if $\texttt{NTHREAD} > 1$ and
                \texttt{NIONCHUNK} is defined \\
             \> \texttt{NTHREAD}, if $\texttt{NTHREAD} > 1$ and
                \texttt{NIONCHUNK} is not defined \\
    Range:   \> $0 \le \texttt{NIONHIS} \le \texttt{NION}$, if
                $\texttt{NTHREAD} = 1$ \\
             \> $1 \le \texttt{NIONHIS} \le \texttt{NION}$, if 
                $\texttt{NTHREAD} > 1$
  \end{keytab}
\end{keydescription}

\begin{keydescription}{\texttt{NIONOUT} --- Number of ions for output}
%
  This parameter specifies the number of ions after which output is 
  written to the \texttt{CELL}, \texttt{CELL2}, and \texttt{SURF} file
  in case of a dynamic simulation (\texttt{LDYN=T}). \texttt{NIONOUT} is
  rounded to the next lowert integer multiple of \texttt{NIONGEOM} (after
  any adjustments to \texttt{NIONGEOM}). In addition, $\texttt{NIONOUT}
  \ge \texttt{NIONGEOM}$ is enforced. If \texttt{NION} is not an 
  integer multiple of \texttt{NIONOUT}, an additional data record is 
  written at the end of the simulation.
  \begin{keytab}
    Type:    \> integer \\
    Default: \> \texttt{NIONGEOM} \\
    Range:   \> $> 0$
  \end{keytab}
\end{keydescription}

\begin{keydescription}{\texttt{NIONTRA} --- Number of ions for trajectory output}
	%
	This parameter specifies that trajectories shall be written to the 
	\texttt{TRA} file for every \texttt{NIONTRA}th ion. 
	\texttt{NIONTRA=1} means that trajectory output shall be written for every 
	ion.  
	If \texttt{NIONTRA} is not a multiple of \texttt{NIONCHUNK}, it is 
	adjusted to the next higher multiple of \texttt{NIONCHUNK}. 
    This parameter can be used for reducing the amount of output to the 
	\texttt{TRA} file. 
    \texttt{NIONTRA} is ignored unless \texttt{ITRA}
	or \texttt{LTRA} of the \texttt{\&OUTPUT} record is set.
	\begin{keytab}
		Type:    \> integer \\
		Default: \> 1  \\
		Range:   \> $> 0$
	\end{keytab}
\end{keydescription}

\begin{keydescription}{\texttt{NR} --- Number of regions}
%
  This variable specifies the number of target regions. It is ignored when
  the target geometry is defined by cells (\texttt{USECELL=T}).
%
  \begin{keytab}
    Type:    \> integer \\
    Default: \> ~ -- ~~~ (obligatory if \texttt{USECELL=F}) \\
    Range:   \> $\texttt{NR} > 0$
  \end{keytab}
\end{keydescription}

\begin{keydescription}{\texttt{NTHREAD} --- Number of threads}
%
  This variable specifies the number of threads. It overwrites the number of
  threads specified as an environment variable or during compilation. It
  is ignored when the code is compiled without multiprocessing or in the
  coarray version of the code. 
%
  \begin{keytab}
    Type:    \> integer \\
    Default: \> as specified by an environment variable or during compilation \\
    Range:   \> $\texttt{NTHREAD} > 0$
  \end{keytab}
\end{keydescription}

\begin{keydescription}{\texttt{RANDSKIP} --- Number of random numbers to be skipped}
%
  This variable specifies the number of random number evaluations before the start 
  of the simulation. Thus, the first \texttt{RANDSKIP} generated numbers of the 
  random number generator are ignored. This is a simpler alternative to specifying
  \texttt{IRAND}, \texttt{IARAND}, \texttt{IBRAND}, and \texttt{MRAND}. The use of 
  \texttt{RANDSKIP} is problematic for amorphous targets.
%
  \begin{keytab}
    Type:    \> integer \\
    Default: \> 0 \\
    Range:   \> $\texttt{RANDSKIP} \ge 0$
  \end{keytab}
\end{keydescription}

\begin{keydescription}{\texttt{RNG} --- Random number generator}
%
  This variable specifies the random number generator used to generate uniform
  random numbers in the range [0,1]. \texttt{'Haas'} uses the ``The multiple 
  prime random number generator'' (ACM Trans.\ Math.\ Software 13 (4) 368-381, 
  1987). \texttt{'Fortran'} uses the default Fortran random number generator 
  \texttt{RANDOM\_NUMBER()}. \texttt{'Knuth'} uses ``Knuth's subtractive random 
  number generator'' as described in W.H. Press et al., Numerical Recipes, 3rd 
  edition, Cambridge University Press, 2007; p.354. Note that the seed of the
  random number generator may be set by \texttt{MRAND}, \texttt{IARAND}, 
  \texttt{IBRAND}, and \texttt{IRAND}. The seed does not completely specify 
  the internal state of Knuth's random number generator. It is therefore not
  possible to report \texttt{MRAND}, \texttt{IARAND}, \texttt{IBRAND}, and
  \texttt{IRAND} at the end of the simulation. However, different seeds lead
  to different random number sequences of Knuth's random number generator.
%
  \begin{keytab}
    Type:    \> character string \\
    Default: \> \texttt{'Knuth'} \\
    Range:   \> \texttt{'Haas'}, \texttt{'Fortran'}, \texttt{'Knuth'} 
  \end{keytab}
\end{keydescription}

\begin{keydescription}{\texttt{REFLECT} --- Initial condition reflection flag}
%
  This variable specifies whether distributions read from \texttt{HIS} or
  \texttt{CELL} files are reflected at the origin. If \texttt{SHIFTTO} is also
  specified, reflection is performed after shifting.
%
  \begin{keytab}
    Type:    \> logical \\
    Default: \> \texttt{F} \\
    Range:   \> \texttt{T}, \texttt{F}
  \end{keytab}
\end{keydescription}

\begin{keydescription}{\texttt{SHIFTTO} --- Shifted surface position for intial condition}
%
  This variable specifies the position the surface of distributions read from
  \texttt{HIS} or \texttt{CELL} files should be shifted to. If \texttt{REFLECT} 
  is also specified, reflection is performed after shifting.

  \begin{keytab}
    Type:    \> real \\
    Default: \> -- (corresponding to no shift)
  \end{keytab}
\end{keydescription}

\begin{keydescription}{\texttt{TEMP} --- Temperature of the target}
%
  This variable specifies the temperature of the target in Kelvin. It is used 
  for the calculation of the vibrational amplitude using the Debye
  function. See also \texttt{TDEBYE}

  \begin{keytab}
    Type:    \> real \\
    Default: \> 300. \\
    Range:   \> $\texttt{TEMP} > 0$
  \end{keytab}
\end{keydescription}

\begin{keydescription}{\texttt{USEBK} --- Use-backup-file flag}
%
  This variable specifies whether a backup file shall be used to
  continue a previous simulation. If \texttt{USEBK=T}, the backup file
  specified by \texttt{BKFILE} must exist. If \texttt{USEION=T}, the
  input file must specify a larger dose \texttt{DOSE} and a larger
  number of ions \texttt{NION} on the \texttt{\&IONS} record than
  those stored in the backup file. The continued simulation starts
  with the ion and damage histograms stored in the backup file and
  performs a simulation with a number of ions corresponding to the
  difference of the numbers specified in the input and backup files
  and a dose corresponding to the difference of the doses specified in
  the input and backup files.  It is not recommended that other
  parameters than \texttt{NION} and \texttt{DOSE} be changed in the
  input file between the previous and the continued run when
  \texttt{USEION=T}.  If \texttt{USEION=F}, a new simulation is
  started from scratch except that the damage histograms are read in
  from the backup file at the beginning of the simulation.
%
  \begin{keytab}
    Type:    \> logical \\
    Default: \> \texttt{F} \\
    Range:   \> \texttt{T}, \texttt{F}
  \end{keytab}
\end{keydescription}

\begin{keydescription}{\texttt{USECELL} --- Use-cell-file flag}
%
  This variable specifies whether the target information shall be read from
  a cell file as specified by the \texttt{CELLFILE} parameter. If
  \texttt{USECELL=T}, the cell file specified by \texttt{CELLFILE} must exist,
  and the material and geometry specifications in the \texttt{\&MATERIAL} and
  \texttt{\&GEOMETRY} records are ignored.  The cell file must be a
  \texttt{CELL} file (see Section~\ref{k:files}). If \texttt{USECELL=T} and
  \texttt{LDYN=T}, the atomic numbers of the ion atoms stored in the cell file
  must be consistent with the atom species given in the ion name (\texttt{NAME}
  parameter of the \texttt{\&IONS} record).

%
  \begin{keytab}
    Type:    \> logical \\
    Default: \> \texttt{F} \\
    Range:   \> \texttt{T}, \texttt{F}
  \end{keytab}
\end{keydescription}

\ifprivate
\begin{keydescription}{\texttt{USEDEF} --- Use-deterministic-defect-file flag}
%
  This variable specifies whether a deterministic defect file \texttt{DEFFILE}
  shall be used in the simulation. If \texttt{USEDEF=T}, the defect file 
  specified by \texttt{DEFFILE} must exist.
%
  \begin{keytab}
    Type:    \> logical \\
    Default: \> \texttt{F} \\
    Range:   \> \texttt{T}, \texttt{F}
  \end{keytab}
\end{keydescription}
\fi

\begin{keydescription}{\texttt{USEHIS} --- Use-histogram-file flag}
%
  This variable specifies whether a histogram shall be used to initialize the 
  damage distribution. If \texttt{USEHIS=T}, the histogram file specified by
  \texttt{HISFILE} must exist. 
%
  \begin{keytab}
    Type:    \> logical \\
    Default: \> \texttt{F} \\
    Range:   \> \texttt{T}, \texttt{F}
  \end{keytab}
\end{keydescription}

\begin{keydescription}{\texttt{USEION} --- Use-ion-distribution-from-backup-file
                       flag}
%
  This variable specifies whether the ion distribution stored on the
  backup file shall be used as initial values. If \texttt{USEION=F}, only
  the damage distribution is taken from the backup file and the ion
  distribution is set to zero. See also \texttt{USEBK}.
%
  \begin{keytab}
    Type:    \> logical \\
    Default: \> \texttt{T} \\
    Range:   \> \texttt{T}, \texttt{F}
  \end{keytab}
\end{keydescription}

