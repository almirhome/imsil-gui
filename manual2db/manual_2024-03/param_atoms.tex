This record is used to define the properties of the atoms. Normally, it is not
necessary to include this record, since the atom properties are defined
internally, and the atom species are defined on the \texttt{\&IONS} and
\texttt{\&MATERIAL} records or, if the target is initialized from a
\texttt{CELL} file, in the \texttt{CELL} file. However, this record can be used
to change the default values of the atom properties.

All parameters of this record are index variable arrays with the index variable
\texttt{ATOM}. \texttt{ATOM} is the atom index. The mapping between atom names
and atom indices is defined by the sequence of occurence of the atom names in
the ion and material names: First the ion and then the materials from region 1
to how many regions are defined. Within the ion and material names from left to
right. Using the same example as in Section~\ref{s:his1d},
%
\begin{verbatim}
 &IONS NAME='B', ... /
 &MATERIAL REGION=2 NAME='SI3N4' ... /
 &MATERIAL REGION=1 NAME='SIO2' ... /
 &MATERIAL REGION=3 NAME='SI' ... /
\end{verbatim}
%
atom indices and names relate as follows:

\begin{tabular}{cc}
  \texttt{ATOM} & \texttt{NAME} \\
  \hline
  1 & \texttt{B} \\
  2 & \texttt{SI} \\
  3 & \texttt{O} \\
  4 & \texttt{N}
\end{tabular}

Note that material names are ordered according to the \texttt{REGION} index
rather than their appearance in the input file.

Occasionally, the same atom species may occur in the ion and one or more
materials. As an example, take
%
\begin{verbatim}
 &IONS NAME='CN', ... /
 &MATERIAL REGION=1 NAME='C5H8O2' ... /
 &MATERIAL REGION=2 NAME='C' ... /
\end{verbatim}
%
For a static simulation (\texttt{LDYN=F}), the C atom in the ion is treated as
separate atom species from the C atoms in the materials:

\begin{tabular}{cc}
  \texttt{ATOM} & \texttt{NAME} \\
  \hline
  1 & \texttt{C} \\
  2 & \texttt{N} \\
  3 & \texttt{C} \\
  4 & \texttt{H} \\
  5 & \texttt{O}
\end{tabular}

The C atoms in the ion may be assigned different properties than the C atoms in
the two materials. On the other hand, in a dynamic simulation (\texttt{LDYN=T}),
the C atoms in the ion are not distinguished from the C atoms in the materials:

\begin{tabular}{cc}
  \texttt{ATOM} & \texttt{NAME} \\
  \hline
  1 & \texttt{C} \\
  2 & \texttt{N} \\
  3 & \texttt{H} \\
  4 & \texttt{O}
\end{tabular}

When a \texttt{CELL} file is read rather than regions being defined, the atom
incides are defined by the order as the atomic numbers appear in line 2 of the
\texttt{CELL} file (cf.\ Section~\ref{s:his1d}). The ion atoms as specified in
the \texttt{\&IONS} record must appear in the \texttt{CELL} file first and in
the same order as in the \texttt{\&NAME} parameter. That means, in the current
version of IMSIL, the ion atom species in a dynamic simulation must already be
in the target when the target is read from a \texttt{CELL} file. If necessary,
the \texttt{CELL} file must be prepared by an external tool (script) to include
the new ion species atom(s).

$1 \le \texttt{ATOM} \le \texttt{NATOM}$ is required, where \texttt{NATOM}
denotes the number of atom species in the simulation.

\begin{keydescription}{\texttt{DENSITY} --- Atom density}
%
  This parameter specifies the number of target atoms per cm$^3$ in a pure material.
  
  \begin{keytab}
    Type:    \> index variable array (\texttt{ATOM}) of real \\
    Default: \> depending on \texttt{NAME} \\
    Range:   \> $> 0$
  \end{keytab}
\end{keydescription}

\begin{keydescription}{\texttt{EFAKE} --- Energies for fake thermal vibrations}
%
  This parameter specifies the ion energies where the pseudo-temperatures 
  \texttt{TEMPFAKE} are defined. See \texttt{TEMPFAKE} for further information.
  The units of \texttt{EFAKE} are keV.
  \begin{keytab}
    Type:    \> index variable array (\texttt{ATOM}) of simple array (2) of real \\
    Default: \> ~ -- \\
    Range:   \> $> 0$
  \end{keytab}
\end{keydescription}

\begin{keydescription}{\texttt{ESURF} --- Surface binding energy}
%
  This parameter specifies the surface binding energy of the pure material in
  eV. Default values are at least 0.1~eV, since lower surface binding energies 
  lead to excessive simulation times. The user may explicitly set any 
  positive surface binding energies, though, in the input file.
  \begin{keytab}
    Type:    \> index variable array (\texttt{ATOM}) of real \\
    Default: \> from internally tabluated values \\
    Range:   \> $> 0$
  \end{keytab}
\end{keydescription}

\begin{keydescription}{\texttt{MASS} --- Atom mass}
%
  This parameter specifies the mass of the atom in units of the
  atomic mass unit.  It overwrites the default values determined by
  the atom names. NOTE: Due to a limitation of the code, ion atom
  masses specified on \texttt{\&ATOMS} are currently ignored. Use 
  \texttt{MASS} on the \texttt{\&IONS} record instead to define ion
  atom masses.
%
  \begin{keytab}
    Type:    \> index variable array (\texttt{ATOM}) of real \\
    Default: \> determined from \texttt{NAME} \\
    Range: \> $> 0$
  \end{keytab}
\end{keydescription}

\begin{keydescription}{\texttt{NAME} --- Name of atom}
%
  This parameter specifies the chemical formula of the atom.
%
  \begin{keytab}
    Type:    \> index variable array (\texttt{ATOM}) of character(2) \\
    Default: \> ~ -- ~~~ (obligatory) \\
    Range:   \> any chemical name of an atom
  \end{keytab}
\end{keydescription}

\begin{keydescription}{\texttt{TDEBYE} --- Debye temperature}
%
  This parameter specifies the Debye temperature of the pure material. It is used
  for calculation of vibrational amplitude using Debye function.  
  \texttt{TDEBYE=0} means a thermal vibration amplitude of 0.083 will be used. 
  Note that some internally tabulated values (e.g. for noble gases) are zero.
  The units of \texttt{TDEBYE} are K.
%
  \begin{keytab}
    Type:    \> index variable array (\texttt{ATOM}) of real \\
    Default: \> internally tabulated values \\
    Range:   \> $> 0$
  \end{keytab}
\end{keydescription}

\begin{keydescription}{\texttt{TEMPFAKE} --- Temperatures for fake thermal vibrations}
%
  This parameter specifies the pseudo-temperatures to be used for the 
  calculation of the amplitudes of the fake thermal vibrations. Each value 
  corresponds to an ion energy as specified by the parameter \texttt{EFAKE}.
  If for an atom index \texttt{ATOM} no value of \texttt{TEMPFAKE} is 
  specified, fake thermal vibrations are disabled. If for an atom index 
  \texttt{ATOM} only one value of \texttt{TEMPFAKE} is specified, the 
  amplitude of the fake thermal vibrations is constant. If two values are
  specified, the amplitude of the fake thermal vibrations is assumed to
  depend on the ion energy $E$ according to a power law such that for 
  $E=$\texttt{EFAKE(1)} the amplitude corresponds to a temperature 
  \texttt{TEMPFAKE(1)} and for $E=$\texttt{EFAKE(2)} the amplitude 
  corresponds to a temperature \texttt{TEMPFAKE(2)} according to the
  Debye model. Thus, if two values of \texttt{TEMPFAKE} are given,
  two values of \texttt{EFAKE} must be specified as well. Note that the
  larger of the physical and fake thermal vibration amplitude is used and
  fake thermal vibrations are ignored in amorphous materials. The units of 
  \texttt{TEMPFAKE} are K.
  \begin{keytab}
    Type:    \> index variable array (\texttt{ATOM}) of simple array (2) of real \\
    Default: \> ~ -- ~~~ (obligatory if \texttt{EFAKE} is defined) \\
    Range:   \> $> 0$
  \end{keytab}
\end{keydescription}

\begin{keydescription}{\texttt{VIB} --- Thermal vibration amplitude}
%
  This parameter specifies the mean square amplitude in \AA\ of
  vibrating lattice atoms in a given direction. The displacements of
  the atoms from their lattice sites is Gaussian distributed.
%
  \begin{keytab}
    Type:    \> index variable array (\texttt{ATOM}) of real \\
    Default: \> Calculated from \texttt{TDEBYE} \\
    Range:   \> $\ge 0$
  \end{keytab}
\end{keydescription}
