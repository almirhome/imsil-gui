Ions leaving the target on the same side as they entered it, are called
\textbf{backscattered} or reflected. Ions leaving a membrane on the opposite
side of where they entered it, are called transmitted. In a conventional 1-D
membrane setup, the ions impinge on the top surface (lowest $z$ coordinate of the
target) and may be emitted from that surface with a negative $z$ component of
their direction of motion. Conversely, if they are emitted from the bottom
surface, they have a positive $z$ component of their direction of motion. IMSIL
takes the sign of the $z$ component of the direction of motion as the criterion
for distinguishing between ``backscattered'' and ``transmitted'' atoms. Note
that this distinction may not always make sense, e.g., when ions can be emitted
from the sides of a 2-D or 3-D target. In those cases it might be more
reasonable to just add backscattered and transmitted ions.

Target atoms that are followed as recoils and leave the target, are called
\textbf{sputtered}. They are also classified into backscattered and transmitted
recoils, which corresponds to backward and forward sputtered atoms. There is no
difference in treating ions and recoils with respect to their leaving the
target. Therefore, their statistics is alway reported alongside.

When accurate sputtering characteristics are desired, the ``accurate sputtering
flag'' should be set. Then recoils are followed down to the surface binding
energy $E_\mathrm{surf}$ rather than down to the displacement energy
$E_\mathrm{d}$, if they are closer to the surface than some distance
$d_\mathrm{near}$. Within this distance, recoils with energies less than the
displacement energy are followed as ``virtual'' recoils. Virtual recoils deposit
energy like real recoils and are counted as sputtered when they leave the
target, but they generate defects (vacancies at the starting point, and
interstitials at the end point if the recoil reenters the target) only when they
leave the target.
  
With the accurate sputtering flag set, also the maximum impact parameter
$p_\mathrm{max}$ is adjusted in the near-surface zone such that no collisions
with energy transfer larger than the surface binding energy are missed.

When the ions enter the target, they gain the surface binding energy, and ions
and recoils loose the surface binding energy when they are emitted. Depending on
the model of the potential barrier, they may also be refracted. A planar
potential barrier \cite{I6909} is usually a good choice and is therefore the
default. The potential barrier model can be changed to isotropic, where there is
no refraction, or to an intermediate model \cite{hobler_assessment_2013} using a
``nonplanarity parameter''.

When a recoil reaches the surface with an energy not sufficient to surmount the
potential barrier, it is reflected. With the planar potential model, it may
happen that the recoil bounces back and forth beween the barrier (which is
at $p_\mathrm{max}$ outside the target) and the target. The number of
reflections may be limited.

\begin{center}
\begin{tabular}{lll}
   parameter \quad          & IMSIL name & to be specified in record \\
   \hline
   accurate sputtering flag                 & {\tt LSPUT}   & {\tt \&SETUP} \\
   surface binding energy $E_\mathrm{surf}$ & {\tt ESURF}   & {\tt \&DAMAGE} \\
   displacement energy $E_\mathrm{d}$       & {\tt ESURF}   & {\tt \&DAMAGE} \\
   $d_\mathrm{near}$                        & {\tt DNEAR}   & {\tt \&DAMAGE} \\
   nonplanarity parameter                   & {\tt KSURF}   & {\tt \&DAMAGE} \\
   maximum number of reflections         & {\tt REFLECTMAX} & {\tt \&DAMAGE} \\
\end{tabular}
\end{center}

