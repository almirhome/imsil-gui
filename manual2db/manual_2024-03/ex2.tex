The following example shows the \texttt{INP} file for the simulation of an
RBS/C spectrum into a 1 $\mu$m thick (100) crystalline silicon with a damage
profile defined in a histogram input file. Damage is in the form of
split-$\langle 110 \rangle$ interstitials. The coordinates of the defects and
strained regions are taken from ab-initio calculations. A silicon cell
containing 1.6 million lattice atoms (10x10x2000 unit cells) is populated
according to the requested damage concentrations (defined in histogram file
\texttt{'hisfile'}) with required defect type (\texttt{FRACSPL=1.0}). Periodic
boundary conditions in lateral dimensions are applied. 3~MeV He$^+$ ions and a
backscattering angle of $170^\circ$ are used.  Recoils are not followed.

\begin{verbatim}
IMSIL test example2
RBS/C (100) spectra of 3 Mev He ions. 
 &setup    nr=1 natom=2 usehis=t hisfile='profile.his' 
           atom1=2 atom2=2 / 
 &ions     name='He' energy=3000 dose=1e14 ranrot=t 
           nion=100000 /
 &material region=1 name='Si' vsurf=1,1,0 wafer=0,0,1 xtal='yes' /
 &damage   ldam=t ldet=t fracspl=1.0 wcol=40,40,8000 /
 &geometry posif=0,10000 /
 &output   lrbs=t lhis=t ldef=t wboxei=6000. tilta=10. /
\end{verbatim}

As discussed in Chapter \ref{k:input}, the first two lines are comments and
have no further meaning. The third line
%
\begin{verbatim}
 &setup    nr=1 natom=2 usehis=t hisfile='profile.his' 
           atom1=2 atom2=2 / 
\end{verbatim}
%
specifies that there is 1 region and a total of 2 atom species (the He ion
plus the Si target atoms, see below).  The damage profile is read from the
histogram file \texttt{profile.his} (\texttt{usehis=t, hisfile='profile.his'}.
The histogram file can be the output of a previous simulation or can be
created separately, with a format defined in Chapter \ref{k:files}. The atom
index \texttt{ATOM1=2} (column number three) corresponds to atom index in the
histogram file which will be used to initialize atom index index 2
\texttt{ATOM2=2}, which is in this case silicon. The next line
%
\begin{verbatim}
 &ions     name='He' energy=3000 dose=1e14 ranrot=t
           nion=100000 /
\end{verbatim}
%
defines the properties of the ions: The ion species is He, the energy
3000~keV, the dose $10^{14}~$cm$^{-2}$, the tilt angle is $0^\circ$ (default
value) and random rotation of ions is used. 100000 projectiles are requested
to be simulated.  The next record
%
\begin{verbatim}
 &material region=1 name='Si' vsurf=1,1,0 wafer=0,0,1 xtal='yes' /
\end{verbatim}
%
specifies the material of the target region as single-crystal (100)-Si 
(\texttt{wafer=0,0,1}).

The next record,
%
\begin{verbatim}
 &damage   ldam=t ldet=t fracspl=1.0 wcol=40,40,8000 /
\end{verbatim}
%
specifies that damage in the crystalline region shall be taken into account
dynamically for the simulation of subsequent trajectories, that the
deterministic model of damage is used, and that the defect type is always
split-$\langle 110 \rangle$ interstitials (\texttt{FRACSPL=1.0}). Since we use
the histogram damage profile \texttt{USEHIS=T} the simulation cell with size
of 10x10x2000 unit cells (notice that the input units for \texttt{WCOL} are a
quarter of the lattice constant) will be populated with
split-$\langle110\rangle$ interstitials \texttt{FRACSPL=1} according to
concentration profile specified with atom index 2 \texttt{ATOM1=2} in
histogram file \texttt{'profile.his'}.

Since there are no \texttt{\&XTAL} \texttt{\&SNPAR}, or \texttt{\&SEPAR}
records, the default parameters for the crystal, nuclear, and electronic
stopping are used.

The next input record
%
\begin{verbatim}
 &geometry posif=0,10000 /
\end{verbatim}
%
specifies that the target geometry is a 1-D layered structure (the default value
of \texttt{ndim} in the \texttt{\&SETUP} record is \texttt{1}). The Si layer (as
specified in the \texttt{\&MATERIAL} record) extends from $z=0$ to
$z=10000$\AA\ ($=1~\mu$m). Finally,
%
\begin{verbatim}
 &output lrbs=t lhis=t ldef=t wboxei=6000. tilta=10. /
\end{verbatim}
%
specifies the RBS energy spectrum box width wboxei=6000.\texttt{eV}. The RBS
backscattering angle is $170^\circ$. Notice that when \texttt{LRBS=T} is set the
angle is calculated as $\texttt{$180^\circ$} - \texttt{TILTA}$ which gives
$180^\circ - 10^\circ = 170^\circ$). Requested output files are: 1-D
histograms of the dopants and the damage into the \texttt{HIS} file,
RBS/S spectra into the \texttt{RBS} file and deterministic defects into the
\texttt{DEF} file.

If the script delivered with the program is used, the input file could be
named \texttt{he3mev.inp}. IMSIL would produce the files \texttt{he3mev.out},
\texttt{he3mev.his}, \texttt{he3mev.rbs}, and \texttt{he3mev.def}.

