This record is used to set the wafer orientation and the parameters for 
the search procedure of collision partners in the crystalline region. 

\begin{keydescription}{\texttt{FFOLDM} --- Maximum flight path to record old 
lattice sites}
%
This parameter specifies the flight path within which lattice sites may be 
recorded to be excluded from the list of potential collision partners.
Units are \AA.
\begin{keytab}
   Type:    \> real \\
   Default: \> 0.5 \\
   Range:   \> $> 0$
\end{keytab}
\end{keydescription}

\begin{keydescription}{\texttt{FFPMAX} --- Maximum free flight path}
%
This parameter specifies the maximum free flight path between collisions.
The purpose of the maximum free flight path is to avoid skipping of 
scattering events due to the finite range of the lattice site list. The 
units are a quarter of the lattice constant for the default diamond lattice,
and the minimum of the lattice constants otherwise.
\begin{keytab}
   Type:    \> real \\
   Default: \> 1.0 for default diamond lattice \\
            \> 0.25 otherwise \\
   Range:   \> $\ge 0$
\end{keytab}
\end{keydescription}

\begin{keydescription}{\texttt{LVIB2} --- 2-D lattice vibrations flag}
%
This parameter specifies whether the lattice vibrations are assumed 2-D
within the plane perpendicular to the ion's direction of motion, or
3-D. \texttt{LVIB2=T} reduces the execution time.
\begin{keytab}
   Type:    \> logical \\
   Default: \> \texttt{F} \\
   Range:   \> \texttt{T}, \texttt{F} 
\end{keytab}
\end{keydescription}

\begin{keydescription}{\texttt{NAME} --- Name of crystal lattice}
%
This parameter specifies the name of the crystal lattice and thus the
crystal lattice system. Note that the hexagonal lattice structures use
orthogonal non-primitive unit cells, the edges of which are reported
in the output file. Among the diamond lattices the optimized version
(\texttt{' '}) is the most efficient implementation. For the other lattices the
alterative names all refer to the same implementation. Alternatively,
\texttt{NAME} may specify the name of a file containing the unit cell definition
(see Chapter \ref{k:files} for details).
\begin{keytab}
   Type:    \> character string \\
   Default: \> \texttt{' '} ... (optimized) diamond \\
   Range:   \> \texttt{' '}, \texttt{'diamond'}, \texttt{'diamond2'} ... diamond \\
            \> \texttt{'sc'} ... simple cubic \\
            \> \texttt{'bcc'} ... body centered cubic \\
            \> \texttt{'fcc'} ... face centered cubic \\
            \> \texttt{'zincblende'}, \texttt{'3C'} ... zincblende \\
            \> \texttt{'wurtzite'}, \texttt{'wurzite'}, \texttt{'2H'}
            ... wurtzite \\
            \> \texttt{'4H'} \\
            \> \texttt{'6H'} \\
            \> Any name ($\le 80$ characters) of an IMSIL unit cell file \\
\end{keytab}
\end{keydescription}

\begin{keydescription}{\texttt{NLOLDM} --- Maximum number of old lattice sites}
%
This parameter specifies the maximum number of old lattice sites to be 
excluded from the list of potential collision partners.
\begin{keytab}
   Type:    \> integer \\
   Default: \> 6 \\
   Range:   \> $0 \le \texttt{NLOLDM} \le 20$
\end{keytab}
\end{keydescription}

\begin{keydescription}{\texttt{NRAD2} --- Search radius squared}
%
\texttt{NRAD2+1} is the square of the radius of a sphere around the ion 
position which is assumed to contain the lattice sites of all potential 
collision partners. \texttt{NRAD2} may only be specified for the default
diamond lattice. The units are a quarter lattice constant.
\begin{keytab}
   Type:    \> integer \\
   Default: \> $\texttt{PMAX}^2 + \texttt{FFPMAX}^2$ \\
   Range:   \> $0 \le \texttt{NRAD2} \le 32$
\end{keytab}
\end{keydescription}

\begin{keydescription}{\texttt{PMAX} --- Maximum impact parameter}
%
This parameter specifies the maximum impact parameter in the crystalline
layer. The units are a quarter of the lattice
constant for the default diamond lattice, and the minimum of the lattice
constants otherwise. 
\begin{keytab}
   Type:    \> real \\
   Default: \> for default diamond lattice: \\
            \>$(\pi~L~\texttt{DENS})^{-1/2}$ if \texttt{NRAD2} is undefined, with\\
            \> $L = 0.43~\texttt{DENS}^{-1/3}$ for $E>10~$keV \\
            \> $L = (0.3+0.7~\texttt{E0}/10~$keV$)~0.43~\texttt{DENS}^{-1/3}$
               for $E<10~$keV; \\
            \> $[\texttt{NRAD2}+1-\texttt{FFPMAX}^2)]^{1/2}$ 
               otherwise \\
            \> for \texttt{NAME='diamond'}: 0.43 \\
            \> for other lattice structures: ~ -- ~~~ (obligatory) \\
   Range:   \> $>0$
\end{keytab}
\end{keydescription}

\begin{keydescription}{\texttt{PSOLDM} --- Maximum scattering angle to record 
old lattice sites}
%
Lattice sites are excluded from the sites of potential collision 
partners only if no collision with a scattering angle larger than 
\texttt{PSOLDM} has occurred meanwhile. \texttt{PSOLDM} has to be specified in 
degrees.
\begin{keytab}
   Type:    \> real \\
   Default: \> $30^\circ$ \\
   Range:   \> $\ge 0$
\end{keytab}
\end{keydescription}

