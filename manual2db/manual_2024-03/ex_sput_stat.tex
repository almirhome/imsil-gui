This example demonstrates a static sputtering simulation of Si by Ar ions.
``Static'' means that the target is not modified during the simulation.

The input file reads:

\begin{verbatim}
Sputtering of a-Si, static mode
Ar on a-Si, 10keV, 60 tilt
 &setup    ndim=1 natom=2 nr=1 ldyn=f lsput=t nion=100000 /
 &ions     name='Ar' energy=10 dose=1 tilt=60 diverg=0 /
 &material region=1 name='Si' xtal='no' /
 &snpar    chkffpcyl=t /
 &damage   ldam=f lrcoil=t esurf=4.1 ed=15 /
 &geometry posif=0,1000 /
 &output   lhis=t wbox=20 nbox=200 lmomab=t lhisab=t wboxa=1 /
\end{verbatim}

On the \texttt{\&SETUP} record, \texttt{lsput=t} is specified, indicating that
accurate sputtering simulation is requested. On the \texttt{\&DAMAGE} record,
the surface binding energy and the displacement energy are specified. Since
accurate sputtering simulation is requested, free flight paths will be chosen
such that no recoils with energy larger than the surface binding energy are
disregarded near the surface. Deeper in the bulk, larger free flight paths are
chosen, as recoils with energy up to the displacement energy can be neglected.

Since accurate damage distributions are not of interest in this case,
\texttt{ldam=f} is set on the \texttt{\&DAMAGE} record. As a consequence, damage
distributions are calculated individually by each thread in a multi-threaded run
and merged only at the end of the simulation. This does not affect the results,
as the implantation dose of $1~\mathrm{cm}^{-2}$ is (unrealistically) low and
damage remains weak throughout the simulation. As a benefit, since the merging
of the histograms is done only at the end of the simulation, the computational
overhead of the merge operation is largely avoided.

The sputter yield is written to the \texttt{OUT} file. In addition, the angular
distribution of the backward sputtered atoms is written to the \texttt{HISAB}
file (\texttt{lhisab=t}). The box width of the histogram is 1\textdegree\
(\texttt{wboxa=1}).
