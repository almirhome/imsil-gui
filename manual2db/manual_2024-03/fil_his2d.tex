\bigskip
%
\begin{center}
\begin{tabular}{|l|p{0.65\textwidth}|l|}
\hline
Name       & File content                                               & flag \\
\hline
\tt HIS2   & 2-D spatial distribution of stopped atoms                  & \tt LHIS2 \\
\tt HISEE2 & 2-D spatial distribution of deposited electronic energy    & \tt LHISEE2 \\
\tt HISNE2 & 2-D spatial distribution of deposited nuclear energy       & \tt LHISNE2 \\
\tt HIS2B  & 2-D spatial distribution of backscattered atoms            & \tt LHIS2B \\
\tt HISA2B & 2-D direction cosine distribution of backscattered atoms   & \tt LHISA2B \\
\tt HIS2T  & 2-D spatial distribution of transmitted atoms              & \tt LHIS2T \\
\tt HISA2T & 2-D direction cosine distribution of transmitted atoms     & \tt LHISA2T \\
\hline
\end{tabular}
\end{center}

\bigskip

The 2-D histogram files are written by the following FORTRAN statements:
%
\begin{verbatim}
      CHARACTER*80 TEXT
      WRITE (22) TEXT
      WRITE (22) NBOXZ,NBOXX,NSPEC
      WRITE (22) (Z(IBOXZ),IBOXZ=1,NBOXZ)
      WRITE (22) (X(IBOXX),IBOXX=1,NBOXX)
      DO I=1,NSPEC
         WRITE (22) ((HIST(I,IBOXX,IBOXZ),IBOXX=1,NBOXX),IBOXZ=1,NBOXZ)
      ENDDO
\end{verbatim}
%
\texttt{TEXT} is a character string containing information about the stored
histogram. \texttt{NSPEC} has the same meaning as for the 1-D histograms.
\texttt{NBOXZ} and \texttt{NBOXX} denote the number of histogram boxes in $z$
and $x$ direction, respectively. Regarding adaptation of the histogram during
the simulation, the same applies as for the 1-D histograms.

\texttt{Z(IBOXZ)} and \texttt{X(IBOXX)} denote the $z$ and the $x$ coordinates,
respectively, of the center of the histogram box with indices \texttt{IBOXZ} and
\texttt{IBOXX}.  If \texttt{LCYL=F} is set on the \texttt{\&OUTPUT} record,
\texttt{HIST(I,IBOXX,IBOXZ)} is the concentration (energy density in case of
\texttt{LHISEE2=T} or \texttt{LHISNE2=T}) of histogram \texttt{I} in the box
(\texttt{IBOXZ},\texttt{IBOXX}). If output in cylinder coordinates is desired
(\texttt{LCYL=T}), \texttt{X} represents the radius $r$ and \texttt{HIST} the
concentration (energy density) times $2r\pi$.
\ifprivate A similar format (with \texttt{NSPEC=1}) is used for the
\texttt{HISIV2} file. \fi 

The units of \texttt{X} and \texttt{Z} are [\AA], while the units of
\texttt{HIST} are [cm$^{-3}$] in the \texttt{HIS2} file and [eV/\AA$^3$] in the
\texttt{HISEE2} and \texttt{HISNE2} files, if the dose is specified as a line
dose (\texttt{DOSEUNITS='cm-1'}) or can be converted to a line dose. Otherwise,
the units will be different. For instance, when an area dose is specified and
the extent of the implant area in $x$ direction is zero
(\texttt{XINIT(1)=XINIT(2)}), the line dose is undefined. The quantity stored
in the \texttt{HIS2} file then has units of [cm$^{-3}$\AA$^{-1}$] and in the
\texttt{HISEE2} and \texttt{HISNE2} files [eV/\AA$^4$]. They may be interpreted
as point responses, and actual concentration or energy densities are obtained
by integration over x[\AA]. When the default dose of $\texttt{DOSE} =
10^{16}~$cm$^{-2} =  1~\mathrm{\AA}^{-2}$ is used, the point responses may be
interpreted as the yield times the probability density function (units
[10$^{-24}$\AA$^{-2}$] and [eV/\AA$^2$], respectively).

The \texttt{HIS2B} and \texttt{HIS2T} files contain the spatial distributions
of the ejected atoms as a function of the lateral coordinates $x$ and $y$ for
backscattered and transmitted atoms, respectively. (In the code snippet given
above, \texttt{Z} has to be replaced by \texttt{Y}). The number of species
listed, \texttt{NSPEC}, is the same as for the spatial atom distributions inside
the target, stored in the \texttt{HIS2} file. For the first \texttt{N1+N2}
columns, the coordinates $x$ and $y$ refer to the position where the atom leaves
the surface boundary layer, which normally is a few {\AA}ngstr{\o}ms above the
surface. For the last \texttt{N2} columns, $x$ and $y$ refer to the position
where the ejected atoms originate from. The histogram values are the yield times
the probability density function (units [\AA$^{-2}$]).

The \texttt{HISA2B} and \texttt{HISA2T} files contain the distributions of the
direction cosines $dir_\mathrm{x}$ and $dir_\mathrm{y}$ of backscattered and
transmitted atoms, respectively. Depending on the \texttt{LRCOIL} parameter of
the \texttt{\&DAMAGE} record, data for \texttt{NSPEC=N1} or \texttt{NSPEC=N1+N2}
atom species are written to the files. The histogram values are the yield times
the probability density function (units 1).
