\bigskip
%
\begin{center}
\begin{tabular}{|l|p{0.6\textwidth}|l|}
\hline
Name       & File content                                 & flag \\
\hline
\tt HIS3   & 3-D histogram of stopped atoms               & \tt LHIS3 \\
\tt HISEE3 & 3-D histogram of deposited electronic energy & \tt LHISEE3 \\
\tt HISNE3 & 3-D histogram of deposited nuclear energy    & \tt LHISNE3 \\
\hline
\end{tabular}
\end{center}

\bigskip

The 3-D histogram files are written by
%
\begin{verbatim}
      CHARACTER*80 TEXT
      WRITE (22) TEXT
      WRITE (22) NSPEC,NBOXX,NBOXY,NBOXZ
      WRITE (22) (X(IBOXX),IBOXX=1,NBOXX)
      WRITE (22) (Y(IBOXY),IBOXY=1,NBOXY)
      WRITE (22) (Z(IBOXZ),IBOXZ=1,NBOXZ)
      DO, I=1,NSPEC
         WRITE (22) (((HIST(I,IBOXX,IBOXY,IBOXZ),  &
                       IBOXX=1,NBOXX),IBOXY=1,NBOXY),IBOXZ=1,NBOXZ)
      ENDDO
\end{verbatim}
%
\texttt{TEXT} is a character string containing information about the stored
histogram. \texttt{NBOXX}, \texttt{NBOXY}, and \texttt{NBOXZ} denote the number
of histogram boxes in $x$, $y$, and $z$ direction, respectively. Regarding
adaptation of the histogram during the simulation, the same applies as for the
1-D and 2-D histograms. 

\texttt{X(IBOXX)}, \texttt{Y(IBOXY)}, and \texttt{Z(IBOXZ)} denote the $x$, $y$
and $z$ coordinate of the center of the histogram box (\texttt{IBOXX},
\texttt{IBOXY}, \texttt{IBOXZ}). \texttt{HIST(I,IBOXX,IBOXY,IBOXZ)} is the
concentration of the histogram for species \texttt{I} in the box
(\texttt{IBOXX},\texttt{IBOXY},\texttt{IBOXZ}).

The units of \texttt{X}, \texttt{Y} and \texttt{Z} are [\AA], while the units of   
\texttt{HIST} are [cm$^{-3}$] in the \texttt{HIS3} file and [eV/\AA$^3$] in the
\texttt{HISEE3} and \texttt{HISNE3} files, if the dose is specified as an ion
count (\texttt{DOSEUNITS='1'}) or can be converted to an ion count. Otherwise, 
for instance, when the extent of the implant area in both $x$ and $y$ direction 
equals zero and an area dose is specified, the units are [cm$^{-3}$\AA$^{-2}$]
in the \texttt{HIS3} file and [eV/\AA$^5$] in the \texttt{HISEE3} and 
\texttt{HISNE3} files, so when integrated over x[\AA] and y[\AA] one obtains
[cm$^{-3}$] and [eV/\AA$^3$], respectively. When the default
$\texttt{DOSE} = 1$ is used, the point response may be interpreted as the yield
times the probability density function (units [10$^{-24}$\AA$^{-3}$] and
[eV/\AA$^3$], respectively).
