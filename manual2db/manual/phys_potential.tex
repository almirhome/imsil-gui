The standard model of the interatomic potential used in IMSIL is the 
universal ZBL potential \cite{I8512}. It reads 
%
\begin{equation}
   V(r) = \frac{Z_1 Z_2 e^2}{r} \cdot \Phi(\frac{r}{a_{12}}), \qquad
   \Phi(R) = \sum_{i=1}^4 a_i \cdot \exp(-b_i R)
   \label{eq:ZBLfun}
\end{equation}
%
$Z_1$ and $Z_2$ denote the atomic numbers of the ion and the target atom, 
respectively, $r$ the interatomic separation, and $a_{12}$ the screening length
%
\begin{equation}
   a_{12} = \frac{0.468~\mbox{\AA}}{Z_1^{0.23} + Z_2^{0.23}}
   \label{eq:ZBLscr}
\end{equation}
%
The coefficients of the screening function $\Phi$ are given by 
$a_i =~$\{0.1818, 0.5099, 0.2802, 0.02817\} and   
$b_i =~$\{3.2,    0.9423, 0.4029, 0.2016\} \cite{I8512}. The scattering angles
and the energy transfer from projectile to recoil are calculated, as usual, from
the scattering angle in the center-of-mass system $\theta$. In the default
model, $\theta$ is calculated by bicubic interpolation in tables of
$\cot(\theta/2)$ \cite{I8730}. Such tables may be set up, in principle, for any
interatomic potential. However, the screening length is assumed according to
Eq.~\ref{eq:ZBLscr} in any case. The scattering table file corresponding to the
universal ZBL potential is called \texttt{SCATTAB}.

The universal ZBL screening function has been obtained by averaging a large 
number of ``pair-specific'' interatomic potentials (i.e. for specific
projectile-recoil atom combinations) \cite{I8512}. Tables for some of the
pair-specific potentials are also provided with the program.

Alternatively, the scattering angle $\theta$ in the center-of-mass system can be 
determined by numerical integration using Gauss-Legendre or Gauss-Mehler
quadrature. The screening function may either be assumed as a sum of
exponentials (Eq.~\ref{eq:ZBLfun}) or may be defined by a table. Tables for the
screening functions of all pair-specific potentials are provided.

\begin{center}
\begin{tabular}{lll}
   parameter \quad & IMSIL name \qquad\qquad & to be specified in record \\
   \hline
   $Z_1$           & derived from ion name      & -- \\
   $M_1$           & derived from ion name or 
                     \texttt{MASS}              & \texttt{\&ATOMS} \\
   $Z_2$           & derived from material name & -- \\
   $M_2$           & derived from material name or
                     \texttt{MASS}              & \texttt{\&ATOMS} \\
   scattering 
   table file      & \texttt{SCATFILE}          & \texttt{\&SNPAR} \\
   screening
   coefficients    & \texttt{COEFFILE}          & \texttt{\&SNPAR} \\
   screening 
   function table  & \texttt{SCRFILE}           & \texttt{\&SNPAR} \\
\end{tabular}
\end{center}

The format of the files \texttt{SCATFILE}, \texttt{COEFFILE}, and 
\texttt{SCRFILE} is described in Section~\ref{s:input}. 

