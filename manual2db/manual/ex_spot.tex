This example shows the simulation of an ideally focused ion beam on a
crystalline Si target covered by a thin SiO$_2$ layer.

The input file reads:

\begin{verbatim}
B spot burn in 100-Si, 2 keV, 7 tilt, dose=1
 &setup    natom=3 nr=2 / 
 &ions     name='B' energy=2 dose=1 doseunits='1'
           tilt=7 diverg=0.3 nion=1000000 /
 &material region=1 name='SiO2' xtal='no' /
 &material region=2 name='Si' xtal='yes' wafer=0,0,1 vsurf=1,0,0 /
 &snpar    lstffp=f /
 &damage   ldam=t lrcoil=t frec=0.3 ndim=3 /
 &geometry posif=0,10,10000 /
 &output   lmom=t lhis=t lhis2=t lhis3=t 
           nbox=200 wbox=10
           nbox2=100,50 wbox2=20,20 
           nbox3=0,0,0 wbox3=20,20,20 /
\end{verbatim}

The dose is specified as one ion (\texttt{dose=1 doseunits='1'} on the
\texttt{\&setup} record). This means that the 3-D histograms will represent the
probability density functions for the stop position of the ions, and for the
positions of the vacancies and interstitials. Since \texttt{xinit},
\texttt{yinit}, and \texttt{zinit} are not specified, these parameters assume
their default value \texttt{0}, so the reference point will be the origin of the
coordinate system for all ions.

Note that in this example the $x$ axis is parallel to the [100] crystal axis
(\texttt{vsurf=1,0,0} on the second \texttt{\&material} record).
\texttt{ndim=3} on the \texttt{\&damage} record means that for dynamic damage
consideration the 3-D damage distribution will be taken into account (``damage
consideration'' means that atoms in the crystalline region are randomly
displaced, which normally results in a reduction of the channeling effect).
This corresponds to the actual situation in a spot burn. If \texttt{ndim=1} were
specified, a 1-D damage distribution would be taken into account derived from
the implant simulation. Damage-wise this would correspond to an implantation
with a homogeneous beam. The 3-D histograms would then correspond to the
probability density functions for one ion in a homogeneous beam.

On the \texttt{\&output} record, output of 1-D, 2-D, and 3-D histograms are
requested. Only the 3-D histogram corresponds to the actual physical situation,
since the 3-D damage distribution is taken into account. The 1-D and 2-D
histograms are just convolutions of the 3-D histogram and do not correspond to
line or area implantations.

On the \texttt{\&output} record, also the two possibilities for specifying the
box sizes of the histograms are demonstrated. For the 1-D and 2-D histograms,
\texttt{nbox} and \texttt{nbox2} are nonzero. This means that the number of
histogram boxes will never exceed the specified values. The box widths are
initialized with the values given by the \texttt{wbox} and \texttt{wbox}
parameter, respectively. The histogram is moved and/or the box widths are
doubled so that all entries are captured. In contrast, \texttt{nbox3=0}, which
means the histgram will always have the box widths specified by \texttt{wbox3},
and boxes will be added as to cover all entries.
