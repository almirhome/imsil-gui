There are the following input files:
\bigskip
%
\begin{center}
\begin{tabular}{|l|p{0.5\textwidth}|l|l|}
\hline
(Default) &                   & \multicolumn{2}{c|}{file name specified by} \\
Name      & File contents, remarks     & parameter  & in record \\
\hline
\tt INP         & Input parameters                         & --           & -- \\
\tt (-)         & \raggedright Geometry file (used if \texttt{NDIM=2} and
\texttt{GEOMFILE} specified) & \texttt{GEOMFILE} & \tt \&GEOMETRY \\ 
\tt (SCATTAB)   & \raggedright Scattering table file (used if \texttt{QUAD='no'} in record 
                  \texttt{\&SNPAR})                        & \tt SCATFILE & \tt \&SNPAR \\
\tt (ZBL)       & \raggedright Screening function file (used if 
                  \texttt{QUAD$\ne$'no'} and if \texttt{SCRFUN='scrfile'} in record 
                  \texttt{\&SNPAR})                        & \tt SCRFILE  & \tt \&SNPAR \\
\tt (ZBLspec)   & \raggedright Screening coefficient file (used if \texttt{QUAD$\ne$'no'} and 
                  if \texttt{SCRFUN='coeffile'} in record \texttt{\&SNPAR})
                                                           & \tt COEFFILE & \tt \&SNPAR \\
\tt (-)         & Random electronic stopping file (used if \texttt{MODEL=7} in
                  record \texttt{\&SEPAR})                 & \tt SEFILE   & \tt \&SEPAR \\
\tt (-)         & Definition of the crystal lattice unit cell  
                                                           & \tt NAME     & \tt \&CRYSTAL \\
\tt (BCK)       & Backup = data necessary to continue an old run (used if \texttt{USEBK=T} in 
                  record \texttt{\&SETUP})                 & \tt BKFILE   & \tt \&SETUP \\
\tt (-)         & Histogram = histogram profiles necessary to initialize 1-D
                  histogram of desired atom index (used if \texttt{USEHIS=T} in
                  record \texttt{\&SETUP})                 & \tt HISFILE  & \tt \&SETUP \\
\tt (-)         & Cell file = 1-D cell-based description of the target
                  containing the densities of each species (used if
                  \texttt{USECELL=T} in record \texttt{\&SETUP}) 
                                                           & \tt CELLFILE & \tt \&SETUP \\
\ifprivate
\tt (DEF)       & Defects = Deterministic defecs file containing the defects to be used  (used 
                  if \texttt{USEDEF=T} in record \texttt{\&SETUP})
                                                           & \tt DEFFILE  & \tt \&SETUP \\
\fi
\tt (-)         & Trajectory file used to initialize the ions (used if \texttt{USETRA=T})
                                                           & \tt TRAFILE  & \tt \&ION \\
                                       
\hline
\end{tabular}
\end{center}

\bigskip

The format of the \texttt{INP} file is descibed in detail in
Chapter~\ref{k:input}. The geometry file is an input file that (usually)
contains only \texttt{\&GEOMETRY} records.

The scattering table file \texttt{SCATFILE} is assumed to be located in the
subdirectory \texttt{scattab} of the \texttt{tables} directory and is read by
the following FORTRAN statements:
%
\begin{verbatim}
      READ (22,*) NE, NP
      READ (22,*) (E(IE), IE=1,NE)
      READ (22,*) (P(IP), IP=1,NP)
      READ (22,*) PCUTOFF
      READ (22,*) ((COTTHH(IE,IP), IE=1,NE), IP=1,NP)
      READ (22,*) ((DCOTDE(IE,IP), IE=1,NE), IP=1,NP)
      READ (22,*) ((DCOTDP(IE,IP), IE=1,NE), IP=1,NP)
      READ (22,*) ((TAU(IE,IP), IE=1,NE), IP=1,NP)
      READ (22,*) ((DTAUDE(IE,IP), IE=1,NE), IP=1,NP)
      READ (22,*) ((DTAUDP(IE,IP), IE=1,NE), IP=1,NP)
\end{verbatim}
%
\texttt{NE} and \texttt{NP} denote the number of tabulated reduced energy and
reduced impact parameter values \texttt{E} and \texttt{P}, respectively. The
reduced energy is the energy divided by $(1+M_1/M_2)Z_1Z_2e^2/a_{12}$, the
reduced impact parameter is the impact parameter divided by $a_{12}$, with
$a_{12}$ the ZBL screening length according to Eq.~(\ref{eq:ZBLscr}).
\texttt{PCUTOFF} is the screened cut-off impact parameter, beyond which the
deflection angle will be set to zero.  \texttt{COTTHH} contains the tabulated
values of $\cot(\theta/2)$ with $\theta$ the scattering angle in the
center-of-mass system. \texttt{DCOTDE} and \texttt{DCOTDP} are the derivatives
of $\cot(\theta/2)$ with respect to reduced energy and reduced impact
parameter, respectively. \texttt{TAU} denotes the scaled time integral,
\texttt{DTAUDE} and \texttt{DTAUDP} its derivatives.

The screening function file \texttt{SCRFILE} is assumed to be located in the
subdirectory \texttt{screentab} of the \texttt{tables} directory and is read by
%
\begin{verbatim}
      DO I=1,N
         READ (22,*) R(I), PHI(I)
      END DO 
\end{verbatim}
%
\texttt{N} denotes the number of data points, \texttt{R} the internuclear
distance in {\AA}ngstr{\o}ms, and \texttt{PHI} the value of the screening
function $\Phi(r)$, see Eq.~(\ref{eq:ZBLfun}). If \texttt{SCRFILE='ZBL'}, then
the element specific ZBL screening functions provided in the subdirectory
\texttt{screentab/zbl} of the \texttt{tables} directory are used.

The screening coefficient file \texttt{COEFFILE} is assumed to be located in the
subdirectory \texttt{screencoefs} of the \texttt{tables} directory and is read
by
%
\begin{verbatim}
      DO
         READ (22,*) Z1, Z2, (A(I,J), B(I,J), I=1,4)
      END DO 
\end{verbatim}
%
until the desired atomic numbers \texttt{Z1} and \texttt{Z2} are found.
\texttt{A} and \texttt{B} are the coefficients $a_i$ and $b_i$, respectively, of
Eq.~(\ref{eq:ZBLfun}). Note that the number of terms in Eq.~(\ref{eq:ZBLfun}),
when used with \texttt{COEFFILE}, must be 4. 

The random electronic stopping power file \texttt{SEFILE} provides an option to
include measured stopping power data, or data obtained by an external program,
in the simulation. It is read by the following FORTRAN statements:
%
\begin{verbatim}
      DO, IE=0,NE
         READ (122,*) E(IE), DEDX(IE)
      ENDDO
\end{verbatim}
%
The file can begin with an arbitrary number of comment lines beginning with the
\texttt{*} character, which will be ignored when the file is read. \texttt{E}
denotes the energy [eV]. \texttt{DEDX} is the average energy loss per path
length [eV/\AA]. \texttt{NE} denotes the number of (\texttt{E},\texttt{DEDX})
points (maximum 300). Below \texttt{E(1)}, the electronic stopping is
interpolated linearly between zero energy/zero stopping and the first table
entry. Above \texttt{E(NE)}, electronic stopping is set to zero.

The file defining a crystal structure (\texttt{NAME} parameter of the
\texttt{\&CRYSTAL} record) is read by the following FORTRAN statements:
%
\begin{verbatim}
      READ (123,*) ALAT(1:3)
      READ (123,*) NSITE
      DO, ISITE=1,NSITE
         READ(123,*) X(ISITE), Y(ISITE), Z(ISITE), IELEM(ISITE)
      ENDDO
\end{verbatim}
%
\texttt{ALAT(1:3)} denote the lattice constants (edge lengths of the unit
cell), \texttt{NSITE} the number of lattice sites, and \texttt{X}, \texttt{Y},
\texttt{Z} the positions of the lattice sites. $0 \le \texttt{X} \le
\texttt{ALAT(1)}$, $0 \le \texttt{Y} \le \texttt{ALAT(2)}$, and $0 \le
\texttt{Z} \le \texttt{ALAT(3)}$ must hold.  \texttt{IELEM} denotes the element
number and must equal \texttt{1} for a pure  element or integers ranging from
\texttt{1} to the number of elements in a  compound. The lattice constants and
the lattice site coordinates  will be scaled such that the correct material
density as specified by the \texttt{DENSITY} parameter of the \texttt{MATERIAL}
record is obtained.

The backup file \texttt{BKFILE} stems from the old days, when simulations might
have taken weeks. Its purpose was to allow the continuation of interrupted
simulation. It may still be useful for debugging purposes. New simulations
should not rely on the \texttt{BKFILE}, since its full functionality is not
guarantteed to be maintained in the future. Note that histograms may be read in
from a histogram file with the \texttt{HISFILE} parameter of the \texttt{\&SETUP}
record. The format of the backup file is binary.

The other possible input files are also output files of IMSIL. Their format is
described in Sections~\ref{s:his1d} and \ref{s:filesmisc}.

