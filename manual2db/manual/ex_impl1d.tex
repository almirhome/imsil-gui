The following example shows the \texttt{INP} file for the simulation of an 
arsenic implantation into (100) silicon through a 10~\AA\ native oxide. All
recoils are followed.

\begin{verbatim}
IMSIL test example
Point defects after 10 keV As implantation
 &setup    nr=2 natom=3 /
 &ions     name='As' energy=10 dose=1e13 tilt=7 rotate=15 diverg=0.3 
           nion=10000 /
 &material region=1 name='SiO2' xtal='no' /
 &material region=2 name='Si' xtal='yes' /
 &damage   ldam=t lrcoil=t /
 &geometry posif=-10,0,100000 /
 &output   lmom=t lhis=t /
\end{verbatim}

As mentioned in Chapter \ref{k:input}, the first two lines appear outside any
input record and therefore have no further meaning. They serve as a comment. The
third line
%
\begin{verbatim}
 &setup    nr=2 natom=3 /
\end{verbatim}
%
specifies that there are 2 regions and a total of 3 atom species (As ion plus Si
and O atoms in the target, see below). Since \texttt{ndim} is not specified,
its default value \texttt{ndim=1} is taken, and a 1-D geometry is assumed (see
also the \texttt{\&geometry} record). The next line
%
\begin{verbatim}
 &ions     name='As' energy=10 dose=1e13 tilt=7 rotate=15 diverg=0.3 
           nion=10000 /
\end{verbatim}
%
defines the properties of the ions: The ion species is As (notice the quotes for
character input), the energy 10~keV, the dose $10^{13}~$cm$^{-2}$, the tilt and
rotation angles 7$^\circ$ and $15^\circ$, respectively, and the beam divergence
$0.3^\circ$. Since the divergence model is not specified, its default is
used, i.e., isotropic within a maximum angle of \texttt{diverg}. 10000
projectiles are requested to be simulated. The next records
%
\begin{verbatim}
 &material region=1 name='SiO2' xtal='no' /
 &material region=2 name='Si' xtal='yes' /
\end{verbatim}
%
specify the materials of the target regions, amorphous SiO$_2$ (region
1) and crystalline Si (region 2). Since the input file contains no
\texttt{\&crystal} record, the default crystal structure (diamond) and
orientation (100) are assumed.

The next record,
%
\begin{verbatim}
 &damage ldam=t lrcoil=t /
\end{verbatim}
%
specifies that damage in the crystalline region shall be taken into account
dynamically for the simulation of subsequent trajectories, that all recoils
shall be followed, and that histograms for interstitials and vacancies shall
be recorded.

Since there are no \texttt{\&snpar} and \texttt{\&separ} records, the default
parameters for the nuclear and electronic stopping are used. The next input
record 
%
\begin{verbatim}
 &geometry posif=-10,0,100000 /
\end{verbatim}
%
specifies the positions of the interfaces between the target layers
(a 1-D geometry is already specified by \texttt{ndim} of the \texttt{\&setup}
record). The first layer (SiO$_2$ as specified in the \texttt{\&material}
record) extends from $z=-10$~\AA\ to $z=0$ and the second (Si) from $z=0$ to
$z=100000$~\AA\ ($=10~\mu$m). Finally,
%
\begin{verbatim}
 &output lmom=t lhis=t /
\end{verbatim}
%
specifies the amount of output: Dopant and damage moments into the {\tt OUT}
file and 1-D histograms of the dopants and the damage into the \texttt{HIS}
file.

If the script delivered with the program is used, the input file could be named
\texttt{as10.inp}. IMSIL would then produce the files \texttt{as10.out} and
\texttt{as10.his}.

