This record is used to describe the materials of the regions. Special properties
of the crystalline region are specified in the \texttt{\&XTAL} record (see
Section~\ref{s:xtal}). The geometry of the target regions is specified in the
\texttt{\&GEOMETRY} record (see Section~\ref{s:geom}). 

The \texttt{\&MATERIAL}
record has the index variables \texttt{REGION}, \texttt{ATOM1}, and
\texttt{ATOM2}, of which \texttt{REGION} is required on each \texttt{\&MATERIAL}
record.  At least one \texttt{\&MATERIAL} record has to appear for each region.
\texttt{ATOM1} is the atom index of a projectile, \texttt{ATOM2} the atom
index of a target atom (recoil). Therefore, it is possible to define the
properties of an atom as a projectile and as a recoil separately. Regarding the
definition of atom indices, see Section~\ref{s:atom}. 


\begin{keydescription}{\texttt{DENSITY} --- Atom density}
%
  This parameter specifies the number of target atoms (irrespective of
  their species) per cm$^3$. E.g., \texttt{DENSITY=7.E22} for
  \texttt{NAME='SIO2'} means that there are $2.33 \cdot
  10^{22}~\mathrm{cm}^{-3}$ silicon atoms and $4.67 \cdot
  10^{22}~\mathrm{cm}^{-3}$ oxygen atoms. 
  %\texttt{'PMMA'} is interpreted as \texttt{'C5H8O2'}.
  \begin{keytab}
    Type:    \> index variable array (\texttt{REGION}) of real \\
    Default: \> depending on \texttt{NAME}, weighted densities of the 
                elementary targets \\
            \> as specified on the \texttt{\&ATOMS} record \\
%             \> except: \\
%             \> \begin{tabular}{ll}
%                   \texttt{'SIO2'} (or \texttt{'SiO2'}, \texttt{'sio2'}):    
%                      & $7 \cdot 10^{22}~\mathrm{cm^{-3}}$ \\
%                   \texttt{'SI3N4'} (or \texttt{'Si3N4'}, \texttt{'si3n4'}):   
%                      & $9 \cdot 10^{22}~\mathrm{cm^{-3}}$ \\
%                   \texttt{'GAAS'} (or \texttt{'GaAs'}, \texttt{'gaas'}):    
%                      & $4.4 \cdot 10^{22}~\mathrm{cm^{-3}}$ \\
%                   \texttt{'C5H8O2'} (or \texttt{'PMMA'}, \texttt{'pmma'}):  
%                      & $8.52 \cdot 10^{22}~\mathrm{cm^{-3}}$ \\
%                \end{tabular} \\
    Range:   \> $> 0$
  \end{keytab}
\end{keydescription}

\begin{keydescription}{\texttt{NAME} --- Material name}
%
  This parameter specifies the chemical name of the target material.
  E.g. \texttt{NAME='SI'} for silicon, \texttt{NAME='SIO2'} for
  SiO$_2$ etc. The number of atom species, the atom names, their
  atomic number, and the probability to encounter each species in the
  target are extracted automatically.  Moreover, default values of the
  atom masses \texttt{MASS} and material densities \texttt{DENSITY} are
  provided by the program, if \texttt{NAME} is defined.
%
  \begin{keytab}
    Type:    \> index variable array (\texttt{REGION}) of character strings \\
    Default: \> ~ -- ~~~ (obligatory) \\
    Range:   \> any chemical name of a material ($\le$ 80 characters)
  \end{keytab}
\end{keydescription}

\begin{keydescription}{\texttt{LPOLY} --- Polycrystalline flag}
%
  This parameter specifies whether the orientation of the crystal system
  shall be rotated randomly before each ion. \texttt{LPOLY=T} requires 
  \texttt{XTAL='yes'} and causes \texttt{WAFER} and \texttt{VSURF} to
  be ignored.
%
  \begin{keytab}
    Type:    \> index variable array (\texttt{REGION}) of logical \\
    Default: \> \texttt{F} \\
    Range:   \> \texttt{T}, \texttt{F}
  \end{keytab}
\end{keydescription}

\begin{keydescription}{\texttt{ROTX} --- Rotation angle of the $x$ axis}
%
  This parameter specifies the rotation angle of the $x$ axis with
  respect to the \texttt{VSURF} vector. The $x$ axis will be along the
  surface. The $x$ axis together with the surface normal defines the
  plane to which the results are projected in 2-D simulations. For a
  line implantation, ions are implanted between \texttt{XINIT(1)} and
  \texttt{XINIT(2)} (see \texttt{\&IONS} record) which are measured
  along the $x$ axis. This means that the mask edge orientation of the
  corresponding impenetrable mask is perpendicular to the $x$ axis.
  Notice that the rotation angle \texttt{ROTATE} (see \texttt{\&IONS}
  record) for the ion incidence direction is not measured with respect
  to the $x$ axis, but with respect to the surface vector
  \texttt{VSURF}.  The units of \texttt{ROTX} are degrees.
  \texttt{ROTX} must only be specified for a crystalline region.
%
  \begin{keytab}
    Type:    \> index variable array (\texttt{REGION}) of real \\
    Default: \> 0 \\
    Range:   \> arbitrary
  \end{keytab}
\end{keydescription}

\begin{keydescription}{\texttt{SYIELD} --- Sputtering yield}
%
  This parameter specifies the sputtering yield, i.e.\ the ratio of
  sputtered particles to implanted ions.  The sputtering yield is used
  to calculate the movement of the surface depending on the implanted
  dose. \texttt{SYIELD} is only allowed for static 1-D simulations when
  regions are defined (\texttt{LDYN=F}, \texttt{NDIM=1}, \texttt{USECELL=F}).
%
  \begin{keytab}
    Type:    \> index variable array (\texttt{REGION},\texttt{ATOM1}) 
                of real \\
    Default: \> 0 \\
    Range:   \> $\ge 0$
  \end{keytab}
\end{keydescription}

\begin{keydescription}{\texttt{TDEBYE} --- Debye temperature}
%
  This parameter specifies the Debye temperature of the material. It is used
  for calculation of vibrational amplitude using Debye function.  
  \texttt{TDEBYE=0} means a thermal vibration amplitude of 0.083 will be used. 
  Note that some internally tabulated values (e.g. for noble gases) are
  inaccurate. The units of \texttt{TDEBYE} are K.
%
  \begin{keytab}
    Type:    \> index variable array (\texttt{REGION},\texttt{ATOM2}) 
                of real \\
    Default: \> \texttt{TDEBYE} of the \texttt{\&ATOMS} record \\
    Range:   \> $> 0$
  \end{keytab}
\end{keydescription}

\begin{keydescription}{\texttt{VIB} --- Thermal vibration amplitude}
%
  This parameter specifies the mean square amplitude in \AA ~of
  vibrating lattice atoms in a given direction. The displacements of
  the atoms from their lattice sites is Gaussian distributed.
%
  \begin{keytab}
    Type:    \> index variable array (\texttt{REGION},\texttt{ATOM2}) 
                of real \\
    Default: \> Calculated from \texttt{TDEBYE} \\
    Range:   \> $\ge 0$
  \end{keytab}
\end{keydescription}

\begin{keydescription}{\texttt{VSURF} --- Surface vector}
%
  This parameter specifies a reference vector in the wafer surface.  If
  \texttt{ROTX=0}, \texttt{VSURF} specifies the direction of the $x$ axis of
  the target coordinate system in the crystal coordinate system (the $x$ and
  $y$ axes of the target coordinate system are parallel to the wafer surface,
  the $z$ axis is perpendicular to the wafer surface). Otherwise, the $x$ axis
  will be rotated by \texttt{ROTX} with respect to \texttt{VSURF}.
  \texttt{VSURF} is the reference direction for the rotation angle of
  the ion incidence direction (see \texttt{ROTATE} in the
  \texttt{\&IONS} record). The components of \texttt{VSURF} have to be
  given in the crystal coordinate system, i.~e.  the coordinate
  system connected with the [100], [010], and [001] axis of the
  crystal.  \texttt{VSURF} is ignored for non-crystalline regions.
%
  \begin{keytab}
    Type:    \> index variable array (\texttt{REGION}) of simple array
                (3) of real \\
    Default: \> 1.,1.,0. \\
    Range: \> perpendicular to \texttt{WAFER}
  \end{keytab}
\end{keydescription}

\begin{keydescription}{\texttt{WAFER} --- Wafer surface normal vector}
%
  This parameter specifies a vector normal to the wafer surface. The
  components of \texttt{WAFER} have to be given in the crystal
  coordinate system, i.~e.  the coordinate system connected with the
  [100], [010], and [001] axis of the crystal. E.~g. for a (110)
  wafer, a surface normal may be (0,1,1), thus \texttt{WAFER=0,1,1}.
  \texttt{WAFER}  is ignored for non-crystalline regions.
%
  \begin{keytab}
    Type:    \> index variable array (\texttt{REGION}) of simple array
                (3) of real \\
    Default: \> 0.,0.,1. \\
    Range:   \> perpendicular to \texttt{VSURF}
  \end{keytab}
\end{keydescription}

\begin{keydescription}{\texttt{XTAL} --- Crystalline region flag}
%
  This parameter specifies for each region \texttt{IR} whether it is amorphous
  (\texttt{XTAL='no'} or \texttt{XTAL='off'}) or crystalline
  (\texttt{XTAL='yes'} or any other value).  Notice that, due to present
  limitations of the program, only one region may be crystalline. See
  \texttt{XTAL} record for further parameters defining the crystal
  structure and controlling simulation in crystalline regions. A crystalline
  region must not be specified in a dynamic simulation (\texttt{LDYN=T}).
%
  \begin{keytab}
    Type:    \> index variable array (\texttt{REGION}) of character strings \\
    Default: \> \texttt{'no'} \\
    Range:   \> any character string ($\le$ 80 characters)
  \end{keytab}
\end{keydescription}








