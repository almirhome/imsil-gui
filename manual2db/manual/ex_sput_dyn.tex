This example demonstrates a dynamic sputtering simulation of Si by Kr ions. In
this simulation, the target is modified as the ions bombard the target. To store
the state of the target at intermediate times, simulations are run repeatedly,
each simulation reading in the output of the previous one. When starting the
simulations with \texttt{mimsil.py}, the \texttt{-p} option must specify the
number of parallel processes to equal \texttt{1} (which is the default) in
order that the simulations run in sequence. Note that the order of the
simulations as specified in the parameters file is followed for \texttt{-p 1}.

The template file reads:

\begin{verbatim}
2 keV Kr in Si, tilt=60
 &setup    ndim=1 natom=2 nr=1 ldyn=t lsput=t
           usecell=#usecell# cellfile='#cellfile#' /
 &ions     name='Kr' energy=2 dose=1e15 tilt=60 diverg=0 nion=#nion# /
 &material region=1 name='Si' xtal='no' /
 &snpar    ef=4.7 denffp=4.7 /
 &damage   lrcoil=t ldam=f ed=15. frec=1 /
 &damage   atom1=1 esurf=0.1 /
 &separ    xnl=0. corlin=1. facscr=1. /
 &geometry posif=0,200 wrough=2 nionupd=100 posfix=190 fracupd=0. /
 &output   lhis=t lmomne=t lmomb=t lmomt=f lcell=t wbox=1 nbox=205 /
\end{verbatim}

Here we demand dynamic simulation by setting \texttt{ldyn=t} on the
\texttt{\&setup} record. As in the static sputtering simulation,
\texttt{lsput=t} is set to invoke accurate models for sputtering.

In order to read the results of the previous simulation, the \texttt{usecell}
and \texttt{cellfile} parameters are used on the \texttt{\&setup} record. The
value of \texttt{usecell} must be a variable, since in the first simulation
there is no cell file to be read, so \texttt{usecell=f} has to be used, while
\texttt{usecell=t} must be specified for the subsequent simulations. The value
of \texttt{cellfile} must be a variable, too, since the name of the cell file to
be read changes from simulation to simulation.

In order that the cell file is written, \texttt{lcell=t} is specified on the
\texttt{\&output} record.

There are two \texttt{\&damage} records. The first applies to all atoms. It is
important to specify \texttt{lrcoil=t} here in order to allow a sputtering
simulation. The second \texttt{\&damage} record applies only to atom 1, i.e.,
the first (and only) ion atom, Ar. Its surface binding energy is set to
0.1~eV. Such a low value leads to slow simulations, since all recoils down to
0.1~eV must be simulated; very short free flight paths and large impact
parameters must therefore be used. The motivation of this setting is that Ar
atoms close to the surface may leave the target easily.

On the \texttt{\&geometry} record, \texttt{wrough=2} specifies a 2~\AA\ thick
roughness layer. \texttt{nionupd=100} specifies that after every 100 simulated
ions relaxation of the target densities is attempted (an actual update is made
only if the relative change in any cell density exceeds \texttt{FRACUPD}).

\begin{verbatim}
def get_params(param):
    fname = 'krdyn_1e15'
    param[fname] = {}
    param[fname]['usecell'] = 'f'
    param[fname]['cellfile'] = ''
    param[fname]['nion'] = '1000'
    
    for dose in range(2,10):
        fname = 'krdyn_' + str(dose) + 'e15'
        param[fname] = {}
        param[fname]['usecell'] = 't'
        param[fname]['cellfile'] = 'krdyn_' + str(dose-1) + 'e15.cell'
        param[fname]['nion'] = '1000'
    
    fname = 'krdyn_10e15'
    param[fname] = {}
    param[fname]['usecell'] = 't'
    param[fname]['cellfile'] = 'krdyn_9e15.cell'
    param[fname]['nion'] = '10000'
\end{verbatim}

The parameters file is written as a Python function in order to allow a Python
loop to implement the loop over the simulations. The first simulation must be
kept separate as there is no cell file to be read. The last simulation is also
kept separate in order to obtain better statistics for the final results
(assuming sputter equilibirium has been reached). Note that \texttt{fname}
defines the file names of the input and output files (without extension). It is
made different for each simulation by the inclusion of the cumulated ion dose.
