There are two scripts for running IMSIL: \texttt{imsil.py} for running a
single simulation from one input file, and \texttt{mimsil.py} for running
multiple simulations whose input files are generated automatically from a
template and a parameters file. It is also possible to pass several input files
to \texttt{imsil.py} or several pairs of template and parameters files to
\texttt{mimsil.py}.

\texttt{imsil.py} takes a file with extension \texttt{.inp} (the input file),
copies it to a directory with the same name but without the extension,
renames it to \texttt{INP}, runs it there and copies any output files back to
the directory where the original version of the input file resides. The output
files are again renamed so their names are converted to lowercase and become
their file extension. E.g., if \texttt{imsil.py} is run with the input file
\texttt{example.inp}, a directory \texttt{example} is created, the input file is
copied to \texttt{example/INP}, IMSIL is run in the directory \texttt{example},
generates \texttt{example/OUT}, and, e.g., \texttt{example/HIS}. These files are
then copied to \texttt{example.out} and \texttt{example.his}, and the directory
\texttt{example} is deleted.

A typical call of IMSIL from the directory where the input file
\texttt{example.inp} resides, reads
%
\begin{verbatim}
   <path-to-imsil.py> -i example.inp
\end{verbatim}
%
This uses the shebang (first) line of the \texttt{imsil.py} script to call the
Python 3 interpreter (otherwise you have to call the Python interpreter with
the above line as arguments). The extension \texttt{.inp} in the specification
of the input file may be omitted. \texttt{-i} is one of several possible flags
and causes IMSIL to run interactively in the foreground. (The default is to run
IMSIL in the background, so you can continue to use your command window.)

You can also start several simulations at once:
%
\begin{verbatim}
   <path-to-imsil.py> -p 2 example1.inp example2.inp
\end{verbatim}
%
\texttt{-p 2} causes the two simulations to run in parallel. Without the
\texttt{-p} option they will run one after the other. More generally,
\texttt{-p} defines the maximum number of simulations that run in parallel. If
there are more input files than specified by the \texttt{-p} option, the
remaining simulations have to wait until a previous simulation terminates.

If you have a passwordless ssh connection to a remote host, you can also
delegate the simulation to the remote host:
%
\begin{verbatim}
   <path-to-imsil.py> -r <remote-hostname> example.inp
\end{verbatim}
%
Replace \texttt{<remote-hostname>} with the IP address or hostname of the
remote host. For this to work, the remote host must be reachable from the local
host with \texttt{ssh <remote-hostname>}, and from the remote host the local
host must be reachable with \texttt{ssh <local-hostname>}. The ssh connection
must be passwordless in both directions. In addition, the shebang line of
\texttt{imsil.py} must be valid on the remote host, i.e., the Python 3
interpreter must be called on the remote host with the same command as on the
local host. During the operation, the IMSIL executable is copied to the remote
host. IMSIL therefore need not be installed on the remote host, but the
executable must be compatible with the remote host's processor.

A file with the same name as the input file but extension \texttt{.post} can
be provided. It must contain valid Python code. If such a file exists,
\texttt{imsil.py} runs the code after the IMSIL simulation has completed.
It can be used to extract parameters from the output data such as the sputter
yield. 

For a full list of options to the \texttt{imsil.py} script see its docstring.

For \texttt{mimsil.py}, a template file with extension \texttt{.tem} and a
parameters file with extension \texttt{.par} must be provided. The template file
is an input file where the parameters to be varied between the simulations are 
replaced by variables. Variables are defined by hash signs prepended and
appended to their names. Usually, the names are taken identical to the parameter 
names. E.g., to vary the tilt angle, one writes \texttt{tilt=\#tilt\#} on the
\texttt{\&ions} record. The tilt angle may then be referenced by \texttt{tilt}
in the parameters file.

The parameters file contains the definition of the parameters and of the input
file name for each simulation. There are two formats of the parameters
file. In the first format the parameters are defined as a list under each file
name:
%
\begin{verbatim}
   FILE: <filename1>
   <parname11>=<value11>
   <parname12>=<value12>
   ...
   FILE: <filename2>
   <parname21>=<value21>
   <parname22>=<value22>
   ...
   ...
\end{verbatim}
%
Here, \texttt{filename<i>} denotes the filename of the input and output files
without extension, \texttt{parname<ij>} the variable names as defined in the
template file, and \texttt{value<ij>} the values to be substituted for the
variables. As an example, let's assume that two simulations are to be
performed, one for a tilt angle of 0\textdegree\ and one for a tilt angle of
10\textdegree. The template and parameters file are \texttt{example.tem} and
\texttt{example.par}, respectively. The parameters file may read
%
\begin{verbatim}
    FILE: example_0
    tilt=0
    FILE: example_10
    tilt=10
\end{verbatim}
%
\texttt{mimsil.py} produces the input files \texttt{example\_0.inp} and
\texttt{example\_10.inp}, where the variable \texttt{\#tilt\#} of the template
file is replaced by \texttt{0} and \texttt{10}, respectively, and runs the two
simulations.

The alternative format of the parameters file defines a Python function
\texttt{get\_params} whose single argument is a dictionary which defines the
filenames and parameters. In the above example, the parameters file could
alternatively read 
%
\begin{verbatim}
    def get_params(param):
        for tilt in (0, 10):
            fname = 'example_' + str(tilt)
            param[fname] = {}
            param[fname]['tilt'] = str(tilt)
\end{verbatim}
%
On entry, \texttt{param} is an empty ordered dictionary. Within the function,
the elements of the dictionary must be defined as dictionaries with the keys equal
to the parameter names and values equal to the parameter values, converted to
strings. The outer dictionary has the file names as keys. Note that the
\texttt{def} statement must be the first in the file, otherwise the format will
not be correctly identified by \texttt{mimsil.py}. Therefore, e.g., imports such
as \texttt{import math} must be made inside the function. This second format of
the parameters file is useful when the number of simulations to be defined is
great.

In both cases, \texttt{mimsil.py} is run with \texttt{example.tem} and
\texttt{example.par} as template and parameters file by
%
\begin{verbatim}
    <path-to-mimsil.py> example.tem
\end{verbatim}
%
Similar as before, \texttt{<path-to-mimsil.py>} denotes the path to the
\texttt{mimsil.py} script. The extension \texttt{.tem} may be omitted or
replaced by \texttt{.par}.

The \texttt{mimsil.py} script imports \texttt{imsil.py} as a module and passes
its options to the main function of \texttt{imsil.py}. The \texttt{-p} and
\texttt{-r} options may therefore be used just as with \texttt{imsil.py}.

A template file with extension \texttt{.ptem} can also be provided for
postprocessing. The \texttt{.ptem} file contains Python code to be run after
each simulation. The same variables as in the \texttt{.tem} file (in the
above example \texttt{\#tilt\#}) can be defined and will be replaced by the
values defined in the \texttt{.par} file. The resulting code will be
saved in a \texttt{.post} file for each input (\texttt{.inp}) file.
