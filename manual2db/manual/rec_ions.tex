This record is used to describe the ions. It has to appear in each input file. 
It has the index variable \texttt{ATOM1} corresponding to the index of the atom
in the ion. The index is increasing from left to right in the ion name as
specified by the \texttt{NAME} parameter. E.g., for \texttt{NAME='BF2'},
\texttt{ATOM1=1} corresponds to B and \texttt{ATOM1=2} to F. 

Another index variable is \texttt{BEAM}. It allows to specify the beam as a
superposition of partial beams. Ions are chosen from each of the partial beams
with a probability proportional to \texttt{WEIGHT}, see Section~\ref{s:beam}.

\ifprivate
\begin{keydescription}{\texttt{DINIT} --- Initial square edge length}
%
  This parameter specifies the edge length of the square within the
  plane defined by \texttt{XINIT(1)}, \texttt{YINIT(1)}, \texttt{RINIT} and
  \texttt{ROTATE}, within which the starting points of the ions are
  selected if \texttt{LPINIT=T}.
%
  \begin{keytab}
    Type:    \> real \\
    Default: \> 5.43 \\
    Range:   \> arbitrary
  \end{keytab}
\end{keydescription}
\fi

\begin{keydescription}{\texttt{DIVERG} --- Beam divergence}
%
  This parameter specifies the beam divergence in degrees. For
  \texttt{MODDIV=1}, the initial ion direction will be distributed
  isotropically within an elliptical cone of apertures \texttt{2*DIVERG(1)}
  and \texttt{2*DIVERG(2)} in $x$ and $y$ direction of the beam coordinate
  system. For \texttt{MODDIV=2}, the initial ion direction will be
  Gaussian distributed with standard deviations \texttt{DIVERG(1)}
  and \texttt{DIVERG(2)} in $x$ and $y$ direction of the beam coordinate
  system. The ion direction will be rotated into the target coordinate
  system according to \texttt{TILT} and \texttt{ROTATE}.
%
  \begin{keytab}
    Type:    \> index variable array (\texttt{BEAM}) of simple arrays (2) of
                real \\
    Default: \> 0, \texttt{DIVERG(1)} \\
    Range:   \> $\ge 0$
  \end{keytab}
\end{keydescription}

\begin{keydescription}{\texttt{DOSE} --- Implantation dose}
%
  This parameter specifies the implantation dose in units defined by 
  \texttt{DOSEUNITS}. If \texttt{DOSEUNITS='cm-2'}, the dose is defined as
  the number of incident ions per area in the $x$-$y$ plane (not perpendicular
  to the beam if there is a tilt angle). If \texttt{DOSEUNITS='cm-1'}, the dose
  is defined as the number of incident ions per length in $y$ direction. For
  \texttt{DOSEUNITS='1'}, \texttt{DOSE} refers to the number of incident ions.
%
  \begin{keytab}
    Type:    \> real \\
    Default: \> corresponding to 1 ion per \AA$^2$,  \AA, or 1 ion
                depending on \texttt{DOSEUNITS} \\
    Range:   \> $> 0$
  \end{keytab}
\end{keydescription}

\begin{keydescription}{\texttt{DOSEUNITS} --- Units of dose}
%
  This parameter specifies the units of the implantation dose specified by
  \texttt{DOSE}. See the description of \texttt{DOSE} for the meaning of
  possible units.
%
  \begin{keytab}
    Type:    \> real \\
    Default: \> \texttt{'cm-2'} \\
    Range:   \> \texttt{'cm-2'}, \texttt{'cm-1'}, \texttt{'1'}
  \end{keytab}
\end{keydescription}

\begin{keydescription}{\texttt{ENERGY} --- Initial ion energy}
%
  This parameter specifies the initial ion energy in keV.
  \begin{keytab}
    Type:    \> real \\
    Default: \> ~ -- ~~~(obligatory) \\
    Range:   \> $> 0$
  \end{keytab}
\end{keydescription}

\begin{keydescription}{\texttt{FWHM} --- Full width at half maximum}
%
  This parameter specifies the full width at half maximum of the blur of the
  ion beam. A Gaussian distribution function with the given FWHM is superposed
  to the initial positions of the ions.  The two values of \texttt{FWHM}
  correspond to the $x$ and $y$ direction in the beam coordinate system. The
  units of \texttt{FWHM} are \AA.
  \begin{keytab}
    Type:    \> simple array (2) of real \\
    Default: \> 0., \texttt{FWHM(1)} \\
    Range:   \> $> 0$
  \end{keytab}
\end{keydescription}

\begin{keydescription}{\texttt{LIINIT} --- Internal starting point flag}
%
  This parameter specifies whether the starting points of the ion
  trajectories shall be chosen at the coordinates given by \texttt{XINIT},
  \texttt{YINIT}, \texttt{ZINIT} ({\tt LIINIT=T}), or by a projection of 
  these points to the surface along the incidence direction of the ion 
  ({\tt LIINIT=F}).  For \texttt{LIINIT=T}, when the starting point is inside
  the target, the first free flight path starts there. When it is outside 
  the target, the point of entrance into the target is searched for in the 
  forward direction only (this makes possible to start trajectories inside a 
  hollow cylinder or sphere). For \texttt{LIINIT=F} the entrance into the 
  target is searched for also in the backward direction.
%
  \begin{keytab}
    Type:    \> logical \\
    Default: \> \texttt{F} \\
    Range:   \> \texttt{T}, \texttt{F}
  \end{keytab}
\end{keydescription}

\begin{keydescription}{\texttt{LLINIT} --- Start as lattice atom flag}
%
  This parameter specifies whether in the case of an internal starting
  point (\texttt{LIINIT=T}) the starting point shall be the position
  of a lattice atom.
%
  \begin{keytab}
    Type:    \> logical \\
    Default: \> \texttt{F} \\
    Range:   \> \texttt{T}, \texttt{F}
  \end{keytab}
\end{keydescription}

\ifprivate
\begin{keydescription}{\texttt{LPINIT} --- Plane implantation flag}
%
  This parameter specifies whether the starting point of the ion has to
  be chosen along a line parallel to \texttt{ROTATE} at the distance
  \texttt{RINIT} from (\texttt{XINIT(1)},\texttt{YINIT(1)}). The initial
  direction is parallel to this plane.
%
  \begin{keytab}
    Type:    \> logical \\
    Default: \> \texttt{F} \\
    Range:   \> \texttt{T} if \texttt{LRINIT=F}, \texttt{F}
  \end{keytab}
\end{keydescription}
\fi

\ifprivate
\begin{keydescription}{\texttt{LRINIT} --- Circle implantation flag}
%
  This parameter specifies whether the starting point of the ion has to
  be chosen along a circle with radius \texttt{RINIT} around
  (\texttt{XINIT(1)},\texttt{YINIT(1)}).
%
  \begin{keytab}
    Type:    \> logical \\
    Default: \> \texttt{F} \\
    Range:   \> \texttt{T} if \texttt{LPINIT=F}, \texttt{F}
  \end{keytab}
\end{keydescription}
\fi

\begin{keydescription}{\texttt{LTINIT} --- Start as target atom flag}
%
  This parameter specifies whether the ion shall be considered
  indistinguishable from a target atom if they have the same atomic
  number.  If \texttt{LTINIT=t} and the ion has the same atomic number
  as a target atom, replacement collisions are allowed and the ion is
  counted as damage.
%
  \begin{keytab}
    Type:    \> logical \\
    Default: \> \texttt{F} \\
    Range:   \> \texttt{T}, \texttt{F}
  \end{keytab}
\end{keydescription}

\begin{keydescription}{\texttt{MASS} --- Ion mass}
%
  This parameter specifies the mass of the ion atoms in units of the
  atomic mass unit.  It overwrites the default values determined by
  the atom names.
%
  \begin{keytab}
    Type:    \> index variable array (\texttt{ATOM1}) of real \\
    Default: \> determined from \texttt{NAME} \\
    Range: \> $> 0$
  \end{keytab}
\end{keydescription}

\begin{keydescription}{\texttt{MODDIV} --- Beam divergence model}
%
  This parameter specifies the distribution function of the ion's
  incidence direction around the nominal beam direction given by
  \texttt{TILT} and \texttt{ROTATE}. \texttt{MODDIV=1} means that the
  actual incidence direction is isotropically distributed within the
  angle \texttt{DIVERG} from the nominal incidence direction.
  \texttt{MODDIV=2} means a Gaussian weighted isotropic distribution 
  with standard deviation \texttt{DIVERG}. While the isotropic distribution 
  (\texttt{MODDIV=1}) is exact for all values of \texttt{DIVERG}, the Gaussian
  weighted distribution is exact only in the limit of small \texttt{DIVERG}.
  Note that for small \texttt{DIVERG} the root mean square value of the angle
  between actual and nominal incidence angle is \texttt{DIVERG}$/\sqrt{2}$ in 
  case of an isotropical distribution (\texttt{MODDIV=1}).
%
  \begin{keytab}
    Type:    \> index variable array (\texttt{BEAM}) of integer \\
    Default: \> 1 \\
    Range:   \> 1, 2
  \end{keytab}
\end{keydescription}

\begin{keydescription}{\texttt{NAME} --- Name of ion species}
%
  This parameter specifies the chemical formula of the ion molecule.
  E.g. \texttt{NAME='B'} for B implantation, \texttt{NAME='BF2'} for
  BF$_2$ implantation, etc.  The number of atom species (2 in the case
  of BF$_2$), the atom names (B and F), the atomic numbers (5 and 9,
  respectively), default values of the atom masses (11.009 and 18.998,
  respectively), and the number of atoms of each atom species in the
  molecule (1 and 2, respectively) are extracted automatically.
%
  \begin{keytab}
    Type:    \> character string \\
    Default: \> ~ -- ~~~ (obligatory) \\
    Range:   \> any chemical name of a molecule ($\le$ 80 characters)
  \end{keytab}
\end{keydescription}

\begin{keydescription}{\texttt{NION} --- Number of simulated ions}
%
  This parameter specifies the number of ions to be simulated. The
  simulation time will be roughly proportional to \texttt{NION} while
  statistical errors decrease with the square root of \texttt{NION}.
  In case of the continuation of a previous run (\texttt{USEBK=T}),
  \texttt{NION} specifies the total number of ions including those
  simulated in the previous run.
%
  \begin{keytab}
    Type:    \> integer \\
    Default: \> ~ -- ~~~(obligatory) \\
    Range:   \> $> 0$
  \end{keytab}
\end{keydescription}

\begin{keydescription}{\texttt{RANROT} --- Random rotation angle flag}
%
  This parameter is a flag indicating that the rotation angle has to be
  selected randomly, simulating a continuously varying rotation angle.
  If \texttt{RANROT=T}, \texttt{ROTATE} must not be specified at the
  same time.
%
  \begin{keytab}
    Type:    \> logical \\
    Default: \> \texttt{F} \\
    Range:   \> \texttt{T}, \texttt{F}
  \end{keytab}
\end{keydescription}

\begin{keydescription}{\texttt{REENTER} --- Reenter flag}
%
  This parameter specifies whether ions are allowed to traverse 
  vacuum regions and to reenter the target somewhere else. If
  \texttt{REENTER=F}, the ion trajectories are terminated as soon
  as they leave the target, and the ions are counted as backscattered or
  transmitted depending on their direction of motion. \texttt{REENTER=T} 
  must not be specified in a 1-D layer simulation (\texttt{NDIM=1},
  \texttt{LCYLX=LCYLY=LCYLZ=LSPHER=F}).
%
  \begin{keytab}
    Type:    \> logical \\
    Default: \>
    \texttt{NDIM>1.OR.LCYLX.OR.LCYLY.OR.LCYLZ.OR.LSPHER} \\
    Range:   \> \texttt{T}, \texttt{F}
  \end{keytab}
\end{keydescription}

\ifprivate
\begin{keydescription}{\texttt{RINIT} --- Radius/distance for starting point}
%
  This parameter specifies the radius of the circle around
  (\texttt{XINIT(1)},\texttt{YINIT(1)}) along which the starting point is
  chosen if \texttt{LRINIT=T}. It specifies the distance from
  (\texttt{XINIT(1)},\texttt{YINIT(1)}) of the line along which the starting
  points are chosen if \texttt{LPINIT=T}. The units of \texttt{RINIT}
  are \AA.
%
  \begin{keytab}
    Type:    \> real \\
    Default: \> 0 \\
    Range:   \> arbitrary
  \end{keytab}
\end{keydescription}
\fi

\begin{keydescription}{\texttt{ROTATE} --- Rotation angle}
%
  This parameter specifies the rotation angle in degrees, i.e. the
  azimuthal orientation of the initial ion direction with respect to
  the the surface vector \texttt{VSURF} (see \texttt{\&MATERIAL} record).
\begin{keytab}
   Type:    \> real \\
   Default: \> 0 \\
   Range:   \> arbitrary
\end{keytab}
\end{keydescription}

\begin{keydescription}{\texttt{TILT} --- Tilt angle}
%
  This parameter specifies the tilt angle in degrees, i.e. the angle
  between the initial ion direction and the wafer normal (defined by
  \texttt{WAFER} in the \texttt{\&MATERIAL} record).
%
  \begin{keytab}
    Type:    \> real \\
    Default: \> 0 \\
    Range:   \> $0 \le \texttt{TILT} < 90$
  \end{keytab}
\end{keydescription}

\begin{keydescription}{\texttt{TRAFILE} --- Trajectory file}
%
  This parameter specifies the name of a trajectory file used to set the
  initial conditions of the ions. The format of the trajectory file
  is described in file in Chapter~\ref{k:files}). Lines with
  \texttt{IFLAG=2} are ignored. Lines with \texttt{IFLAG=1} are used to
  mark lattice atoms that are initially excluded from being a collision
  partner. The parameters \texttt{ENERGY} and \texttt{NION} of the
  \texttt{\&ION} record still must be specified, but it is sufficient to use
  upper bounds to the actual values. Normally, one may want to set
  \texttt{LIINIT=T}. Relevant only if \texttt{USETRA=T}.
  \begin{keytab}
    Type:    \> character string \\
    Default: \> ~ -- ~~~ (obligatory if \texttt{USETRA=T}) \\
    Range:   \> Any name ($\le 80$ characters) of an IMSIL trajectory file
  \end{keytab}
\end{keydescription}

\begin{keydescription}{\texttt{USETRA} --- Use-trajectory-file flag}
%
  This variable specifies whether the ions are initialized with data from a
  trajectory file. If \texttt{USETRA=T}, the trajectory file specified by
  \texttt{TRAFILE} must exist.
%
  \begin{keytab}
    Type:    \> logical \\
    Default: \> \texttt{F} \\
    Range:   \> \texttt{T}, \texttt{F}
  \end{keytab}
\end{keydescription}

\begin{keydescription}{\texttt{WEIGHT} --- Beam weight}
%
  This parameter specifies the weight of each partial beam. Ions will be chosen
  from the partial beams with probability proportional to \texttt{WEIGHT}. The
  weights will internally be normalized to unity, i.e., as specified they need
  not sum up to unity.
%
  \begin{keytab}
    Type:    \> index variable array (\texttt{BEAM}) of real \\
    Default: \> 1 \\
    Range:   \> $\ge 0$
  \end{keytab}
\end{keydescription}

\begin{keydescription}{\texttt{XINIT} --- Initial x value}
%
  This parameter specifies the $x$ coordinate in \AA\ of the starting point of
  the simulated ion trajectories (\texttt{LIINIT=T}) or of the focal point of the
  beam (\texttt{LIINIT=F}). If \texttt{LIINIT=F}, the starting point is obtained
  by projection of the focal point to slightly outside the surface. The starting
  point or focal point is slightly modified in a crystalline target. If
  two values are given for \texttt{XINIT}, the starting or focal point is
  generated randomly with uniform distribution between the two values.
  \ifprivate 
    \texttt{XINIT(1)} specifies a reference point for determining the
    starting point in case of \texttt{LRINIT=T} or \texttt{LPINIT=T}.
    See these keys for further information.
  \fi
%
  \begin{keytab}
    Type:    \> simple array (2) of real \\
    Default: \> 0, \texttt{XINIT(1)} \\
    Range:   \> \texttt{XINIT(2)} $\ge$ \texttt{XINIT(1)}
  \end{keytab}
\end{keydescription}

\begin{keydescription}{\texttt{YINIT} --- Initial y value}
%
  This parameter specifies the $y$ coordinate in \AA\ of the starting point of
  the simulated ion trajectories (\texttt{LIINIT=T}) or of the focal point of the
  beam (\texttt{LIINIT=F}). If \texttt{LIINIT=F}, the starting point is obtained
  by projection of the focal point to slightly outside the surface. The starting
  point or focal point is slightly modified in a crystalline target. If
  two values are given for \texttt{YINIT}, the starting or focal point is
  generated randomly with uniform distribution between the two values.
  \ifprivate 
    \texttt{YINIT(1)} specifies a reference point for determining the
    starting point in case of \texttt{LRINIT=T} or \texttt{LPINIT=T}.
    See these keys for further information.
  \fi
%
  \begin{keytab}
    Type:    \> simple array (2) of real \\
    Default: \> 0, \texttt{YINIT(1)} \\
    Range:   \> \texttt{YINIT(2)} $\ge$ \texttt{YINIT(1)}
  \end{keytab}
\end{keydescription}

\begin{keydescription}{\texttt{ZINIT} --- Initial z value}
%
  This parameter specifies the $z$ coordinate in \AA\ of the starting point of
  the simulated ion trajectories (\texttt{LIINIT=T}) or of the focal point of the
  beam (\texttt{LIINIT=F}). If \texttt{LIINIT=F}, the starting point is obtained
  by projection of the focal point to slightly outside the surface. The starting
  point or focal point is slightly modified in a crystalline target. If
  two values are given for \texttt{ZINIT}, the starting or focal point is
  generated randomly with uniform distribution between the two values.
  \ifprivate 
    \texttt{ZINIT(1)} specifies a reference point for determining the
    starting point in case of \texttt{LRINIT=T} or \texttt{LPINIT=T}.
    See these keys for further information.
  \fi
%
  \begin{keytab}
    Type:    \> simple array (2) of real \\
    Default: \> 0, \texttt{ZINIT(1)} \\
    Range:   \> \texttt{ZINIT(2)} $\ge$ \texttt{ZINIT(1)}
  \end{keytab}
\end{keydescription}

