This record is used to specify the amount of output to be written to output files.

\begin{keydescription}{\texttt{AMAXA} --- Analyzer maximum acceptance angle}
%
  This parameter specifies the upper limit to the acceptance angle for
  backscattered or transmitted projectiles.  Ions with directions farther away 
  from the analyzer direction than \texttt{AMAXA} are not included in the
  statistics and they are not written to the trajectory file. For the definition
  of the analyzer direction see \texttt{TILTA} and \texttt{ROTATA}. The units of
  \texttt{AMAXA} are degrees. 
  \begin{keytab}
    Type:    \> real \\
    Default: \> 180  \\
    Range:   \> $\texttt{AMINA} \le \texttt{AMAXA} \le 180$
  \end{keytab}
\end{keydescription}

\begin{keydescription}{\texttt{AMINA} --- Analyzer minimum acceptance angle}
%
  This parameter specifies the lower limit to the acceptance angle for
  backscattered or transmitted projectiles.  Ions with directions closer to 
  the analyzer direction than \texttt{AMINA} are not included in the
  statistics and they are not written to the trajectory file. For the definition
  of the analyzer direction see \texttt{TILTA} and \texttt{ROTATA}. The units of
  \texttt{AMINA} are degrees. 
  \begin{keytab}
    Type:    \> real \\
    Default: \> 0  \\
    Range:   \> $0 \le \texttt{AMINA} \le \texttt{AMAXA}$
  \end{keytab}
\end{keydescription}

\begin{keydescription}{\texttt{DPOWER} --- Displacement power in displacement histograms}
%
  This parameter specifies the box width variation of displacement histograms (\texttt{HISD} 
  file): Internally, $d^\texttt{1/DPOWER}$ is scored in an equidistant histogram, but the box 
  boundaries are taken to the \texttt{DPOWER}-th power when witten to the \texttt{HISD} file, 
  thus the displacements $d$ appear as non-equidistant box boundaries in the \texttt{HISD} 
  file. For \texttt{DPOWER=1}, the histogram is equidistant; for \texttt{DPOWER>1}, box widths 
  are smaller at small energies and larger at large energies. Each histogram box will receive 
  the same number of particles if the diplacement distribution is proportional to 
  $d^\texttt{-DPOWER+1}$. Note that the displacement distribution must decay faster than 
  $d^\texttt{-2}$ if it is given by a power law. Therefore, for the default value of 
  \texttt{DPOWER=3} we still expect increasing statistical error with increasing displacement 
  $d$.
  \begin{keytab}
    Type:    \> integer \\
    Default: \> 3  \\
    Range:   \> $>0$
  \end{keytab}
\end{keydescription}

\begin{keydescription}{\texttt{EMAXA} --- Analyzer maximum acceptance energy}
%
  This parameter specifies the upper limit to the acceptance energy, Ions with energies larger 
  than \texttt{EMAXA} are not included in the RBS statistics or in the statistics of 
  transmitted or backscattered ions. The units of \texttt{EMAXA} are keV.
  \begin{keytab}
    Type:    \> real \\
    Default: \> \texttt{E0}  \\
    Range:   \> $\texttt{EMINA} \le \texttt{EMAXA}$
  \end{keytab}
\end{keydescription}

\begin{keydescription}{\texttt{EMINA} --- Analyzer minimum acceptance energy}
%
  This parameter specifies the lower limit to the acceptance energy, i.e., ions with energies 
  lower than \texttt{EMINA} are not included in the RBS statistics or in the statistics of 
  transmitted or backscattered ions. The units of \texttt{EMINA} are keV.
  \begin{keytab}
    Type:    \> real \\
    Default: \> 0  \\
    Range:   \> $0 \le \texttt{EMINA} \le \texttt{EMAXA}$
  \end{keytab}
\end{keydescription}

\begin{keydescription}{\texttt{EPOWER} --- Energy power in histogram}
%
  This parameter specifies the box width variation of energy histograms for backward and 
  forward sputtering if \texttt{LHISEB=T} and/or \texttt{LHISET=T}: Internally, 
  $E^\texttt{1/EPOWER}$ is scored in an equidistant histogram, but the box boundaries are 
  taken to the \texttt{EPOWER}-th power when witten to the \texttt{HISEB} and/or \texttt{HISET} 
  file, thus the displacements $E$ appear as non-equidistant box boundaries in the 
  \texttt{HISEB} and/or \texttt{HISET} file. For \texttt{EPOWER=1}, the histogram is 
  equidistant; for \texttt{EPOWER>1}, box widths are smaller at small energies and larger at 
  large energies. Each histogram box will receive the same number of particles if the energy 
  distribution of the sputtered particles is proportional to $E^\texttt{-EPOWER+1}$. The 
  default value of \texttt{EPOWER=3} corresponds to the energy distribution predicted by 
  Sigmund. 
  \begin{keytab}
    Type:    \> integer \\
    Default: \> 3  \\
    Range:   \> $>0$
  \end{keytab}
\end{keydescription}

\ifprivate
\begin{keydescription}{\texttt{I2PAIR} --- Target atom index for pair distibution function}
%
  This parameter the index of the target atom which has to be considered for the pair 
  distribution function on a 2-D grid (\texttt{LHISP2=T}).
  \begin{keytab}
    Type:    \> logical \\
    Default: \> 1 \\
    Range:   \> $1 \le \texttt{I2PAIR} \le \texttt{N2}$
  \end{keytab}
\end{keydescription}
\fi

\begin{keydescription}{\texttt{ITRA} --- Output indicator for trajectory file}
%
  This parameter indicates what output is to be written to the \texttt{TRA} file. Lines with 
  the last value (\texttt{IFLAG}, see the description of the \texttt{TRA} file
  in Chapter~\ref{k:files}) corresponding to a value of \texttt{ITRA} will be written.
  \begin{keytab}
    Type:    \> array of integer (arbitrary size) \\
    Default: \> '-1,0,1,2,3,4,5' if \texttt{LTRA=T}, ' ' otherwise \\
    Range:   \> $-1 \le \texttt{ITRA} \le 5$
  \end{keytab}
\end{keydescription}

\begin{keydescription}{\texttt{LCELL} --- Output of dynamic cell target}
%
  This parameter specifies whether the contents of the dynamic cell target structure shall be 
  written to the \texttt{CELL} or \texttt{CELL2} file.
 \begin{keytab}
    Type:    \> logical \\
    Default: \> \texttt{F} \\
    Range:   \> \texttt{T}, \texttt{F} 
  \end{keytab}
\end{keydescription}

\begin{keydescription}{\texttt{LCYL} --- Output in cylinder coordinates}
%
  This parameter specifies whether the 2-D histograms are distributions in cylinder 
  coordinates. The first coordinate is the radius, the second is depth. The axis of the 
  cylinder is parallel to the $z$ axis and has (x,y) coordinates
  (\texttt{XINIT},\texttt{YINIT}) from the \texttt{\&IONS} record.
  \begin{keytab}
    Type:    \> logical \\
    Default: \> \texttt{F} \\
    Range:   \> \texttt{T}, \texttt{F} 
  \end{keytab}
\end{keydescription}

\ifprivate
\begin{keydescription}{\texttt{LDEF} --- Output of deterministic defects}
%
  This parameter specifies whether the deterministic defect contained in a simulation cell 
  shall be written to the \texttt{DEF} file.
  \begin{keytab}
    Type:    \> logical \\
    Default: \> \texttt{F} \\
    Range:   \> \texttt{T}, \texttt{F} 
  \end{keytab}
\end{keydescription}
\fi

\begin{keydescription}{\texttt{LDIST} --- Output of distance function}
%
  This parameter specifies whether the signed distance from the surface shall be written to the 
  \texttt{DIST2} file. Only effective if \texttt{NDIM=2}.
  \begin{keytab}
    Type:    \> logical \\
    Default: \> \texttt{F} \\
    Range:   \> \texttt{T}, \texttt{F} 
  \end{keytab}
\end{keydescription}

\begin{keydescription}{\texttt{LHIS} --- Output of 1-D spatial atom distribution}
%
  This parameter specifies whether the 1-D histogram of the spatial atom distributions shall be 
  written to the \texttt{HIS} file. \texttt{LHIS=T} is required if
  \texttt{LDAM=T} and \texttt{\&DAMAGE/NDIM=1}.
  \begin{keytab}
    Type:    \> logical \\
    Default: \> \texttt{T} if \texttt{LDAM=T} and \texttt{\&DAMAGE/NDIM=1} \\
             \> \texttt{F} otherwise \\
    Range:   \> \texttt{T}, \texttt{F} 
  \end{keytab}
\end{keydescription}

\begin{keydescription}{\texttt{LHIS2} --- Output of 2-D spatial atom distributions}
%
  This parameter specifies whether the 2-D histogram of the spatial atom distribution shall be 
  written to the \texttt{HIS2} file. \texttt{LHIS2=T} is required if
  \texttt{LDAM=T} and \texttt{\&DAMAGE/NDIM=2}.
  \begin{keytab}
    Type:    \> logical \\
    Default: \> \texttt{T} if \texttt{LDAM=T} and \texttt{\&DAMAGE/NDIM=2} \\
             \> \texttt{F} otherwise \\
    Range:   \> \texttt{T}, \texttt{F} 
  \end{keytab}
\end{keydescription}

\begin{keydescription}{\texttt{LHIS2B} --- Output of 2-D backscattered atom spatial distributions}
%
  This parameter specifies whether the 2-D histogram of the exit points
  and points of origin of backscattered atoms shall be written to
  the \texttt{HIS2B} file.
  \begin{keytab}
    Type:    \> logical \\
    Default: \> \texttt{F} \\
    Range:   \> \texttt{T}, \texttt{F} 
  \end{keytab}
\end{keydescription}

\begin{keydescription}{\texttt{LHIS2T} --- Output of 2-D transmitted atom spatial distributions}
%
  This parameter specifies whether the 2-D histogram of the exit points
  and points of origin of transmitted atoms shall be written to
  the \texttt{HIS2T} file.
  \begin{keytab}
    Type:    \> logical \\
    Default: \> \texttt{F} \\
    Range:   \> \texttt{T}, \texttt{F} 
  \end{keytab}
\end{keydescription}

\begin{keydescription}{\texttt{LHIS3} --- Output of 3-D spatial atom distributions}
%
  This parameter specifies whether the histogram of the 3-D spatial atom
  distribution shall be written to the \texttt{HIS3} file. \texttt{LHIS3=T} is
  required if \texttt{LDAM=T} and \texttt{\&DAMAGE/NDIM=3}.
  \begin{keytab}
    Type:    \> logical \\
    Default: \> \texttt{T} if \texttt{LDAM=T} and \texttt{\&DAMAGE/NDIM=3} \\
             \> \texttt{F} otherwise \\
    Range:   \> \texttt{T}, \texttt{F} 
  \end{keytab}
\end{keydescription}

\begin{keydescription}{\texttt{LHISA2B} --- Output of 2-D backscattered angular distributions}
%
  This parameter specifies whether the histogram for the 2-D angular
  distribution of the backscattered atoms shall be written to the
  \texttt{HISA2B} file. The independent variables of the histogram are the
  direction cosines with respect to the $x$ and $y$ axis.
  \begin{keytab}
    Type:    \> logical \\
    Default: \> \texttt{F} \\
    Range:   \> \texttt{T}, \texttt{F} 
  \end{keytab}
\end{keydescription}

\begin{keydescription}{\texttt{LHISA2T} --- Output of 2-D transmitted angular distributions}
%
  This parameter specifies whether the histogram for the 2-D angular
  distribution of the transmitted atoms shall be written to the
  \texttt{HISA2T} file. The independent variables of the histogram are the
  direction cosines with respect to the $x$ and $y$ axis.
  \begin{keytab}
    Type:    \> logical \\
    Default: \> \texttt{F} \\
    Range:   \> \texttt{T}, \texttt{F} 
  \end{keytab}
\end{keydescription}

\begin{keydescription}{\texttt{LHISAAB} --- Output of backscattered angular distributions (3-D)}
%
  This parameter specifies whether the histogram for the angular distribution of the 
  backscattered atoms shall be written to the \texttt{HISAAB} file. The angle is measured
  with respect to the analyzer direction.
  \begin{keytab}
    Type:    \> logical \\
    Default: \> \texttt{F} \\
    Range:   \> \texttt{T}, \texttt{F} 
  \end{keytab}
\end{keydescription}

\begin{keydescription}{\texttt{LHISAAT} --- Output of transmitted angular distributions (3-D)}
%
  This parameter specifies whether the histogram for the angular distribution of the 
  transmitted atoms shall be written to the \texttt{HISAAT} file. The angle is measured
  with respect to the analyzer direction.
  \begin{keytab}
    Type:    \> logical \\
    Default: \> \texttt{F} \\
    Range:   \> \texttt{T}, \texttt{F} 
  \end{keytab}
\end{keydescription}

\begin{keydescription}{\texttt{LHISAB} --- Output of backscattered angular distributions (2-D)}
%
  This parameter specifies whether the histogram for the angular distribution of the 
  backscattered atoms shall be written to the \texttt{HISAB} file. The angle is measured
  between the projection to the $xz$ plane and the negative $z$ axis.
  \begin{keytab}
    Type:    \> logical \\
    Default: \> \texttt{F} \\
    Range:   \> \texttt{T}, \texttt{F} 
  \end{keytab}
\end{keydescription}

\begin{keydescription}{\texttt{LHISAT} --- Output of transmitted angular distributions (2-D)}
%
  This parameter specifies whether the histogram for the angular distribution of the 
  transmitted atoms shall be written to the \texttt{HISAT} file. The angle is measured
  between the projection to the $xz$ plane and the negative $z$ axis.
  \begin{keytab}
    Type:    \> logical \\
    Default: \> \texttt{F} \\
    Range:   \> \texttt{T}, \texttt{F} 
  \end{keytab}
\end{keydescription}

\begin{keydescription}{\texttt{LHISB} --- Output of backscattered spatial distributions}
%
  This parameter specifies whether the histogram for the spatial distribution of the 
  backscattered atoms shall be written to the \texttt{HISB} file.
  \begin{keytab}
    Type:    \> logical \\
    Default: \> \texttt{F} \\
    Range:   \> \texttt{T}, \texttt{F} 
  \end{keytab}
\end{keydescription}

\begin{keydescription}{\texttt{LHISD} --- Output of displacement distributions}
%
  This parameter specifies whether the histogram of the atom displacement distribution shall be 
  written to the \texttt{HISD} file. The atom displacements are the distances between the 
  starting and end points of the recoils that do not leave the target.
  \begin{keytab}
    Type:    \> logical \\
    Default: \> \texttt{F} \\
    Range:   \> \texttt{T}, \texttt{F} 
  \end{keytab}
\end{keydescription}

\begin{keydescription}{\texttt{LHISEB} --- Output of backscattered energy distributions}
%
  This parameter specifies whether the histogram for the energy distribution of the 
  backscattered atoms shall be written to the \texttt{HISEB} file.
  \begin{keytab}
    Type:    \> logical \\
    Default: \> \texttt{F} \\
    Range:   \> \texttt{T}, \texttt{F} 
  \end{keytab}
\end{keydescription}

\begin{keydescription}{\texttt{LHISEE} --- Output of electronic energy deposition 
    distributions}
%
  This parameter specifies whether the histogram of the 1D electronic energy
  deposition distribution shall be written to the \texttt{HISNE} file.
  \begin{keytab}
    Type:    \> logical \\
    Default: \> \texttt{F} \\
    Range:   \> \texttt{T}, \texttt{F} 
  \end{keytab}
\end{keydescription}

\begin{keydescription}{\texttt{LHISEE2} --- Output of 2-D electronic energy deposition distributions}
%
  This parameter specifies whether the histogram of the 2-D electronic energy
  deposition distribution shall be written to the \texttt{HISEE2} file.
  \begin{keytab}
    Type:    \> logical \\
    Default: \> \texttt{F} \\
    Range:   \> \texttt{T}, \texttt{F} 
  \end{keytab}
\end{keydescription}

\begin{keydescription}{\texttt{LHISEE3} --- Output of 3-D electronic energy deposition distributions}
%
  This parameter specifies whether the histogram of the 3-D electronic energy
  deposition distribution shall be written to the \texttt{HISEE3} file.
  \begin{keytab}
    Type:    \> logical \\
    Default: \> \texttt{F} \\
    Range:   \> \texttt{T}, \texttt{F} 
  \end{keytab}
\end{keydescription}

\begin{keydescription}{\texttt{LHISET} --- Output of transmitted energy distributions}
%
  This parameter specifies whether the histogram for the energy distribution of the 
  transmitted atoms shall be written to the \texttt{HISET} file.
  \begin{keytab}
    Type:    \> logical \\
    Default: \> \texttt{F} \\
    Range:   \> \texttt{T}, \texttt{F} 
  \end{keytab}
\end{keydescription}

\begin{keydescription}{\texttt{LHISNE} --- Output of nuclear energy deposition distributions}
%
  This parameter specifies whether the histogram of the 1D nuclear energy
  deposition distribution shall be written to the \texttt{HISNE} file.
  \begin{keytab}
    Type:    \> logical \\
    Default: \> \texttt{F} \\
    Range:   \> \texttt{T}, \texttt{F} 
  \end{keytab}
\end{keydescription}

\begin{keydescription}{\texttt{LHISNE2} --- Output of 2-D nuclear energy deposition distributions}
%
  This parameter specifies whether the histogram of the 2-D nuclear energy
  deposition distribution shall be written to the \texttt{HISNE2} file.
  \begin{keytab}
    Type:    \> logical \\
    Default: \> \texttt{F} \\
    Range:   \> \texttt{T}, \texttt{F} 
  \end{keytab}
\end{keydescription}

\begin{keydescription}{\texttt{LHISNE3} --- Output of 3-D nuclear energy deposition distributions}
%
  This parameter specifies whether the histogram of the 3-D nuclear energy
  deposition distribution shall be written to the \texttt{HISNE3} file.
  \begin{keytab}
    Type:    \> logical \\
    Default: \> \texttt{F} \\
    Range:   \> \texttt{T}, \texttt{F} 
  \end{keytab}
\end{keydescription}

\begin{keydescription}{\texttt{LHISIV} --- Output of I--V pair distributions}
%
  This parameter specifies whether the histogram for the I--V pair distribution function shall 
  be written to the \texttt{HISIV} file. \texttt{LHISIV=T} requires the capture radius model to 
  be activated (\texttt{LCAP=T} on the \texttt{\&DAMAGE} record), and is not possible for 
  dynamic simulations (\texttt{LDYN=T} on the \texttt{\&SETUP} record).
  \begin{keytab}
    Type:    \> logical \\
    Default: \> \texttt{F} \\
    Range:   \> \texttt{T}, \texttt{F} 
  \end{keytab}
\end{keydescription}

\ifprivate
\begin{keydescription}{\texttt{LHISIV2} --- Output of I--V pair distributions on 2-D grid}
%
  This parameter specifies whether the histogram for the I--V pair distribution
  function shall be written to the \texttt{HISIV2} file for each box of the 2-D
  histogram. The 2-D histogram is usually the same as that written with
  \texttt{LHIS2=T}.
  \begin{keytab}
    Type:    \> logical \\
    Default: \> \texttt{F} \\
    Range:   \> \texttt{T}, \texttt{F} 
  \end{keytab}
\end{keydescription}
\fi

\begin{keydescription}{\texttt{LHISM} --- Output of first redistributive moment depth
    distribution}
%
  This parameter specifies whether the histogram of the first redistributive moment as a 
  function of depth shall be written to the \texttt{HISM} file. See Section~\ref{s:his1d} for 
  an explanation of the first redistributive moment. 
  \begin{keytab}
    Type:    \> logical \\
    Default: \> \texttt{F} \\
    Range:   \> \texttt{T}, \texttt{F} 
  \end{keytab}
\end{keydescription}

\begin{keydescription}{\texttt{LHISP} --- Output of the deposited momentum depth distribution}
%
  This parameter specifies whether the histogram of the deposited momentum distribution shall 
  be written to the \texttt{HISP} file. See Section~\ref{s:his1d} for an explanation of 
  deposited momentum. 
  \begin{keytab}
    Type:    \> logical \\
    Default: \> \texttt{F} \\
    Range:   \> \texttt{T}, \texttt{F} 
  \end{keytab}
\end{keydescription}

\begin{keydescription}{\texttt{LHIST} --- Output of transmitted spatial distributions}
%
  This parameter specifies whether the histogram for the spatial distribution of the 
  transmitted atoms shall be written to the \texttt{HIST} file.
  \begin{keytab}
    Type:    \> logical \\
    Default: \> \texttt{F} \\
    Range:   \> \texttt{T}, \texttt{F} 
  \end{keytab}
\end{keydescription}

\iffalse
\begin{keydescription}{\texttt{LLINCOLL} --- Linear collision sequence flag}
%
  This parameter specifies whether linear collision sequences shall be
  analyzed. Currently, only linear collision sequences leading to sputtering
  are analyzed. If a linear collision sequence is identified, \texttt{N, DIRLIN(:)} 
  is appended to the respective line in the \texttt{TRA} file. Here,
  \texttt{N} denotes the length of the linear collision sequence, and
  \texttt{DIRLEN} its direction vector in the crystal coordinate
  system. Linear collision sequences are only identified in crystalline regions
  and are reported only in the \texttt{TRA} file.
  \begin{keytab}
    Type:    \> logical \\
    Default: \> \texttt{F} \\
    Range:   \> \texttt{T}, \texttt{F} 
  \end{keytab}
\end{keydescription}
\fi

\begin{keydescription}{\texttt{LMOM} --- Output of spatial atom distribution moments}
%
  This parameter specifies whether the global moments of the ion distribution shall be written 
  to the \texttt{OUT} file. If \texttt{LDAM=T} the moments of the interstitial distribution are
  written as well. If in addition \texttt{LRCOIL=T} also the moments of the vacancies are 
  written.
  \begin{keytab}
    Type:    \> logical \\
    Default: \> \texttt{F} \\
    Range:   \> \texttt{T}, \texttt{F} 
  \end{keytab}
\end{keydescription}

\begin{keydescription}{\texttt{LMOMAAB} --- Output of backscattered atoms
angular moments (3-D)}
%
  This parameter specifies whether the moments of the angular distribution 
  of backscattered ions and backsputtered atoms shall be written to the
  \texttt{OUT} file. The angle is measured with respect to the analyzer direction.
  \begin{keytab}
    Type:    \> logical \\
    Default: \> \texttt{F} \\
    Range:   \> \texttt{T}, \texttt{F} 
  \end{keytab}
\end{keydescription}

\begin{keydescription}{\texttt{LMOMAAT} --- Output of transmitted atoms angular
moments (3-D)}
%
  This parameter specifies whether the moments of the angular distribution 
  of transmitted ions and forward sputtered atoms shall be written to
  the \texttt{OUT} file. The angle is measured with respect to the analyzer direction.
  \begin{keytab}
    Type:    \> logical \\
    Default: \> \texttt{F} \\
    Range:   \> \texttt{T}, \texttt{F} 
  \end{keytab}
\end{keydescription}

\begin{keydescription}{\texttt{LMOMAB} --- Output of backscattered atoms
angular moments (2-D)}
%
  This parameter specifies whether the moments of the angular distribution 
  of backscattered ions and backsputtered atoms shall be written to the
  \texttt{OUT} file. The angle is measured
  between the projection to the $xz$ plane and the negative $z$ axis.
  \begin{keytab}
    Type:    \> logical \\
    Default: \> \texttt{F} \\
    Range:   \> \texttt{T}, \texttt{F} 
  \end{keytab}
\end{keydescription}

\begin{keydescription}{\texttt{LMOMAT} --- Output of transmitted atoms angular
moments (2-D)}
%
  This parameter specifies whether the moments of the angular distribution 
  of transmitted ions and forward sputtered atoms shall be written to
  the \texttt{OUT} file. The angle is measured
  between the projection to the $xz$ plane and the negative $z$ axis.
  \begin{keytab}
    Type:    \> logical \\
    Default: \> \texttt{F} \\
    Range:   \> \texttt{T}, \texttt{F} 
  \end{keytab}
\end{keydescription}

\begin{keydescription}{\texttt{LMOMB} --- Output of backscattered atoms
spatial moments}
%
  This parameter specifies whether the moments of the and spatial distribution 
  of backscattered ions and backsputtered atoms shall be written to the \texttt{OUT} file.
  \begin{keytab}
    Type:    \> logical \\
    Default: \> \texttt{F} \\
    Range:   \> \texttt{T}, \texttt{F} 
  \end{keytab}
\end{keydescription}

\begin{keydescription}{\texttt{LMOMEB} --- Output of backscattered atoms
energy moments}
%
  This parameter specifies whether the moments of the energy distribution 
  of backscattered ions and backsputtered atoms shall be written to the
  \texttt{OUT} file. 
  \begin{keytab}
    Type:    \> logical \\
    Default: \> \texttt{F} \\
    Range:   \> \texttt{T}, \texttt{F} 
  \end{keytab}
\end{keydescription}

\begin{keydescription}{\texttt{LMOMEE} --- Output of electronic energy deposition moments}
%
  This parameter specifies whether the moments of the electronic energy
  deposition distribution shall be written to the \texttt{OUT} file.
  \begin{keytab}
    Type:    \> logical \\
    Default: \> \texttt{F} \\
    Range:   \> \texttt{T}, \texttt{F} 
  \end{keytab}
\end{keydescription}

\begin{keydescription}{\texttt{LMOMET} --- Output of transmitted atoms energy
moments}
%
  This parameter specifies whether the moments of the energy distribution 
  of transmitted ions and forward sputtered atoms shall be written to
  the \texttt{OUT} file. 
  \begin{keytab}
    Type:    \> logical \\
    Default: \> \texttt{F} \\
    Range:   \> \texttt{T}, \texttt{F} 
  \end{keytab}
\end{keydescription}

\begin{keydescription}{\texttt{LMOMNE} --- Output of nuclear energy deposition moments}
%
  This parameter specifies whether the moments of the nuclear energy deposition distribution 
  shall be written to the \texttt{OUT} file.
  \begin{keytab}
    Type:    \> logical \\
    Default: \> \texttt{F} \\
    Range:   \> \texttt{T}, \texttt{F} 
  \end{keytab}
\end{keydescription}

\begin{keydescription}{\texttt{LMOMT} --- Output of transmitted ions spatial moments}
%
  This parameter specifies whether the moments of the spatial distribution 
  of transmitted ions and forward sputtered atoms shall be written to the \texttt{OUT} file.
  \begin{keytab}
    Type:    \> logical \\
    Default: \> \texttt{F} \\
    Range:   \> \texttt{T}, \texttt{F} 
  \end{keytab}
\end{keydescription}

\begin{keydescription}{\texttt{LPERX} --- Periodic boundary condition flag}
%
  This parameter specifies whether the x-coordinates of the positions of events that are used 
  for output, are mapped into the periodic boundary interval (\texttt{XPER} of 
  \texttt{\&geometry} record).
  \begin{keytab}
    Type:    \> logical \\
    Default: \> \texttt{LPERX} of \texttt{\&GEOMETRY} record \\
    Range:   \> \texttt{T} only if \texttt{LPERX=T} of \texttt{\&GEOMETRY}, \texttt{F} 
  \end{keytab}
\end{keydescription}

\begin{keydescription}{\texttt{LRBS} --- RBS energy spectrum histogram}
%
  This parameter specifies whether the RBS simulation is performed and the energy spectrum is 
  written to the \texttt{RBS} file.
  \begin{keytab}
    Type:    \> logical \\
    Default: \> \texttt{F} \\
    Range:   \> \texttt{T}, \texttt{F} 
  \end{keytab}
\end{keydescription}

\ifprivate
\begin{keydescription}{\texttt{LSURF} --- Output of surface}
%
  This parameter specifies whether the surface shall be written to the \texttt{SURF} file.
 \begin{keytab}
    Type:    \> logical \\
    Default: \> \texttt{F} \\
    Range:   \> \texttt{T}, \texttt{F} 
  \end{keytab}
\end{keydescription}
\fi

\begin{keydescription}{\texttt{LTRA} --- Output of trajectory}
%
  This parameter specifies that information about the trajectories shall be written to the 
  \texttt{TRA} file. If \texttt{ITRA} is specified, only the output corresponding to the values 
  of \texttt{ITRA} are written, and \texttt{LTRA} is ignored. If \texttt{LTRA=T} and 
  \texttt{ITRA} is not specified, all possible output is written. If \texttt{LTRA=F} and
  \texttt{ITRA} is not specified, no trajectory file is written (regarding the \texttt{TRA} 
  file and its format see Chapter~\ref{k:files}).
  \begin{keytab}
    Type:    \> logical \\
    Default: \> \texttt{F} \\
    Range:   \> \texttt{T}, \texttt{F} 
  \end{keytab}
\end{keydescription}

\begin{keydescription}{\texttt{MCOORD} --- Coordinate system model}
%
  This parameter specifies the model of the coordinate system to be used with the statistics 
  output. The meaning of the model numbers are specified below.
  \begin{keytab}
    Type:    \> integer \\
    Default: \> 1 \\
    Range:   \> $\texttt{1} \ldots $ same as in \texttt{\&GEOM} \\
             \> $\texttt{2} \ldots $ aligned with the ion's incidence direction \\
             \> $\texttt{3} \ldots $ aligned with direction given by \texttt{TILT} and 
                \texttt{ROTATE} \\
             \> $\texttt{4} \ldots $ aligned with direction given by \texttt{TILTA} and 
                \texttt{ROTATA} \\
             \> $\texttt{5} \ldots $ crystal coordinate system
  \end{keytab}
\end{keydescription}

\begin{keydescription}{\texttt{NBOX} --- Maximum number of 1-D histogram boxes}
%
  This parameter specifies the maximum number of boxes in 1-D histograms. If
  zero, there is no upper bound to the number of boxes, and the box width is
  kept equal to \texttt{WBOX}.
  \begin{keytab}
    Type:    \> integer \\
    Default: \> 32 \\
    Range:   \> $\ge 0$
  \end{keytab}
\end{keydescription}

\begin{keydescription}{\texttt{NBOX2} --- Maximum number of 2-D histogram boxes} 
%
  This parameter specifies the maximum number of boxes in 2-D histograms. The first value 
  corresponds to the x-direction, the second value corresponds to the
  z-direction. If zero, there is no upper bound to the number of boxes in that
  direction, and the box width is kept equal to \texttt{WBOX2}. Either all or
  none of \texttt{NBOX2} must be zero.
  \begin{keytab}
    Type:    \> simple array (2) of integer \\
    Default: \> 36, 24 for point responses \\
    Range:   \> $\ge 0$
  \end{keytab}
\end{keydescription}

\begin{keydescription}{\texttt{NBOX3} --- Maximum number of 3-D histogram boxes} 
%
  This parameter specifies the maximum number of boxes in 3-D histograms. The first value 
  corresponds to the x-direction, the second to the y-direction, and the third one to the 
  z-direction. If zero, there is no upper bound to the number of boxes, and the
  box width is kept equal to \texttt{WBOX3}. Either all or none of \texttt{NBOX3} must be zero. 
  \begin{keytab}
    Type:    \> simple array (3) of integer \\
    Default: \> 36, 36, 24 \\
    Range:   \> $\ge 0$
  \end{keytab}
\end{keydescription}

\begin{keydescription}{\texttt{NBOXA} --- Maximum number of angular histogram boxes}
%
  This parameter specifies the maximum number of boxes in angular histograms
  (with respect to the analyzer direction) in the backscattered/transmitted ions
  statistics.  The maximum number of boxes in angular histograms of the 2-D
  projection will be \texttt{2*NBOXA}. If zero, there is no upper bound to the
  number of boxes, and the box width is kept equal to \texttt{WBOXA}.
  \begin{keytab}
    Type:    \> integer \\
    Default: \> 90 \\
    Range:   \> $\ge 0$
  \end{keytab}
\end{keydescription}

\begin{keydescription}{\texttt{NBOXA2} --- Maximum number of 2-D angular histogram boxes}
%
  This parameter specifies the maximum number of boxes in 2-D angular histograms
  in the backscattered/transmitted ions statistics.  If
  zero, there is no upper bound to the number of boxes, and the box width is
  kept equal to \texttt{WBOXA2}.
  \begin{keytab}
    Type:    \> simple array (2) of integer \\
    Default: \> 50, \texttt{NBOXA2(1)} \\
    Range:   \> $\ge 0$
  \end{keytab}
\end{keydescription}

\begin{keydescription}{\texttt{NBOXD} --- Maximum number of displacement and pair histogram 
    boxes}
%
  This parameter specifies the maximum number of boxes in displacement and pair histograms 
  (\texttt{HISD}, \texttt{HISIV}, and \texttt{HISIV2} files). If zero, there is
  no upper bound to the number of boxes, and the box width is kept equal to
  \texttt{WBOXD}. 
  \begin{keytab}
    Type:    \> integer \\
    Default: \> 100 \\
    Range:   \> $\ge 0$
  \end{keytab}
\end{keydescription}

\begin{keydescription}{\texttt{NBOXE} --- Maximum number of energy histogram boxes}
%
  This parameter specifies the maximum number of boxes in energy histograms in the 
  backscattered/transmitted ions statistics. If zero, there is no upper bound to
  the number of boxes, and the box width is kept equal to \texttt{WBOXE}.
  \begin{keytab}
    Type:    \> integer \\
    Default: \> 100 \\
    Range:   \> $\ge 0$
  \end{keytab}
\end{keydescription}

\begin{keydescription}{\texttt{NHISIV} --- Number of pair distribution histograms}
%
  This parameter specifies the number of pair distribution histograms. The positions of point 
  defects considered may be restricted for each histogram to a range in depth by 
  \texttt{POSIVMIN} and \texttt{POSIVMAX}.
  \begin{keytab}
    Type:    \> integer \\
    Default: \> required if \texttt{LHISIV=T} or \texttt{LHISIV2=T} \\
    Range:   \> $1 \le \texttt{NHISIV} \le 10$
  \end{keytab}
\end{keydescription}

\begin{keydescription}{\texttt{NIONBK} --- Number of ions for backup}
	%
	This parameter specifies that a backup file shall be written/updated after every 
	\texttt{NIONBK} ions. \texttt{NIONBK=0} means that no backup file shall be written. In 
	case \texttt{NIONBK>NION}, a backup file is written after the last ion. \texttt{NIONBK} 
	cannot be specified in case of a dynamic simulation (\texttt{LDYN=T}).
	\begin{keytab}
		Type:    \> integer \\
		Default: \> 0  \\
		Range:   \> $\ge 0$
	\end{keytab}
\end{keydescription}

\begin{keydescription}{\texttt{NIONTRA} --- Number of ions for trajectory output}
	%
	This parameter specifies that a trajectory shall be written to the \texttt{TRA} file for 
	every \texttt{NIONTRA}th ion. \texttt{NIONTRA=1} means that trajectory output shall be 
	written	for every ion. In case \texttt{NIONTRA>NION}, a backup file is written after the 
	last ion. This parameter can be used for debugging to limit the trajectory output to the 
	cascade of interest.
\begin{keytab}
	Type:    \> integer \\
	Default: \> 1  \\
	Range:   \> $> 0$
\end{keytab}
\end{keydescription}

\begin{keydescription}{\texttt{NLATMOM} --- Maximum order of lateral moments}
%
  This parameter specifies the maximum order of the calculated lateral moments.
  \texttt{NLATMOM=0} means there is no output of lateral moments. 
  \begin{keytab}
    Type:    \> integer \\
    Default: \> 2 \\
    Range:   \> $0 \le \texttt{NLATMOM} \le 4$
  \end{keytab}
\end{keydescription}

\begin{keydescription}{\texttt{NMIXMOM} --- Maximum order of mixed moments}
%
  This parameter specifies the maximum order of the calculated mixed moments.
  \texttt{NMIXMOM=0} or \texttt{NMIXMOM=1} means there is no output of mixed moments. 
  \begin{keytab}
    Type:    \> integer \\
    Default: \> 0 \\
    Range:   \> $0 \le \texttt{NMIXMOM} \le 4$
  \end{keytab}
\end{keydescription}

\ifprivate
\begin{keydescription}{\texttt{NUPDOUT} --- Number of updates for writing output}
%
  This parameter specifies the number of cell relaxations (updates) after which data are 
  written to the \texttt{CELL2} and \texttt{SURF} file, if \texttt{LCELL=T} and 
  \texttt{LSURF=T}, respectively. There will be \texttt{NION/(NIONUPD*NUPDOUT} records written 
  to the \texttt{CELL2} and \texttt{SURF} file.  No data are written if \texttt{NIONUPD=0}. In 
  addition, in any case one record is written at the beginning and at the end of the 
  simulation if \texttt{LCELL=T} and \texttt{LSURF=T}, respectively. 
  \begin{keytab}
    Type:    \> integer \\
    Default: \> 1 \\
    Range:   \> $\ge 0$
  \end{keytab}
\end{keydescription}
\fi

\begin{keydescription}{\texttt{POSIVMAX} --- Maximum depth for pair distribution histogram}
%
  This parameter specifies the maximum depth of the point defects to be considered for each 
  pair distribution histogram.
  \begin{keytab}
    Type:    \> simple array (\texttt{NHISIV}) of real \\
    Default: \> $\infty$ \\
    Range:   \> $\texttt{POSIVMIN} < \texttt{POSIVMAX}$
  \end{keytab}
\end{keydescription}

\begin{keydescription}{\texttt{POSIVMIN} --- Minimum depth for pair distribution histogram}
%
  This parameter specifies the minimum depth of the point defects to be considered for each 
  pair distribution histogram.
  \begin{keytab}
    Type:    \> simple array (\texttt{NHISIV}) of real \\
    Default: \> 0 \\
    Range:   \> $\texttt{POSIVMIN} < \texttt{POSIVMAX}$
  \end{keytab}
\end{keydescription}

\begin{keydescription}{\texttt{ROTATA} --- Analyzer rotation angle}
%
  This parameter specifies the rotation angle of the sample with respect to the analyzer used 
  for transmission and backscattering statistics. For the meaning of the analyzer direction see
  \texttt{TILTA}. {\tt ROTATA} is defined analogously to \texttt{ROTATE}. The units of 
  \texttt{ROTATA} are degrees.
  \begin{keytab}
    Type:    \> real \\
    Default: \> 0 
  \end{keytab}
\end{keydescription}

\begin{keydescription}{\texttt{SHORTTRA} --- Shorten trajectory file}
%
  This parameter is a flag, if set, specifying that collision points
  along the trajectory (\texttt{IFLAG=2}) are only written to the \texttt{TRA}
  file if a recoil starts motion in this collision. \texttt{SHORTTRA} is ignored
  unless \texttt{ITRA='..,2,..'} or \texttt{LTRA=T} is specified.
  \begin{keytab}
    Type:    \> logical \\
    Default: \> \texttt{F} \\
    Range:   \> \texttt{T}, \texttt{F}
  \end{keytab}
\end{keydescription}

\begin{keydescription}{\texttt{TILTA} --- Analyzer tilt angle}
%
  This parameter specifies the tilt angle of the sample with respect to the
  analyzer direction. The analyzer direction is the reference direction for
  \texttt{AMINA} and \texttt{AMAXA}. In addition, it defines the statistics
  coordinate system if \texttt{MCOORD=4}. For RBS statistics, it is used
  independent of the value of \texttt{MCOORD}. For backscattering statistics and
  RBS, the analyzer is just opposite the direction specified by \texttt{TILTA}
  and \texttt{ROTATA}, i.e., in these cases \texttt{TILTA=0} means the analyzer
  is in negative $z$ direction, while otherwise it is in positive $z$ direction.
  In particular, for RBS, \texttt{TILTA=0} means the RBS detector is
  located along the negative $z$ axis ($180^\circ$ backscattering angle). The
  units of \texttt{TILTA} are degrees.
  \begin{keytab}
    Type:    \> real \\
    Default: \> 0 \\
    Range:   \> $0 \le \texttt{TILTA} \le 90$
  \end{keytab}
\end{keydescription}

\begin{keydescription}{\texttt{WBOX} --- Initial 1-D histogram box width}
%
  This parameter specifies the initial box width of all 1-D histograms. If
  \texttt{NBOX=0}, the box widths remain constant throughout the simulation.
  Otherwise, they are doubled whenever necessary to accomodate all scores in
  \texttt{NBOX} boxes. The units are \AA. The value of \texttt{WBOX} might be
  slightly adjusted by IMSIL if \texttt{LVAC=T} is set. 
  \begin{keytab}
    Type:    \> real \\
    Default: \> 100/128 \\
    Range:   \> $> 0$
  \end{keytab}
\end{keydescription}

\begin{keydescription}{\texttt{WBOX2} --- Initial 2-D histogram box widths}
%
  This parameter specifies the initial box widths of all 2-D histograms. The first value 
  corresponds to the box width in $x$ direction, the second value corresponds to the box width 
  in $z$ direction. If \texttt{NBOX2=0}, the box widths remain constant
  throughout the simulation. Otherwise, they are doubled whenever necessary
  to accomodate all scores in \texttt{NBOX2} boxes. The values of \texttt{WBOX2}
  might be slightly adjusted by IMSIL if \texttt{LVAC=T} is set. The units are
  \AA.  
  \begin{keytab}
    Type:    \> simple array (2) of real \\
    Default: \> \texttt{WBOX}, \texttt{WBOX2(1)} \\
    Range:   \> $> 0$
  \end{keytab}
\end{keydescription}

\begin{keydescription}{\texttt{WBOX3} --- Initial 3-D histogram box widths}
%
  This parameter specifies the initial box widths of the 3-D histograms. The first value 
  corresponds to the box width in $x$ direction, the second to the box width in $y$ direction, 
  and the third one to the box width in $z$ direction. If \texttt{NBOX3=0}, the
  box widths remain constant throughout the simulation. Otherwise, they are
  doubled whenever necessary to accomodate all scores in \texttt{NBOX3} boxes.
  The values of \texttt{WBOX3} might be slightly adjusted by IMSIL if
  \texttt{LVAC=T} is set. The units are \AA. 
  \begin{keytab}
    Type:    \> simple array (3) of real \\
    Default: \> \texttt{WBOX}, \texttt{WBOX3(1)}, \texttt{WBOX3(1)} \\
    Range:   \> $> 0$
  \end{keytab}
\end{keydescription}

\begin{keydescription}{\texttt{WBOXA} --- Initial angular histogram box width}
%
  This parameter specifies the initial box width of all angular histograms for the 
  backscattered/transmitted ions statistics. If \texttt{NBOXA=0}, the box
  widths remain constant throughout the simulation. Otherwise, they are doubled
  whenever necessary to accomodate all scores in \texttt{NBOXA} boxes. The
  units are degrees. 
  \begin{keytab}
    Type:    \> real \\
    Default: \> 1 \\
    Range:   \> $> 0$
  \end{keytab}
\end{keydescription}

\begin{keydescription}{\texttt{WBOXA2} --- Initial 2-D angular histogram box width}
%
  This parameter specifies the initial box widths of all 2-D angular histograms
  for the backscattered/transmitted ions statistics. If \texttt{NBOXA2=0}, the
  box widths remain constant throughout the simulation. Otherwise, they are doubled
  whenever necessary to accomodate all scores in \texttt{NBOXA2} boxes. Note
  that the independent variables of 2-D angular distributions are the direction
  cosines with respect to the $x$ and $y$ axis.
  \begin{keytab}
    Type:    \> simple array (2) of real \\
    Default: \> 0.01, \texttt{WBOXA2(1)} \\
    Range:   \> $> 0$
  \end{keytab}
\end{keydescription}

\begin{keydescription}{\texttt{WBOXD} --- Initial displacement histogram box width}
%
  This parameter specifies the initial box width of the displacement histograms 
  (\texttt{HISD} file). If \texttt{NBOXD=0}, the box widths remain constant
  throughout the simulation. Otherwise, they are doubled whenever necessary
  to accomodate all scores in \texttt{NBOXD} boxes. The units are \AA\ if
  \texttt{DPOWER=1}. If \texttt{DPOWER>1}, the box width applies to 
  $d^\texttt{1/DPOWER}$, and the units of \texttt{WBOXD} are
  \AA$^\texttt{1/DPOWER}$. As a result, the right histogram edges are at
  $d = (i \cdot \texttt{WBOXD})^\texttt{DPOWER}$ where $i$ are integers.
  \begin{keytab}
    Type:    \> real \\
    Default: \> 0.1 \\
    Range:   \> $> 0$
  \end{keytab}
\end{keydescription}

\begin{keydescription}{\texttt{WBOXE} --- Initial energy histogram box width}
%
  This parameter specifies the initial box width of all energy histograms for the 
  RBS/backscattered/transmitted ions statistics. If \texttt{NBOXE=0}, the box
  widths remain constant throughout the simulation. Otherwise, they are doubled
  whenever necessary to accomodate all scores in \texttt{NBOXE} boxes. The
  units are eV if \texttt{EPOWER=1}. If \texttt{EPOWER>1}, the box width applies to
  $E^\texttt{1/EPOWER}$, and the units of \texttt{WBOXE} are
  eV$^\texttt{1/EPOWER}$. As a result, the right histogram edges are at
  $E = (i \cdot \texttt{WBOXE})^\texttt{EPOWER}$ where $i$ are integers.
  \begin{keytab}
    Type:    \> real \\
    Default: \> \texttt{ENERGY**(1/EPOWER)/NBOXE} \\
    Range:   \> $> 0$
  \end{keytab}
\end{keydescription}

