In the first part of the \texttt{OUT} file, most of the input parameters,
including those which have not been specified in the \texttt{INP} file and
therefore assume their default values, and some internal parameters are listed.
In the second part, after the header ``\texttt{SIMULATION RESULTS}'',
statistical output data are listed. This starts with energy deposition and
energy transfer statistic. Energy \textbf{transfer} means energy transfers from
a projectile to a recoil that is set into motion. \textbf{Deposition} denotes
all other kinds of energy transfer or loss by a projectile. This includes
electronic energy loss (``\texttt{Electrons}'', called ``ionization'' in TRIM
\cite{SRIM}), nuclear energy deposition (``\texttt{Phonons+Potential}'', called
``phonons'' in TRIM), and the total energy of backscattered and transmitted
projectiles.

For the purpose of this manual, we define \textbf{yield} as the average amount
of the quantity of interest for one impinging ion. ``Yield'' is commonly used in
``sputtering yield'', which means the number of sputtered atoms per incident
ion. Similarly, we define the yield of stopped atoms (ions or recoils) and
vacancies, of energy deposition, and of backscattered and transmitted
projectiles. The yield of deposited energy, e.g., is the total deposited energy
per ion.

Normally, one is interested only in the mean value $\langle Y \rangle$ of the
yield. However, the yields $Y_i$ are different for each impinging ion $i$, so
there is a yield distribution, which can be described by \textbf{moments}. The
moments are defined by
%
\begin{eqnarray}
    \mathrm{mean\ value}: & 
        \langle Y \rangle = \frac{1}{N} \sum_{i=1}^N Y_i \label{eq:mom1} \\
    \mathrm{standard\ deviation}: & 
        \sigma_\mathrm{Y} = \left[ \frac{1}{N} \sum_{i=1}^N (Y_i-\langle Y \rangle)^2  
        \right] ^ {1/2} \label{eq:mom2} \\
    \mathrm{skewness}: &
        \gamma_\mathrm{Y} = \frac{1}{N} \sum_{i=1}^N (Y_i-\langle Y \rangle)^3 / 
        \sigma_\mathrm{Y}^3 \label{eq:mom3} \\
    \mathrm{kurtosis}: &
        \beta_\mathrm{Y} = \frac{1}{N} \sum_{i=1}^N (Y_i-\langle Y \rangle)^4 / 
        \sigma_\mathrm{Y}^4 \label{eq:mom4}
\end{eqnarray}
%
These moments are listed in the first four lines following a header. In the
following four lines, the 1-$\sigma$ standard deviations (corresponding to the
statistical uncertainties) of the moments are listed.

The energy statistics and the yield moments are always written to the
\texttt{OUT} file. Other moments are written if requested in the input file,
see the \texttt{LMOM**} parameters of the \texttt{\&OUTPUT} record. In the
moment definitions for other quantities than the sputter yield, $Y$ has to be
replaced in Eqs.~(\ref{eq:mom1})--(\ref{eq:mom4}) by the quantity of interest.
E.g., for the moments of a ``vertical'' distribution, $Y_i$ has to be replaced
by the coordinate $z_i$ of the stopped projectile. Note that what is commonly
called the projected range of the implanted ions, is listed under 
\texttt{Spatial moments of stopped atoms and vacancies}/%
\texttt{Vertical moments}/\texttt{mean value}, the longitudinal straggling
under \texttt{Spatial moments of stopped atoms and vacancies}/%
\texttt{Vertical moments}/\texttt{std.dev.}, and the lateral straggling under 
\texttt{Spatial moments of stopped atoms and vacancies}/%
\texttt{Lateral x-moments}/\texttt{std.dev.\ }and/or
\texttt{Spatial moments of stopped atoms and vacancies}/%
\texttt{Lateral y-moments}/\texttt{std.dev.}. These and other parameters can be
extracted using the \texttt{read\_out} function of the \texttt{read\_output.py}
module provided with IMSIL (see Section~\ref{k:scripts}).
