The geometry of the target is either described in terms of regions or in terms 
of cells. In 1-D, the geometry is simply defined by the positions of the surfaces
and the interfaces between the regions (\texttt{POSIF}), e.g.,
%
\begin{verbatim}
 &geometry posif=-10,0,10000 /
\end{verbatim}
%
Cells are usually finer-grained than regions. Only regions are specified on the
\texttt{\&GEOMETRY} record. Cells are either defined by reading a \texttt{CELL}
file, or they are internally generated from regions in case of a dynamic
simulation (\texttt{LDYN=T}).

In 2-D, the geometry of each region is specified by a closed polygon. To define
the polygons, it is first necessary to specify the coordinates of the vertices
(corner points) using the \texttt{POINT} index variable, e.g.,
%
\begin{verbatim}
 &geometry point=1 pos=0,0 /
 &geometry point=2 pos=-5000,0 /
 ...
\end{verbatim}
%
and then to specify the polygons by listing the point indices in the desired
sequence, using the \texttt{REGION} index variable:
%
\begin{verbatim}
 &geometry region=0 points=1,2,3,4,5,6,7,8 /
 &geometry region=1 points=1,4,5,6 /
 ...
\end{verbatim}
%
Note that the circumference of the complete simulation area can be specified as
region 0. If it is not specified, it is determined by IMSIL. There must not be
undefined areas within the circumference, and no regions may overlap.

For 2-D geometries, IMSIL uses an internal mesh containing the minimum distance
of each cell from the boundary. It is used to reduce the number of checks that
are necessary to determine whether the ion leaves a region.

It is possible to impose cylindrical (\texttt{LCYLX=T}, \texttt{LCYLY=T}, or
\texttt{LCYLZ=T}) or spherical (\texttt{LSPHER=T}) symmetry. For cylindrical
geometry, the distance along the axis of symmetry replaces the $x$ coordinate,
while the distance from the axis replaces the $z$ coordinate. For spherical
symmetry, the distance from the center of symmetry replaces the $z$ coordinate. 

\begin{keydescription}{\texttt{CENTER} --- Center of cylindrical and spherical 
regions}
%
  This parameter specifies the center of the cylindrical or spherical regions,
  if \texttt{LCYLX=T} or \texttt{LCYLY=T} or \texttt{LCYLZ=T} or 
  \texttt{LSPHER=T}. The three components of \texttt{CENTER} correspond to $x$, 
  $y$, and $z$ coordinate of the center. For cylindrical symmetries, the
  symmetry line passes through the position specified by \texttt{CENTER} and is
  parallel to the $x$, $y$, or $z$ axis depending on whether the \texttt{LCYLX},
  \texttt{LCYLY}, or \texttt{LCYLZ} flag is set.
  \begin{keytab}
    Type:    \> simple array (3) of real \\
    Default: \> 0., 0., 0. \\
    Range:   \> arbitrary
  \end{keytab}
\end{keydescription}

\begin{keydescription}{\texttt{FRACUPD} --- Minimum fractional density
change for cell update}
%
  A mesh update is triggered in a 1-D dynamic simulation if the change of the
  cell width relative to the nominal cell width (\texttt{WBOX} of the
  \texttt{\&OUTPUT} record) would exceed \texttt{FRACUPD} in any cell. The cell
  widths change due to relaxation of density changes. The condition is only
  checked every \texttt{NIONUPD} ions. 
  \begin{keytab}
    Type:    \> integer \\
    Default: \> 0.1 \\
    Range:   \> $\ge 0$
  \end{keytab}
\end{keydescription}

\ifprivate
\begin{keydescription}{\texttt{FREEX} --- Free boundaries in x-direction}
%
  This parameter specifies whether the left/right boundary of the
  simulation domain is free to move during relaxation steps. The simulation
  domain is the smallest possible rectangle, consistent with the grid spacings,
  containing the target. Therefore, this boundary condition only applies to
  target boundaries with (close to) the smallest/largest coordinates.
  \begin{keytab}
    Type:    \> simple array (2) of logical \\
    Default: \> \texttt{(F,F)} \\
    Range:   \> \texttt{T}, \texttt{F} 
  \end{keytab}
\end{keydescription}

\begin{keydescription}{\texttt{FREEZ} --- Free boundaries in z-direction}
%
  This parameter specifies whether the top/bottom boundary of the
  simulation domain is free to move during relaxation steps. The simulation
  domain is the smallest possible rectangle, consistent with the grid spacings,
  containing the target. Therefore, this boundary condition only applies to
  target boundaries with (close to) the smallest/largest coordinates.
  \begin{keytab}
    Type:    \> simple array (2) of logical \\
    Default: \> \texttt{(T,F)} \\
    Range:   \> \texttt{T}, \texttt{F} 
  \end{keytab}
\end{keydescription}
\fi

\begin{keydescription}{\texttt{GEOMFILE} --- Geometry file}
%
  This parameter specifies the name of a geometry file, which contains further 
  \texttt{\&GEOMETRY} records with \texttt{POINT} and \texttt{REGION}
  definitions. This has the same effect as if the contents of the geometry file
  was inserted before the first \texttt{\&GEOMETRY} record of the input file.
  As a consequence, any specifications in the input file will overwrite
  corresponding specifications in the geometry file. Any other content of the
  geometry file other than \texttt{\&GEOMETRY} records is ignored. If multiple
  geometry files are specified, only the last one will be included. The geometry
  file is intended for the specification of 2-D geometries (points and regions),
  so one geometry file can be used for multiple input files. If a geometry file
  is specified, the geometry points and regions are not written to the output
  file.
  \texttt{GEOMFILE} is ignored for 1-D geometries (\texttt{NDIM=1}).
  \begin{keytab}
    Type:    \> character string \\
    Default: \> -- \\
    Range:   \> Any valid file name ($\le 80$ characters)
  \end{keytab}
\end{keydescription}

\ifprivate
\begin{keydescription}{\texttt{GRID} --- Grid type}
%
  This parameter specifies the type of the grid defining the cells for a
  2-D dynamic simulation (\texttt{LDYN=T}).
  \begin{keytab}
    Type:    \> character(80) \\
    Default: \> \texttt{'Cartesian'} \\
    Range:   \> \texttt{'Cartesian'}, \texttt{'curvilinear'}
  \end{keytab}
\end{keydescription}
\fi

\begin{keydescription}{\texttt{LPERX} --- Periodic x-boundary conditions flag}
%
  This parameter specifies whether periodic boundary conditions should be
  used in x-direction in a 2-D simulation. \texttt{LPERX=T} requires
  specification of \texttt{XPER}.
  \begin{keytab}
    Type:    \> logical \\
    Default: \> \texttt{F} \\
    Range:   \> \texttt{T}, \texttt{F} 
  \end{keytab}
\end{keydescription}

\begin{keydescription}{\texttt{LCYLX} --- Cylinder region flag, axis parallel $x$}
%
  This parameter specifies that the target should be considered cylindrical 
  symmetric. The axis of rotation is parallel to the $x$ axis and goes through 
  \texttt{CENTER}. For \texttt{NDIM=1} the radii of the 
  cylindrical regions are specified by \texttt{POSIF}. For \texttt{NDIM=2}
  the polygon defined by \texttt{POS} and \texttt{POINTS} is rotated around 
  the axis. \texttt{POS(1)} is considered the coordinate along the axis, 
  \texttt{POS(2)} the distance from the axis (radius). Any combination of two
  or more of \texttt{LCYLX=T}, \texttt{LCYLY=T}, \texttt{LCYLZ=T}, 
  \texttt{LSPHER=T}, and \texttt{LROBIN=T} is illegal. \texttt{LCYLX=T} can 
  only be specified for \texttt{LDYN=F}.
  \begin{keytab}
    Type:    \> logical \\
    Default: \> \texttt{F} \\
    Range:   \> \texttt{T}, \texttt{F} 
  \end{keytab}
\end{keydescription}

\begin{keydescription}{\texttt{LCYLY} --- Cylinder region flag, axis parallel $y$}
%
  This parameter specifies that the target should be considered cylindrical 
  symmetric. The axis of rotation is parallel to the $x$ axis and goes through 
  \texttt{CENTER}. For \texttt{NDIM=1} the radii of the 
  cylindrical regions are specified by \texttt{POSIF}. For \texttt{NDIM=2}
  the polygon defined by \texttt{POS} and \texttt{POINTS} is rotated around 
  the axis. \texttt{POS(1)} is considered the coordinate along the axis, 
  \texttt{POS(2)} the distance from the axis (radius). Any combination of two
  or more of \texttt{LCYLX=T}, \texttt{LCYLY=T}, \texttt{LCYLZ=T}, 
  \texttt{LSPHER=T}, and \texttt{LROBIN=T} is illegal. 
  \texttt{LCYLY=T} can only be specified for \texttt{LDYN=F}.
  \begin{keytab}
    Type:    \> logical \\
    Default: \> \texttt{F} \\
    Range:   \> \texttt{T}, \texttt{F} 
  \end{keytab}
\end{keydescription}

\begin{keydescription}{\texttt{LCYLZ} --- Cylinder region flag, axis parallel $z$}
%
  This parameter specifies that the target should be considered cylindrical 
  symmetric. The axis of rotation is parallel to the $x$ axis and goes through 
  \texttt{CENTER}. For \texttt{NDIM=1} the radii of the 
  cylindrical regions are specified by \texttt{POSIF}. For \texttt{NDIM=2}
  the polygon defined by \texttt{POS} and \texttt{POINTS} is rotated around 
  the axis. \texttt{POS(1)} is considered the coordinate along the axis, 
  \texttt{POS(2)} the distance from the axis (radius). Any combination of two
  or more of \texttt{LCYLX=T}, \texttt{LCYLY=T}, \texttt{LCYLZ=T}, 
  \texttt{LSPHER=T}, and \texttt{LROBIN=T} is illegal. 
  \texttt{LCYLZ=T} can only be specified for \texttt{LDYN=F}.
  \begin{keytab}
    Type:    \> logical \\
    Default: \> \texttt{F} \\
    Range:   \> \texttt{T}, \texttt{F} 
  \end{keytab}
\end{keydescription}

\begin{keydescription}{\texttt{LSPHER} --- Spherical region flag}
%
  This parameter specifies that the target regions should be
  considered spherical around the center given by \texttt{CENTER} rather than 
  planar. The radii of the spherical regions are specified by \texttt{POSIF}. 
  Any combinations of \texttt{LCYLX=T}, \texttt{LCYLY=T}, \texttt{LCYLZ=T}, 
  \texttt{LSPHER=T}, and \texttt{LROBIN=T} must not be specified. 
  \texttt{LSPHER=T} can only be specified for \texttt{NDIM=1}, \texttt{LDYN=F}.
  \begin{keytab}
    Type:    \> logical \\
    Default: \> \texttt{F} \\
    Range:   \> \texttt{T}, \texttt{F} 
  \end{keytab}
\end{keydescription}

\begin{keydescription}{\texttt{NBOX2} --- Maximum number of mesh boxes}
%
  This parameter specifies the maximum number of boxes in each direction 
  of the internal mesh. Geometry checks are performed by interpolation of the 
  distance function stored on the internal mesh. Therefore, a finer mesh means 
  a finer resolution of the geometry. The first value of \texttt{NBOX2} 
  corresponds to the number of boxes in x-direction, the second value 
  corresponds to the number of boxes in z-direction. Each of the values of 
  \texttt{NBOX2} is effective only if the number of boxes necessary to cover 
  the geometry with a box width of the respective value of \texttt{WBOX2} 
  would exceed \texttt{NBOX2}. In this case the box width is adjusted so that 
  the number of boxes equals \texttt{NBOX2}. Thus, \texttt{NBOX2} sets upper 
  limits to the number of boxes used in the simulation. The first (x) value of 
  \texttt{NBOX2} is ignored if \texttt{LPERX=T}.
  \begin{keytab}
    Type:    \> simple array (2) of integer \\
    Default: \> 1000, \texttt{NBOX2(1)} \\
    Range:   \> $> 0$
  \end{keytab}
\end{keydescription}

\begin{keydescription}{\texttt{NIONUPD} --- Number of ions for mesh update}
%
  This parameter specifies the number of implanted ions after which the
  cell sizes are updated in a 1-D dynamic simulation, see \texttt{FRACUPD}). 
  Note that too infrequent mesh updates affect the simulation results.
  \texttt{NIONUPD} must not appear on \texttt{POINT} or \texttt{REGION} records.  
  \begin{keytab}
    Type:    \> integer \\
    Default: \> 1 \\
    Range:   \> $> 0$
  \end{keytab}
\end{keydescription}

\begin{keydescription}{\texttt{POINT} --- Point index}
%
  This parameter specifies the index of a point whose coordinates are specified
  on the same record by \texttt{POS}. The point indices are referred to by the
  \texttt{POINTS} arrays. 
  \begin{keytab}
    Type:    \> integer \\
    Default: \> -- \\
    Range:   \> $>0$ 
  \end{keytab}
\end{keydescription}

\begin{keydescription}{\texttt{POINTS} --- Points defining a region polygon}
%
  This parameter specifies the indices of the points (as defined by
  \texttt{POINT}) defining the region polygon.  Specification of
  \texttt{POINTS} requires the index variable \texttt{REGION} on the same
  record. A maximum of 20 points may be specified on one record.
  Specifications of \texttt{POINTS} on different records for the same
  \texttt{REGION} are joined together.  In this way, larger arrays of points 
  may be split to several records.  
  \begin{keytab}
    Type:    \> index variable array (\texttt{REGION}) of simple array
    (arbitrary size) of integer \\
    Default: \> -- \\
    Range:   \> point indices defined by \texttt{POINT} 
  \end{keytab}
\end{keydescription}

\begin{keydescription}{\texttt{POS} --- Position of an interface point}
%
  This parameter specifies the position of an interface point. The
  position is a simple array of size 2, consisting of the $x$ and $z$
  coordinate in case of Cartesian coordinates. In case of cylindrical
  coordinates \texttt{POS(1)} refers to the distance along the axis of
  symmetry, and \texttt{POS(2)} to the distance from the axis of symmetry.
  The units of the coordinates are \AA.  Specification of \texttt{POS} 
  requires the index variable \texttt{POINT} on the same record. 
  \begin{keytab}
    Type:    \> index variable array (\texttt{POINT}) of simple array
                (2) of real \\
    Default: \> -- \\
   Range:    \> arbitrary
\end{keytab}
\end{keydescription}

\begin{keydescription}{\texttt{POSFIX} --- Fixed target position}
%
  This parameter specifies the position which remains fixed during
  target relaxation in a 1-D dynamic simulation (\texttt{NDIM=1},
  \texttt{LDYN=T}) or during the initial relaxation when the geometry is read 
  from a \texttt{CELL} file (\texttt{NDIM=1}, \texttt{USECELL=T}).  If 
  \texttt{POSFIX} is unspecified, it is assumed in the center of the target.
  The center position is updated during a dynamic simulation. \texttt{POSFIX}
  also falls back to the target center if it happens to fall outside the target
  in the course of a dynamic simulation.  The units of \texttt{POSFIX} are \AA.
  \begin{keytab}
    Type:    \> real \\
   Default:  \> center of simulation domain \\
   Range:    \> arbitrary
\end{keytab}
\end{keydescription}

\begin{keydescription}{\texttt{POSIF} --- Positions of the interfaces}
%
  This parameter specifies the positions of the interfaces between
  neighboring regions in case of a 1-D simulation (\texttt{NDIM=1} on
  the \texttt{\&SETUP} record). The first value of \texttt{POSIF}
  defines the front surface of the target, the last value the back
  surface.  A maximum of 20 positions may be specified on one record.
  Specifications of \texttt{POSIF} on different records are joined together.  In
  this way, larger arrays of positions may be split to several records.  The
  units of \texttt{POSIF} are \AA. 
  \begin{keytab}
    Type:    \> simple array (\texttt{NR+1}) of real \\
    Default: \> 0., 10$^{30}$ if only one region, \\
             \> required otherwise \\
    Range:   \> values must be increasing
  \end{keytab}
\end{keydescription}

\begin{keydescription}{\texttt{WBOX2} --- 2-D box width}
%
  This parameter specifies the box width of the internal mesh. The first 
  value corresponds to the cell width in x-direction, the second value 
  corresponds to the cell width in z-direction. The values of 
  \texttt{WBOX2} might be adjusted by IMSIL if the maximum number of boxes
  \texttt{NBOX2} were exceeded, or in case of periodic boundary conditions
  to fit an integer number of intervals between the boundaries. The units 
  are \AA.  Geometry checks are performed by interpolation of the distance
  function stored on the internal mesh. Therefore, a finer mesh means a finer
  resolution of the geometry.
  \begin{keytab}
    Type:    \> simple array (2) of real \\
    Default: \> 1.0, \texttt{WBOX2(1)} \\
    Range:   \> $> 0$
  \end{keytab}
\end{keydescription}

\begin{keydescription}{\texttt{WROUGH} --- Thickness of roughness layer}
%
  This parameter specifies the thickness of the surface roughness layer.
  Within the roughness layer, the density is assumed to increase linearly
  from 0 (at the surface) to its nominal value (at the other bound of the
  roughness layer). The units are \AA. 
  \begin{keytab}
    Type:    \> real \\
    Default: \> 0 \\
    Range:   \> $\ge 0$
  \end{keytab}
\end{keydescription}

\begin{keydescription}{\texttt{XPER} --- x-positions of periodic boundaries}
%
  This parameter specifies the positions of the boundaries in case of 
  periodic boundary conditions (\texttt{LPERX=T}). No points (\texttt{POS}) 
  must be outside these boundaries. In general, some points will be on the
  boudaries. These points will be snapped to the boundary with a tolerance 
  corresponding to the accuracy of single precision floating point data.
  The units are \AA. 
  \begin{keytab}
    Type:    \> simple array (2) of real \\
    Default: \> -- \\
    Range:   \> $\texttt{XPER(2)} > \texttt{XPER(1)}$
  \end{keytab}
\end{keydescription}

\begin{keydescription}{\texttt{XSCALE} --- Scaling factor in x-direction}
%
  This parameter specifies a scaling factor for the $x$ coordinates of the
  polygon points as specified by \texttt{POS}. All values \texttt{POS(1)} will
  be multiplied by \texttt{XSCALE}. Note that \texttt{XPER} and
  \texttt{WBOX2(1)} will not be scaled. 
  \begin{keytab}
    Type:    \> real \\
    Default: \> 1 \\
    Range:   \> arbitrary
  \end{keytab}
\end{keydescription}

\begin{keydescription}{\texttt{ZSCALE} --- Scaling factor in z-direction}
%
  This parameter specifies a scaling factor for the $z$ coordinates of the
  polygon points as specified by \texttt{POS}. All values \texttt{POS(2)}
  will be multiplied by \texttt{ZSCALE}. Note that \texttt{WBOX2(2)} will not be
  scaled.
  \begin{keytab}
    Type:    \> real \\
    Default: \> 1 \\
    Range:   \> arbitrary
  \end{keytab}
\end{keydescription}

