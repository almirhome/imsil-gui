This record is used to define the parameters of nuclear stopping. For
a description of some of the parameters refer to Chapters \ref{s:coll}
and \ref{s:pot}.  The \texttt{\&SNPAR} record has the index variables
\texttt{ATOM1} and \texttt{ATOM2}. \texttt{ATOM1} corresponds to the index 
of the projectile, while \texttt{ATOM2} corresponds to the index of the target
atom/recoil. Regarding the definition of atom indices, see Section~\ref{s:atom}.


\begin{keydescription}{\texttt{CHKFFPCYL} --- Check ffp cylinder flag}
%
  This parameter specifies whether a check is to be performed on whether
  a recoil partner in an amorphous region is inside the previous free flight 
  path cylinder. If this is the case, the collision is abandoned. 
  \begin{keytab}
    Type:    \> logical \\
    Default: \> \texttt{F} \\
    Range:   \> \texttt{T}, \texttt{F}
  \end{keytab}
\end{keydescription}

\begin{keydescription}{\texttt{COEFFILE} --- Screening function coefficient file}
%
  This parameter specifies the file containing the coefficients of a
  sum-of-exponentials screening function. This model is used for the calculation 
  of the scattering angles, if \texttt{QUAD} specifies a numerical integration
  method and \texttt{SCRFUN='coeffile'}. Files are assumed to be located in the 
  subdirectory \texttt{'screencoefs'} of the search path. 
  \begin{keytab}
    Type:    \> index variable array (\texttt{ATOM2},\texttt{ATOM1}) 
                of character strings \\ 
    Default: \> \texttt{'ZBLspec'} \\
    Range:   \> \texttt{'ZBLspec'} or the name of a user-supplied file 
                containing screening function \\
             \> coefficients ($\le$ 80 characters)
  \end{keytab}
\end{keydescription}

\begin{keydescription}{\texttt{DENFFP} --- Minimum energy transfer}
%
  This parameter specifies the minimum energy transfer in eV which has
  to be considered in a collision. It is used for the calculation of
  the free flight path in amorphous targets. The free flight path is
  chosen such that both the energy transfer is larger than
  \texttt{DENFFP} and the scattering angle is larger than
  \texttt{PSIFFP}.
  \begin{keytab}
    Type:    \> real \\
    Default: \> 15 \\
    Range:   \> $>0$
  \end{keytab}
\end{keydescription}

\begin{keydescription}{\texttt{EF} --- Final projectile energy}
%
  This parameter specifies the cutoff energy in eV below which the
  simulation of the ion trajectory is terminated. For \texttt{LSPUT=T} the
  cutoff energy is reduced to the minimum surface binding energy for position
  closer to the surface than \texttt{DNEAR} of the \texttt{DAMAGE} record.
  Note that additional displacement is possible at the end of the trajectory if
  \texttt{LFLOW=T}. The units of \texttt{EF} are eV. 
  \begin{keytab}
    Type:    \> real \\
    Default: \> 10 \\
    Range:   \> $0 < \texttt{EF} \le \texttt{ENERGY}$
  \end{keytab}
\end{keydescription}

\begin{keydescription}{\texttt{ETIME} --- Maximum energy for time integral}
%
  This parameter specifies the energy below which the time
  integral is considered for the calculation of the turning points of
  the trajectories. The turning point is the intersection of the asymptotes
  of the classical scattering problem. If the turning point is beyond the
  plane perpendicular to the incidence direction that contains the collision
  partner, the collision point is moved into that plane. The units of
  \texttt{ETIME} are keV.
  \begin{keytab}
    Type:    \> real \\
    Default: \> 5 \\
    Range:   \> $\ge 0$
  \end{keytab}
\end{keydescription}

\begin{keydescription}{\texttt{FFPMAX} --- Maximum free flight path}
%
  This parameter specifies an upper limit to the free flight path in amorphous
  material.  \texttt{FFPMAX} is applied to the statistical free flight path, not
  to its mean value obtained from \texttt{DENFFP} and \texttt{PSIFFP}. 
  \texttt{FFPMAX} is silently reduced to the size of the simulation domain in
  $z$ direction (but not to less than 0.1\AA) in a dynamic simulation
  (\texttt{LDYN=T}) or when the target is read from a \texttt{CELL} file
  (\texttt{LCELL=T}). The units of \texttt{FFPMAX} are \AA.
  \begin{keytab}
    Type:    \> real \\
    Default: \> 10\% of target thickness \\
    Range:   \> $0 \le \texttt{FFPMAX} \le$ size of simulation domain
  \end{keytab}
\end{keydescription}

\begin{keydescription}{\texttt{LAPSISEF} --- Apsis cutoff energy flag}
%
  If true, this parameter specifies that the kinetic energy at the apsis of 
  collision rather than the energy after collision should be checked against 
  \texttt{EF} to determine trajectory cutoff. This can provide a better
  comparison with MD simulations.
  \begin{keytab}
    Type:    \> logical \\
    Default: \> \texttt{F} \\
    Range:   \> \texttt{T}, \texttt{F}
  \end{keytab}
\end{keydescription}

\begin{keydescription}{\texttt{LINEL} --- Inelastic scattering flag}
%
  This parameter specifies whether the local electronic energy loss is
  considered in the calculation of the scattering angles and energy
  transfer.
  \begin{keytab}
    Type:    \> logical \\
    Default: \> \texttt{F} \\
    Range:   \> \texttt{T}, \texttt{F}
  \end{keytab}
\end{keydescription}

\iffalse
\begin{keydescription}{\texttt{LMD} --- Molecular dynamics flag}
%
  This parameter specifies whether a molecular dynamics or a binary 
  collision simulation is to be performed. For \texttt{LMD=T} there must
  be one crystalline and no other region. In this region there must be
  only one target atom species.
  \begin{keytab}
    Type:    \> logical \\
    Default: \> \texttt{F} \\
    Range:   \> \texttt{T}, \texttt{F}
  \end{keytab}
\end{keydescription}
\fi

\iffalse
\begin{keydescription}{\texttt{LMEHL} --- Gauss-Mehler quadrature flag}
%
  This parameter specifies whether the scattering angles and the time
  integral shall be calculated by Gauss-Mehler quadrature using the ZBL
  interatomic potential. For \texttt{LMEHL=F} the scattering angles and time 
  integrals are calculated by interpolation in the scattering table (see
  \texttt{SCATFILE}).
  \begin{keytab}
    Type:    \> logical \\
    Default: \> \texttt{F} \\
    Range:   \> \texttt{T}, \texttt{F}
  \end{keytab}
\end{keydescription}
\fi

\begin{keydescription}{\texttt{LNEGX1} --- Enforce negative time integral shift flag}
%
  This flag, when set, enforces negative (or zero) time integral shifts along 
  the incoming asymptote. This means that in the rare case when the time 
  integral shift is positive, it is set to zero. Only relevant when the time 
  integral is considered, i.e., when the projectile energy is smaller than 
  \texttt{ETIME}. For channeling simulations \texttt{LNEGX1=T} is recommended, 
  otherwise nearly simultaneous collision may be missed.
  \begin{keytab}
    Type:    \> logical \\
    Default: \> \texttt{T} \\
    Range:   \> \texttt{T}, \texttt{F}
  \end{keytab}
\end{keydescription}

\begin{keydescription}{\texttt{LPMAXTAB} --- Maximum impact parameter table flag}
%
  This parameter specifies whether the maximum impact parameter as a function 
  of energy should be written to the \texttt{PMAX} file. The table lists the
  energy and the corresponding impact parameters in different regions (in terms
  of distance from the surface).
  \begin{keytab}
    Type:    \> logical \\
    Default: \> \texttt{F} \\
    Range:   \> \texttt{T}, \texttt{F}
  \end{keytab}
\end{keydescription}

\begin{keydescription}{\texttt{LPOSFFP} --- Enforce positive free flight paths flag}
%
  This flag, when set, enforces positive (or zero) free flight paths. Negative 
  free flight paths may occur when \texttt{LPOSFFP=F} if the time integral is 
  considered. For channeling simulations \texttt{LPOSFFP=F} is considered, 
  since this will more reliably catch nearly simultaneous collisions (double 
  scattering is avoided by \texttt{NLOLDM} of the \texttt{\&XTAL} record).
  \begin{keytab}
    Type:    \> logical \\
    Default: \> \texttt{F} \\
    Range:   \> \texttt{T}, \texttt{F}
  \end{keytab}
\end{keydescription}

\begin{keydescription}{\texttt{LSTFFP} --- Statistical free flight path flag}
%
  This parameter specifies whether the free flight paths are calculated 
  statistically (\texttt{LSTFFP=T}) or deterministically (\texttt{LSTFFP=F}). 
  In both cases, a free flight path is calculated from the parameters 
  \texttt{DENFFP}, \texttt{FFPMAX}, \texttt{PMAXMAX}, \texttt{PMAXMIN}, 
  \texttt{PMAXPSI}, \texttt{PSIFFP}, and the current projectile energy and 
  species. If \texttt{LSTFFP=F}, this free flight path is taken as is. If 
  \texttt{LSTFFP=T}, this value is taken as the mean value of an exponential
  distribution of the free flight path.
  \begin{keytab}
    Type:    \> logical \\
    Default: \> \texttt{T} \\
    Range:   \> \texttt{T}, \texttt{F}
  \end{keytab}
\end{keydescription}

\begin{keydescription}{\texttt{NABSC} --- Number of abscissae for quadrature
   of the scattering angle}
%
  This parameter specifies the number of abscissas to be used in the quadrature 
  for the calculation of the scattering angle in the center-of-mass system (if 
  \texttt{QUAD} specifies an integration method).
  \begin{keytab}
    Type:    \> index variable array (\texttt{ATOM2},\texttt{ATOM1}) 
                of integer \\
    Default: \> 4 \\
    Range:   \> $> 0$
  \end{keytab}
\end{keydescription}

\iffalse
\begin{keydescription}{\texttt{NABSCTIME} --- Number of abscissas for quadrature
   of the time integral}
%
  This parameter specifies the number of abscissas to be used in the quadrature 
  for the calculation of the time integral (if \texttt{QUADTIME} specifies an 
  integration method).
  \begin{keytab}
    Type:    \> index variable array (\texttt{ATOM2},\texttt{ATOM1}) 
                of integer \\
    Default: \> 4 \\
    Range:   \> $> 0$
  \end{keytab}
\end{keydescription}
\fi

\iffalse
\begin{keydescription}{\texttt{NMEHL} --- Number of abscissas in Gauss-Mehler 
    quadrature}
%
  This parameter specifies the number of abscissas in the Gauss Mehler 
  quadrature for the calculation of the scattering angle (if \texttt{LMEHL=T})
  and for the calculation of the time integral (if \texttt{E<ETIME}).
  \begin{keytab}
    Type:    \> integer \\
    Default: \> 32 \\
    Range:   \> $> 0$
  \end{keytab}
\end{keydescription}
\fi

\begin{keydescription}{\texttt{PMAXMAX} --- Maximum maximum impact
    parameter in amorphous material}
%
  This parameter specifies an upper limit to the maximum impact
  parameter in amorphous regions (regions with \texttt{XTAL='no'}).
  The actual maximum impact parameter is usually smaller according to
  {\tt DENFFP} and \texttt{PSIFFP}.  There may be another effective
  upper limit to the maximum impact parameter imposed by the
  scattering table used.  The units of \texttt{PMAXMAX} are \AA.
  \begin{keytab}
    Type:    \> real \\
    Default: \> 4 \\
    Range:   \> $>0$.
  \end{keytab}
\end{keydescription}

\begin{keydescription}{\texttt{PMAXMIN} --- Minimum maximum impact
    parameter in amorphous material}
%
  This parameter specifies a lower limit to the maximum impact
  parameter in amorphous regions (regions with \texttt{XTAL='no'}).
  The actual maximum impact parameter is usually larger according to
  {\tt DENFFP} and \texttt{PSIFFP}.  The units of \texttt{PMAXMIN} are \AA.
  \begin{keytab}
    Type:    \> real \\
    Default: \> 0.004 \\
    Range:   \> $>0$.
  \end{keytab}
\end{keydescription}

\begin{keydescription}{\texttt{PMAXPSI} --- Maximum maximum impact
    parameter for the angle criterion}
%
  This parameter specifies an upper limit to the maximum impact
  parameter as determined from the scattering angle criterion
  (\texttt{PSIFFP}).  If the scattering angle criterion gives a
  maximum impact parameter larger than \texttt{PMAXPSI}, it is reset
  to \texttt{PMAXPSI}. Note that the maximum impact parameter can be
  larger than \texttt{PMAXPSI} due to {\tt DENFFP}. 
  The units of \texttt{PMAXPSI} are \AA.  
  \begin{keytab}
    Type:    \> real \\
    Default: \> $1.53 / \min( 1,
                 \sqrt{0.15+0.85~\texttt{ENERGY}/10~\mathrm{keV}} )$ \\
    Range:   \> $>0$.
  \end{keytab}
\end{keydescription}

\begin{keydescription}{\texttt{PSIF} --- Final ion angle}
%
  This parameter specifies the cutoff angle of the ion in degrees, at
  which the simulation of the trajectory is terminated. It is measured
  with respect to the surface normal unless \texttt{LPINIT=T} on the
  \texttt{\&IONS} record, in which case it is measured with respect to
  the plane defined by the surface normal and \texttt{ROTATE} of the
  \texttt{\&IONS} record.  \texttt{PSIF=0} means that no cutoff angle
  has to be taken into account.
  \begin{keytab}
    Type:    \> real \\
    Default: \> 0 \\
    Range:   \> $0 \le \texttt{PSIF} \le 90$
  \end{keytab}
\end{keydescription}

\begin{keydescription}{\texttt{PSIFFP} --- Minimum scattering angle}
%
  This parameter specifies the minimum scattering angle in degrees which has
  to be considered in a collision. It is used for the calculation of the 
  free flight path in amorphous targets. It is chosen such that both 
  the energy transfer is larger than \texttt{DENFFP} and the scattering angle
  is larger than \texttt{PSIFFP}.
  \begin{keytab}
    Type:    \> real \\
    Default: \> 1 \\
    Range:   \> $0 < \texttt{PSIFFP} < 180$
  \end{keytab}
\end{keydescription}

\begin{keydescription}{\texttt{QUAD} --- Method of scattering angle quadrature}
%
  This parameter specifies the method of numerical quadrature for the
  calculation of the scattering angles in the center-of-mass system. 
  \texttt{'Legendre'} means Gauss-Legendre quadrature, \texttt{'Mehler'} 
  Gauss-Mehler quadrature (also known as Gauss-Chebyshev quadrature). 
  \texttt{'no'} means that the scattering angle is determined by 
  interpolation in the scattering table specified by \texttt{SCATFILE}.
  Either both or none of \texttt{QUAD} and \texttt{QUADTIME} must be 
  \texttt{'no'}.
  \begin{keytab}
    Type:    \> index variable array (\texttt{ATOM2},\texttt{ATOM1}) 
                of character strings \\
    Default: \> \texttt{'no'} \\
    Range:   \> \texttt{'Legendre'}, \texttt{'Mehler'}, \texttt{'no'}
  \end{keytab}
\end{keydescription}

\begin{keydescription}{\texttt{QUADTIME} --- Method of time integral quadrature}
%
  This parameter specifies the method of numerical quadrature for the
  calculation of the time integral. This parameter is currently unused.
  Time integrals are calculated using the scattering table \texttt{SCATTAB}. 
  \iffalse
  \texttt{'Legendre'} means Gauss-Legendre 
  quadrature, \texttt{'Mehler'} Gauss-Mehler quadrature (also known as 
  Gauss-Chebyshev quadrature). \texttt{'no'} means that the time integral is 
  determined by interpolation in the scattering table specified by 
  \texttt{SCATFILE}. 
  \begin{keytab}
    Type:    \> index variable array (\texttt{ATOM2},\texttt{ATOM1}) 
                of character(8) \\
    Default: \> \texttt{'no'} \\
    Range:   \> \texttt{'Legendre'}, \texttt{'Mehler'}, \texttt{'no'}
  \end{keytab}
  \fi
\end{keydescription}

\begin{keydescription}{\texttt{REFLECTMAX} --- Maximium number of surface 
reflections}
%
  This parameter specifies the maximum number of reflections at the surface
  potential. Such reflections may occur when a projectile tries to leave the 
  target but does not have enough energy to do so. When the number of surface 
  reflections of a projectile exceeds \texttt{REFLECTMAX}, the projectile is 
  stopped. 
  \begin{keytab}
    Type:    \> integer \\ 
    Default: \> $\infty$ \\
    Range:   \> $\ge 0$
  \end{keytab}
\end{keydescription}

\begin{keydescription}{\texttt{SCATFILE} --- Scattering table file}
%
  This parameter specifies a file containing tabulated scattering angles
  and time integrals. This model is used if 
  \texttt{QUAD='no'}. Then the scattering angles and time integrals are determined 
  by bicubic interpolation in the table contained in \texttt{SCATFILE}. Files 
  are assumed to be located in the subdirectory \texttt{'scattab'} of the 
  search path.
  \begin{keytab}
    Type:    \> index variable array (\texttt{ATOM2},\texttt{ATOM1}) 
                of character strings \\ 
    Default: \> \texttt{'BSI'}, \texttt{'SISI'}, \texttt{'PSI'}, 
                \texttt{'ASSI'}, \texttt{'BO'}, or \texttt{'ERSI'} for the 
                appropriate \\
             \> atom combinations, \\
             \> \texttt{'SCATTAB'} otherwise \\
    Range:   \> in addtion to the possible default values, 
                \texttt{'SCATTAB\_big'}, or the name \\
             \> ($\le$ 80 characters) of a user-supplied file containing a scattering table 
  \end{keytab}
\end{keydescription}

\begin{keydescription}{\texttt{SCRFILE} --- Screening function file}
%
  This parameter specifies a file containing tabulated values of the
  screening function. This model is used if \texttt{QUAD} specifies a 
  numerical integration method and 
  \texttt{COEFFILE} is either unspecified or overriden by specification of
  \texttt{SCRFILE}. \texttt{COEFFILE} and \texttt{SCRFILE} must not 
  appear on the same input record. Specification of \texttt{COEFFILE} or 
  \texttt{SCRFILE} on later input records override those on earlier ones. 
  \texttt{SCRFILE='ZBL'} and \texttt{SCRFILE='DMol'} are interpreted as 
  \texttt{SCRFILE='zbl/XXYY'} and \texttt{SCRFILE='dmol/XXYY'},
  respectively, where \texttt{XX} and \texttt{YY} are replaced by the chemical 
  element names of the respective atoms. Only atom combinations with the 
  atomic number of \texttt{XX} smaller than or equal to the atomic number of 
  \texttt{YY} are available. Files are assumed to be located in the subdirectory 
  \texttt{'screentab'} of the search path.
  \begin{keytab}
    Type:    \> index variable array (\texttt{ATOM2},\texttt{ATOM1}) 
                of character strings \\ 
    Default: \> \texttt{'ZBL'} \\
    Range:   \> \texttt{'ZBL'}, \texttt{'DMol'}, 
                \texttt{'zbl/XXYY'}, \texttt{'dmol/XXYY'} 
                (for the meaning of \texttt{XXYY} see \\
             \> above), or the name ($\le$ 80 characters) of a user-supplied
                file containing  \\
             \> a screening function table 
  \end{keytab}
\end{keydescription}

\begin{keydescription}{\texttt{SCRFUN} --- Screening function}
%
  This parameter specifies the screening function for scattering angle
  calculation. Valid choices are either
  built-in functions (\texttt{'ZBL'}, \texttt{'KrC'}, \texttt{'Moliere'}),
  \texttt{'coeffile'} as a reference to a coefficients file, or 
  \texttt{'scrfile'} as a reference to a screening function file. In the two
  latter cases, the file is specified by \texttt{COEFFILE} and \texttt{SCRFILE},
  respectively. \texttt{SCRFUN} is ignored if \texttt{QUAD='no'}
  \begin{keytab}
    Type:    \> index variable array (\texttt{ATOM2},\texttt{ATOM1}) 
                of character strings \\ 
    Default: \> \texttt{'ZBL'} \\
    Range:   \> \texttt{'ZBL'}, \texttt{'KrC'}, \texttt{'Moliere'},
    			\texttt{'coeffile'}, \texttt{'scrfile'}
  \end{keytab}
\end{keydescription}

\begin{keydescription}{\texttt{SCRLEN} --- Screening length}
%
  This parameter specifies the model for the screening length for the
  calculation of the scattering angle. The model specifies the dependence 
  on the atomic numbers of the projectile and the target atom. 
  \texttt{SCRLEN='1'} means a screening length of $1~\rm\AA$. For the
  calculation of the time integral, always the \texttt{ZBL} screening
  length is used.
  \begin{keytab}
    Type:    \> index variable array (\texttt{ATOM2},\texttt{ATOM1}) 
                of character strings \\ 
    Default: \> \texttt{'1'} for \texttt{SCRFUN='scrfile'} \\
    		 \> \texttt{'ZBL'} otherwise \\
    Range:   \> \texttt{'1'}, \texttt{'ZBL'}, \texttt{'Firsov'}, \texttt{'Lindhard'}
  \end{keytab}
\end{keydescription}

